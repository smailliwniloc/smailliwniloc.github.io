\documentclass[10pt,a4paper]{article}
\usepackage[utf8]{inputenc}
\usepackage[english]{babel}
\usepackage{csquotes}
\usepackage{amsmath}
\usepackage{amsfonts}
\usepackage{amssymb}
\usepackage{graphicx}
\usepackage[margin=0.5in]{geometry}
\usepackage{amsthm}
\usepackage{enumitem}
\usepackage{tikz}
\usetikzlibrary{calc}
\newtheorem{question}{Question}
\newtheorem*{question*}{Question}
\newtheorem{theorem}{Theorem}
\newtheorem*{theorem*}{Theorem}
\newtheorem{lemma}{Lemma}
\newtheorem{proposition}{Proposition}
\newtheorem{corollary}{Corollary}

\theoremstyle{definition}
\newtheorem{answer}{Answer}
\newtheorem*{answer*}{Answer}

\theoremstyle{definition}
\newtheorem{verify}{Verification}
\newtheorem*{verify*}{Verification}

\numberwithin{equation}{section}
\numberwithin{theorem}{section}
\numberwithin{proposition}{section}
\numberwithin{lemma}{section}
\numberwithin{corollary}{section}


\title{Analysis Theorems}
\author{Colin Williams}

\begin{document}
\maketitle

\section{Topology Review}

\begin{theorem}$ $
\\$Y$ is closed if and only if $Y^c$ is open.
\end{theorem}

\begin{theorem}$ $
\begin{enumerate}
\item The complement of arbitrary unions is the arbitrary intersection of complements.
\item The arbitrary union of open sets is open and the finite intersection of open sets is is open.
\item The arbitrary intersection of closed sets is closed and the finite union of closed sets is closed.
\end{enumerate}
\end{theorem}

\begin{theorem}$ $
\\Let $Y \subset X$, then $\overline{Y}$ is closed and if $Z \subset X$ is closed with $Y \subset Z$, then $\overline{Y} \subset Z$.
\end{theorem}

\begin{proposition}$ $
\\Every compact set is closed.
\end{proposition}

\begin{proposition}$ $
\\If $Y$ is bounded, then $\forall \; x \in X$, there exists $r_x > 0$ such that $Y \subset U_{r_x}(x)$. 
\end{proposition}

\begin{proposition}$ $
\\$Y$ is bounded if and only if diam$(Y) < +\infty$.
\end{proposition}

\begin{proposition}$ $
\\diam$(\overline{Y}) = $ diam$(Y)$
\end{proposition}

\begin{proposition}$ $
\\Every compact set is bounded. 
\end{proposition}

\begin{theorem}[Heine-Borel Theorem]$ $
\\If $(X, d)$ is $\mathbb{R}^n$ with the standard Euclidean metric, then $Y \subset X$ is compact if and only if $Y$ is closed and bounded. 
\end{theorem}

\begin{theorem}[Relativity]$ $
Let $Y \subset X$, then
\begin{enumerate}
\item Let $Z \subset Y$, then $Z$ is open in $Y$ if and only if $Z = Y \cap G$ where $G$ is open in $X$
\item Let $Z \subset Y$, then $Z$ is compact in $Y$ if and only if $Z$ is compact in $X$
\end{enumerate}
\end{theorem}


\newpage
\section{Convergence}

\begin{proposition}$ $
\\If $(x_n)$ is convergent, then $(x_n)$ is bounded. 
\end{proposition}

\begin{proposition}$ $
\\$\{x_n\}' \subset (x_n)^*$
\end{proposition}

\begin{lemma}$ $
\\Let $S \subset X$, suppose $S' \neq \emptyset$, let $x \in S'$, then $\forall \; r > 0$, $U_r(x) \cap S$ is infinite
\end{lemma}

\begin{proposition}$ $
\\The set $(x_n)^*$ of subsequential limits of $(x_n)$ is closed in $X$. 
\end{proposition}

\begin{proposition}$ $
\\Every Cauchy Sequence is bounded. 
\end{proposition}

\begin{proposition}$ $
\\Every convergent sequence is Cauchy
\end{proposition}

\begin{proposition}$ $
\\If $(X, d)$ is a complete metric space, and $Y \subset X$, then $(Y, d)$ is complete if and only if $Y$ is closed in $X$.
\end{proposition}

\begin{theorem}$ $
\\If $X$ is compact, then $\forall \; (x_n) \subset X$, $(x_n)^* \neq \emptyset$. In other words, every sequence in a compact metric space has a convergent subsequence. 
\end{theorem}

\begin{theorem}$ $
\\Every compact metric space $(X, d)$ is complete.
\end{theorem}

\begin{lemma}$ $
\\Suppose $Y \subset X$, then $Y$ is compact if and only if $Y$ is closed in $X$.
\end{lemma}

\begin{lemma}$ $
\\Let $\{K_n\}_{n \in \mathbb{N}}$ be a countable family of compact non-empty subsets of $X$ such that $K_1 \supset K_2 \supset K_3 \supset \cdots $ and diam$(K_n) \to 0$ as $n \to \infty$, then there exists some $x \in X$ such that 
\begin{align*}
\bigcap_{n = 1}^\infty K_n = \{x\}.
\end{align*}
\end{lemma}

\begin{theorem}$ $
\\Every monotonic sequence in $\mathbb{R}$ is convergent (possible to $\pm \infty$). 
\end{theorem}

\begin{corollary}$ $
\\$\emptyset \neq (x_n)^* \subset [-\infty, \infty]$ for every sequence in $\mathbb{R}$. 
\end{corollary}

\begin{proposition}$ $
\\$\sup(x_n)^*, \inf(x_n)^* \in (x_n)^*$
\end{proposition}

\begin{theorem}$ $
\begin{align*}
\limsup_{n \to \infty} x_n = \sup(x_n)^* && \text{and} && \liminf_{n \to \infty} x_n = \inf(x_n)^*
\end{align*}
\end{theorem}

\begin{corollary}$ $
\\The following are equivalent:
\begin{enumerate}
\item $\lim_{n \to \infty} x_n = x$
\item $(x_n)^* = \{x\}$
\item $\limsup_{n \to \infty} x_n = \liminf_{n \to \infty} x_n = x$
\end{enumerate}
\end{corollary}

\begin{theorem}[Comparison Test]$ $
\\If $\forall \; n \in \mathbb{N}$, $0 \leq x_n \leq y_n$, then \begin{align*}
\sum_{n = 1}^\infty x_n \leq \sum_{n = 1}^\infty y_n
\end{align*}
\end{theorem}

\begin{theorem}[Root Test]$ $
\\If we define 
\begin{align*}
r := \limsup_{n \to \infty} \left( \sqrt[n]{x_n} \right)
\end{align*}
Then we have
\begin{align*}
\sum_{n = 1}^\infty x_n \begin{cases}
< +\infty &\text{if $r < 1$}\\
? &\text{if $r = 1$}\\
= +\infty &\text{if $r > 1$}
\end{cases}
\end{align*}
\end{theorem}

\begin{proposition}[Root vs. Ratio]$ $
\\Suppose for all $n \in \mathbb{N}$, $x_n > 0$, then
\begin{align*}
\liminf_{n \to \infty} \frac{x_{n + 1}}{x_n} \leq \liminf_{n \to \infty} \sqrt[n]{x_n} \leq \limsup_{n \to \infty} \sqrt[n]{x_n} \leq \limsup_{n \to \infty} \frac{x_{n+1}}{x_n}
\end{align*}
\end{proposition}


\begin{theorem}[Dirichlet Test]$ $
\\Suppose the following:
\begin{enumerate}
\item $\displaystyle \sup\left\{\left| \sum_{n = 1}^N y_n \right| : N \in \mathbb{N}\right\} < +\infty$
\item $\forall \; n \in \mathbb{N}, x_n \geq x_{n + 1}$
\item $\displaystyle \lim_{n \to \infty} x_n = 0$
\end{enumerate}
Then, $\displaystyle \sum_{n = 1}^\infty x_n y_n$ is convergent.
\end{theorem}

\begin{theorem}[Rearrangement Theorem]$ $
\\If we have a bijective map $k \mapsto n_k$, and 
\begin{align*}
S = \sum_{n = 1}^\infty x_n && S' = \sum_{k = 1}^\infty x_{n_k}.
\end{align*}
Then, 
\begin{align*}
\sum_{n = 1}^\infty |x_n| < +\infty \implies \sum_{k = 1}^\infty |x_{n_k}| < +\infty \qquad \text{ and } \qquad S = S'
\end{align*}
\end{theorem}

\begin{theorem}$ $
\\If we have convergence in the previous theorem, but not absolute convergence, then we can have $S'$ converge to whatever value we want.
\end{theorem}

\newpage
\section{Continuity}
\begin{proposition}$ $
\\If $f: X \to Y$ is continuous at $a$ and $g: Y \to Z$ is continuous at $y = f(a)$, then $g \circ f: X \to Z$ is continuous at $x = a$. 
\end{proposition}

\begin{proposition}$ $
\\$f$ is continuous if and only if $\forall \; G \subset Y$ which are open in $Y$, then $f^{-1}(G)$ is also open in $X$. 
\end{proposition}

\begin{theorem}$ $
\\If $X$ is compact and $f$ is continuous, then $f(X)$ is compact. 
\end{theorem}

\begin{corollary}$ $
\\If $f: X \to Y$ for $X$ and $f$ as above and $Y = \mathbb{R}$, then $f$ attains its maximum and minimum on $X$. 
\end{corollary}

\begin{theorem}$ $
\\If $X$ is compact metric space, $f: X \to Y$ is a continuous bijection, then $Y$ is compact and $f^{-1}: Y \to X$ is also continuous.
\end{theorem}

\begin{theorem}$ $
\\If $X$ is compact and $f$ is continuous, then $f: X \to Y$ is uniformly continuous. 
\end{theorem}

\begin{theorem}$ $
\\If $(X, d)$ is a connected metric space and $f: X \to Y$ is continuous, then $f(X)$ is connected. 
\end{theorem}

\begin{corollary}$ $
\\If $X$ and $f$ are as above, then $f(X)$ is an interval.
\end{corollary}


\newpage
\section{Differentiability}

\begin{theorem}[Fermat]$ $
\\If $f: (a, b) \to \mathbb{R}$ and $c \in (a, b)$ such that $f(c) = \max\{f(a, b)\}$ or $f(c) = \min\{f(a, b)\}$, then if $f'(c)$ exists, we must have $f'(c) = 0$. 
\end{theorem}

\begin{theorem}[Rolle's Theorem]$ $
\\Suppose $f: [a, b] \to \mathbb{R}$ is continuous and that $f: (a, b) \to \mathbb{R}$ is differentiable such that $f(a) = f(b)$, then there exists some $c \in (a, b)$ such that $f'(c) = 0$. 
\end{theorem}

\begin{theorem}[Lagrange's Mean Value Theorem]$ $
\\Assume that $f: [a, b] \to \mathbb{R}$ is continuous and $f: (a, b) \to \mathbb{R}$ is differentiable, then there exists some $c \in (a, b)$ such that 
\begin{align*}
f'(c) = \frac{f(b) - f(a)}{b - a}
\end{align*}
\end{theorem}

\begin{corollary}$ $
\\If $f:(a, b) \to \mathbb{R}$ is differentiable and $f'(x) = 0$ for all $x \in (a, b)$, then $f$ is constant. 
\end{corollary}

\begin{theorem}$ $
\\Suppose $f: U_r(x_0) \to \mathbb{R}$ is continuous and that $f: \overset{\circ}{U_r}(x_0) \to \mathbb{R}$ is differentiable and $\lim_{x \to x_0} f'(x)$ exists, then $f'(x_0) = \lim_{x \to x_0} f'(x)$
\end{theorem}

\begin{theorem}[Generalized MVT]$ $
\\Let $f, g: [a, b] \to \mathbb{R}$ be continuous and $f, g: (a, b) \to \mathbb{R}$ be differentiable. Assume $g'(x) \neq 0$ for all $x \in (a, b)$, then 
\begin{align*}
\frac{f(b) - f(a)}{g(b) - g(a)} = \frac{f'(c)}{g'(c)} \qquad \text{ for some } c \in (a, b)
\end{align*}
\end{theorem}

\begin{theorem}[Taylor's Formula]$ $
\\If $f: U_r(x_0) \to \mathbb{R}$ is $(n + 1)$-times differentiable, then 
\begin{align*}
f(x) = f(x_0) + \frac{f'(x_0)}{1!} (x - x_0) + \frac{f''(x_0)}{2!}(x - x_0)^2 + \cdots + \frac{f^{(n)}(x_0)}{n!}(x - x_0)^n + \frac{f^{(n+1)}(c)}{(n+1)!}(x - x_0)^{n+1}
\end{align*}
for some $c$ between $x_0$ and $x$. 
\end{theorem}


\newpage
\section{Sequences of Functions}
\begin{theorem}$ $
\\A sequence of functions $(f_n)$ is uniformly convergent if and only if it is Cauchy. 
\end{theorem}

\begin{theorem}$ $
If $(f_n)$ is a sequence of continuous functions which is Cauchy, then $\lim_{n \to \infty} f_n$ is also continuous. 
\end{theorem}

\begin{theorem}$ $
Suppose the following:
\begin{enumerate}
\item $(X, d)$ is compact
\item $(f_n) \subset C(X)$
\item $f = \lim_{n \to \infty} f_n$ is continuous
\item For all $x \in X$, $f_1(x) \geq f_2(x) \geq \cdots $
\end{enumerate}
Then, $f_n \to f$ uniformly as $n \to \infty$. 
\end{theorem}

\begin{theorem}[Weierstrass Theorem]$ $
\\Let $f: [a, b] \to \mathbb{R}$ be continuous. Then $\forall \; r > 0$, there exists some polynomial $P(x)$ such that 
\begin{align*}
||f - P|| = \sup\{|f(x) - P(x)| : x \in [a, b]\} < r
\end{align*}
\end{theorem}

\begin{theorem}
Let $X = (a, b)$ and suppose $f_n : X \to \mathbb{R}$ is such that 
\begin{enumerate}
\item For all $n \in \mathbb{N}$, $f_n$ is differentiable.
\item $(f_n')$ is a Cauchy Sequence.
\item There exists an $x \in X$ such that $(f_n(x))$ is convergent. 
\end{enumerate}
Then, $(f_n)$ is a Cauchy sequence and $\lim_{n \to \infty} f_n(x)$ is differentiable with derivative $\frac{d}{dx}[\lim_{n \to \infty} f_n(x)] = \lim_{n \to \infty} f_n'(x)$
\end{theorem}

\begin{proposition}[Homework Assignment]$ $
\\If $(X, d)$ is compact and $(f_n) \subset C(X)$ with 
\begin{align*}
\sum_{n = 1}^\infty |f_n(x)| \in C(X)
\end{align*}
Then, $\sum_{n = 1}^\infty f_n(x)$ is uniformly convergent. 
\end{proposition}

\begin{theorem}[Arzela-Ascoli]$ $
\\Let $(X, d)$ be compact and let $f_n : X \to \mathbb{R}$ be an equicontinuous, pointwise bounded sequence. Then $f_n$ is uniformly bounded and there is a subsequence $f_{n_k}$ that converges uniformly. 
\end{theorem}

\begin{theorem}[Adaptation of Arzela-Ascoli]$ $
\\Let $(X, d)$ be compact and suppose $f_n : X \to \mathbb{R}$ is equicontinuous and pointwise convergent, then $f_n$ is uniformly convergent and $\lim_{n \to \infty} f_n(x)$ is uniformly continuous on $X$. 
\end{theorem}


\newpage 


\section{Multivariable Functions}

\begin{theorem}$ $
\\Let $X \subset \mathbb{R}^n$ be open, then $f: X \to \mathbb{R}^m$ is continuously differentiable if and only if $D_1f, D_2f, \ldots, D_nf: X \to \mathbb{R}^m$ are all continuous. 
\end{theorem}


\begin{theorem}[Chain Rule]$ $
\\Let $X \subset \mathbb{R}^n$ and $Y \subset \mathbb{R}^m$ be open. Let $f: X \to \mathbb{R}^m$ be differentiable at some $x_0 \in X$. Suppose $y_0 = f(x_0) \in Y$ and that $g: Y \to \mathbb{R}^k$ is differentiable at $y_0$. Then $g \circ f: X \to \mathbb{R}^k$ is differentiable at $x_0$ with $(g \circ f)'(x_0) = g'(f(x_0))f'(x_0)$
\end{theorem}

\begin{lemma}$ $
\\Let $I_n \in L(\mathbb{R}^n, \mathbb{R}^n)$ be such that $I_n x = x \; \forall \; x \in \mathbb{R}^n$. Let $A \in L(\mathbb{R}^n, \mathbb{R}^n)$ be such that $||A|| < 1$. Then $I_n - A$ is invertible and 
\begin{align*}
||(I - A)^{-1}|| \leq \frac{1}{1 - ||A||}
\end{align*}
\end{lemma}

\begin{lemma}$ $
\\Let $\gamma: GL(\mathbb{R}^n) \to GL(\mathbb{R}^n)$. Then $\gamma(A) = A^{-1}$ is continuous and $GL(\mathbb{R}^n)$ is open in the space $L(\mathbb{R}^n, \mathbb{R}^n)$. 
\\
\\\textit{Note that $GL(X)$ represents the invertible linear transformations from $X \to X$}
\end{lemma}

\begin{lemma}[Contraction Principle]$ $
\\Let $(X, d)$ be a complete metric space. Suppose that $\varphi: X \to X$ is a (strict) contraction, i.e. $\exists \; C \in (0, 1)$ such that $\forall \; x, y \in X, d(\varphi(x), \varphi(y)) \leq Cd(x, y)$. Then, there exists a unique fixed point $x \in X$ such that $\varphi(x) = x$. 
\end{lemma}

\begin{lemma}$ $
\\Let $X \subset \mathbb{R}^n$ be open and convex. Let $f: X \to \mathbb{R}^m$ be differentiable. Suppose $\exists \; M > 0$ such that $||f'(x)|| \leq M$ for all $x \in X$. Then, $\forall \; a, b \in X, |f(b) - f(a)| \leq M|b - a|$
\end{lemma}

\begin{theorem}[Inverse Function Theorem]$ $
\\Let $X \subset \mathbb{R}^n$ be open and $f: X \to \mathbb{R}^n$ be continuously differentiable with $x_0 \in X$ and $f'(x_0) \in GL(\mathbb{R}^n)$. Then, there exists an open $U \subset \mathbb{R}^n$ with $x_0 \in U \subset X$ and such that 
\begin{enumerate}
\item $f: U \to \mathbb{R}^n$ is injective. 
\item $V = f(U)$ is open in $\mathbb{R}^n$.
\item $f^{-1}: V \to \mathbb{R}^n$ is continuously differentiable.
\item $(f^{-1})'(f(x)) = (f^{-1}(x))^{-1}$ for all $x \in U$. 
\end{enumerate}
\end{theorem}

\begin{theorem}[Implicit Function Theorem]$ $
\\Let $X \subset \mathbb{R}^{n+m}$ be open and $f: X \to \mathbb{R}^n$ be continuously differentiable. Denote $[f'(x, y)](\alpha, \beta) = [f'(x, y)](\alpha, 0) + [f'(x, y)](0, \beta) = f_x'(x, y)\alpha + f_y'(x, y)\beta$ where $f_x'(x, y) \in L(\mathbb{R}^n, \mathbb{R}^n)$ and $f_y'(x, y) \in L(\mathbb{R}^m, \mathbb{R}^n$.
\\
\\Suppose $(x_0, y_0) \in X$ such that $f_x'(x_0, y_0) \in GL(\mathbb{R}^n)$. Then, there exists open sets $U, V$ with $U \subset X$ and $V \subset \mathbb{R}^m$ such that $(x_0, y_0) \in U, y_0 \in V$. 
\\
\\Assume $f(x_0, y_0) = 0$, then there exists a unique $g: V \to \mathbb{R}^n$ which is continuously differentiable such that $\forall \; y \in V$, $(g(y), y) \in U$ and $f(g(y), y) = 0$ for $y \in V$ and $g(y_0) = x_0$. Furthermore,
\begin{align*}
g'(y_0) = -f_x'(x_0, y_0)^{-1}f_y'(x_0, y_0)
\end{align*}
\end{theorem}

\begin{proposition}$ $
\\Let $X \subset \mathbb{R}^2$ be open. Let $f: X \to \mathbb{R}$ be such that $D_{21}f$ is defined and in $X$. Let $(x_0, y_0) \in X$ and let $a, b > 0$ be such that $[x_0, x_0 + a] \times [y_0, y_0 + b] \subset X$, then $\exists \; x_1 \in (x_0, x_0 + a), y_1 \in (y_0, y_0 + b)$ such that 
\begin{align*}
f(x_0 + a, y_0 + b) - f(x_0, y_0 + b) - f(x_0 + a, y_0) + f(x_0, y_0) = abD_21 f(x_1, y_1)
\end{align*}
\end{proposition}


\begin{theorem}$ $
\\Let $X \subset \mathbb{R}^2$ be open, and $f: X \to \mathbb{R}$ be such that $D_{21}f, D_2f$ are defined in $X$. Let $(x_0, y_0) \in X$ be such that $D_{21}f$ is continuous at $(x_0, y_0)$, then $D_{12}f(x_0, y_0)$ exists and $D_{12}f(x_0, y_0) = D_{21}f(x_0, y_0)$
\end{theorem}


\begin{corollary}[Taylor's Linearization]$ $
\\The following linearization holds in a sufficiently small neighborhood:
\begin{align*}
f(x_0 + a, y_0 + b) = f(x_0, y_0) + D_1f(x_0, y_0)a + D_2f(x_0, y_0)b + \frac{D_{11}f(x_0, y_0)}{2}a^2 + \frac{D_{22}f(x_0, y_0)}{2}b^2 + D_{12}f(x_0, y_0)ab + o(a^2 + b^2)
\end{align*}
\end{corollary}

\begin{theorem}[HW 8 $\#$1]$ $
\\Let $f \in \mathbf{L}(\mathbb{R}^n, \mathbb{R}^m)$. Then, $f$ is injective if and only if $\ker(f) = \{0\}$. 
\end{theorem}

\begin{theorem}[HW 8 $\#$3]$ $
\\Let $X$ be an open subset of $\mathbb{R}^n$ and let $f: X \to \mathbb{R}^m$ be such that $D_1f, D_2f, \ldots, D_nf$ are defined and bounded in $X$. Then, $f$ is continuous. 
\end{theorem}

\begin{theorem}[HW 8 $\#$4]$ $
\\Let $X$ be an open subset of $\mathbb{R}^n$ and let $f: X \to \mathbb{R}$ be differentiable. Suppose that $x_0 \in X$ is a point of maximum of $f$, then $f'(x_0) = 0$. 
\end{theorem}

\begin{theorem}[HW 9 $\#$1]$ $
\\Let $X \subset \mathbb{R}^n$ be open. Let $x_0 \in X$ and let $f: X \to \mathbb{R}^m$ be differentiable at $x_0$. Let $Y \subset \mathbb{R}^m$ be open and such that $f(x_0) \in Y$. Finally, let $g: Y \to \mathbb{R}^k$ be differentiable at $f(x_0)$. Then,
\begin{align*}
D_j(g \circ f)(x_0) = \sum_{i = 1}^m (D_i g)(f(x_0))(D_jf^i)(x_0)
\end{align*}
where $f^i(x)$ is the $i$-th component of $f(x)$. 
\end{theorem}

\begin{theorem}[HW 9 $\#$2]$ $
\\Let $X \subset \mathbb{R}^n$ be open and let $f: X \to \mathbb{R}^n$ be continuously differentiable and such that $f'(x)$ is invertible for every $x \in X$. Then $f(X)$ is open in $\mathbb{R}^n$. 
\end{theorem}

\begin{theorem}[HW 10 $\#$1]$ $
\\Let $X \subset \mathbb{R}^n$ be an open connected set. Let $f: X \to \mathbb{R}^n$ be differentiable and such that $f'(x) = 0$ for all $x \in X$. Then, $f$ is a constant function. 
\end{theorem}

\begin{theorem}[HW 11 $\#$2]$ $
\\If $f: \mathbb{R}^2 \to \mathbb{R}$ is continuously differentiable, then $f$ is not injective. 
\end{theorem}

\newpage

\section{Integration}

\begin{proposition}$ $
\\Let $\{Y_{\alpha}\} \subset 2^X$, then there is a unique $\Sigma$ that is a minimal $\sigma$-algebra such that $\{Y_{\alpha}\} \subset \Sigma$. 
\\
\\This $\Sigma$ is called the $\sigma$-completion of $\{Y_{\alpha}\}$
\end{proposition}

\begin{lemma}$ $
\\If $\mu: \Sigma \to [0, +\infty]$ is a positive measure on $X$, then the following properties hold:
\begin{enumerate}
\item $\mu(\emptyset) = 0$
\item If $Y_1, Y_2, \ldots, Y_N \in \Sigma$ with $m \neq n \implies Y_m \cap Y_n = \emptyset$, then 
\begin{align*}
\mu\left(\bigcup_{n = 1}^N Y_n \right) = \sum_{n = 1}^N \mu(Y_n)
\end{align*}
\item If $Y_1, Y_2 \in \Sigma$ are such that $Y_1 \subset Y_2$, then $\mu(Y_1) \leq \mu(Y_2)$.
\item If $\{Y_n : n \in \mathbb{N}\}$ are nested subsets where $Y_1 \subset Y_2 \subset Y_3 \subset \cdots$, then 
\begin{align*}
\mu\left(\bigcup_{n = 1}^N Y_n \right) = \lim_{n \to \infty} \mu(Y_n)
\end{align*}
\end{enumerate}
\end{lemma}

\begin{proposition}$ $
\\Let $(X, \Sigma, \mu)$ be a measure space and let $(Y, d_Y), (Z, d_Z)$ be metric spaces. Let $f: X \to Y$ be measurable and let $g: Y \to Z$ be continuous, then $g \circ f: X \to Z$ is measurable. 
\end{proposition}

\begin{theorem}$ $
\\Suppose $(X, \Sigma, \mu)$ is a measure space. Let $f, g: X \to \mathbb{R}$ be measurable and consider $F: \mathbb{R}^2 \to \mathbb{R}$ some continuous function. Then $F(f(x), g(x))$ is measurable
\end{theorem}

\begin{proposition}$ $
\\Let $(f_n(x))$ be a sequence functions such that $f_n : X \to [-\infty, +\infty]$ is measurable for all $n \in \mathbb{N}$. Then, all of the following are measurable:
\begin{itemize}
\item $a(x) = \sup\{f_n(x) : n \in \mathbb{N}\}$
\item $b(x) = \inf\{f_n(x) : n \in \mathbb{N}\}$
\item $c(x) = \limsup_{n \to \infty} f_n(x)$
\item $d(x) = \liminf_{n \to \infty} f_n(x)$
\end{itemize}
\end{proposition}

\begin{proposition}$ $
\\Let $f: X \to [0, +\infty]$ be measurable, then $\exists$ simple, measurable functions $s_n: X \to \mathbb{R}$ such that $\forall \; x \in X$
\begin{align*}
0 \leq s_1(x) \leq s_2(x) \leq s_3(x) \leq \cdots \leq f(x) = \lim_{n \to \infty} s_n(x) = \sup_{n \in \mathbb{N}} s_n(x)
\end{align*}
\end{proposition}

\begin{proposition}$ $
\begin{align*}
\int_X \sum_{n = 1}^N s_n d \mu = \sum_{n = 1}^N \int_X s_n d \mu
\end{align*}
\end{proposition}

\begin{theorem}$ $
\\Let $f: X \to [0, +\infty]$ be measurable. Define $\nu_f : \Sigma \to [0, +\infty]$ by
\begin{align*}
\nu_f(Y) = \int_Y f d \mu
\end{align*}
then $\nu_f$ is a positive measure. 
\end{theorem}

\begin{theorem}[Monotone Convergence Theorem]$ $
\\Suppose $(f_n)$ is a sequence of functions such that for all $n \in \mathbb{N}$, $f_n: X \to [0, +\infty]$ is measurable and $$f_1 \leq f_2 \leq f_3 \leq \cdots \leq f = \lim_{n \to \infty} f_n = \sup_{n \in \mathbb{N}} f_n$$ Then, $$\int_X f d\mu = \lim_{n \to \infty} \int_X f_n d\mu$$
\end{theorem}

\begin{lemma}[Fatou Lemma]$ $
\\Let $f_n : X \to [0, +\infty]$ be measurable, then
\begin{align*}
\int_X \liminf_{n \to \infty} f_n d\mu \leq \liminf_{n \to \infty} \int_X f_n d\mu
\end{align*}
\end{lemma}

\begin{proposition}[Additivity]$ $
\\Let $f, g: X \to \mathbb{R}$ be in $\mathcal{L}(\mu)$. Then, $f + g \in \mathcal{L}(\mu)$ with 
\begin{align*}
\int_X f + g d\mu = \int_X f d\mu + \int_X g d\mu
\end{align*}
\end{proposition}

\begin{proposition}[Triangle Inequality]$ $
\begin{align*}
\left| \int_X f d\mu \right| \leq \int_X |f| d\mu
\end{align*}
\end{proposition}

\begin{theorem}[Dominated Convergence]$ $
\\Suppose $f_n : X \to [-\infty, +\infty]$ are measurable functions such that 
\begin{enumerate}
\item $\exists \; g \in \mathcal{L}(\mu)$ such that $|f_n| \leq g$ for all $n \in \mathbb{N}$. 
\item $f_n \to f$ pointwise as $n \to \infty$.
\end{enumerate}
Then, $f_n \in \mathcal{L}(\mu), f \in \mathcal{L}(\mu)$ and 
\begin{align*}
\int_X f_n d\mu \to \int_X f d \mu \qquad \text{as $n \to \infty$}
\end{align*}
\end{theorem}

\begin{lemma}$ $
\\For $X \subset \mathbb{R}$ and $(X, \Sigma, \mu)$ a measure space, define the Outer Measure as
\begin{align*}
\mu^*(X) = \inf\{ \sum_{n = 1}^\infty \mu(G_n) : G_n \in \Sigma_0, G_n open, X \subset \bigcup_{n = 1}^\infty G_n \}
\end{align*}
where $\Sigma_0$ is the collection of unions of intervals (open, closed, or half-open) of $\mathbb{R}$. Then $\mu^*(X) = \mu(X)$ for all $X \in \Sigma_0$. 
\end{lemma}

\begin{theorem}$ $
\\Let $(\mathbb{R}, \Sigma, \mu)$ be the Lebesgue Measure Space. Let $f: [a, b] \to \mathbb{R}$ be bounded. Then $f$ is Riemann Integrable if and only if $\mu(\{x \in [a, b] : f$ is discontinuous at $x\}) = 0$. In this case, $f \in \mathcal{L}(\mu)$ and $\int_a^b f dx = \int_{[a, b]} f d\mu$ where the first is the Riemann Integral and the second is the Lebesgue Integral. 
\end{theorem}

\begin{theorem}[HW 12 $\#$1]$ $
\\Let $Y_n$ be a sequence in $\Sigma$ such that $Y_1 \supset Y_2 \supset Y_3 \supset \cdots$. If $\mu(Y_1) < +\infty$, then
\begin{align*}
\mu\left(\bigcap_{n \in \mathbb{N}} Y_n \right) = \lim_{n \to \infty} \mu(Y_n)
\end{align*}
\end{theorem}

\begin{theorem}[HW 12 $\#$3]$ $
\\If $f, g: X \to [-\infty, +\infty]$ are two measurable functions, then $\{x: f(x) = g(x)\}$ is a measurable set. 
\end{theorem}

\end{document}