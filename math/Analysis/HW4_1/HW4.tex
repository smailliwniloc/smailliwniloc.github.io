\documentclass[10pt,a4paper]{article}
\usepackage[utf8]{inputenc}
\usepackage[english]{babel}
\usepackage{csquotes}
\usepackage{amsmath}
\usepackage{amsfonts}
\usepackage{amssymb}
\usepackage{graphicx}
\usepackage[margin=0.5in]{geometry}
\usepackage{amsthm}
\usepackage{enumitem}
\usepackage{tikz}
\usetikzlibrary{calc}
\newtheorem{question}{Question}
\newtheorem*{question*}{Question}
\newtheorem{theorem}{Theorem}
\newtheorem*{theorem*}{Theorem}
\newtheorem{lemma}{Lemma}

\theoremstyle{definition}
\newtheorem{answer}{Answer}
\newtheorem*{answer*}{Answer}

\theoremstyle{definition}
\newtheorem{verify}{Verification}
\newtheorem*{verify*}{Verification}

\numberwithin{equation}{section}


\title{Analysis HW 4}
\author{Colin Williams}

\begin{document}
\maketitle

\section*{Question 1}
Let $(X, d_X)$ and $(Y, d_Y)$ be metric spaces where $f: X \to Y$ is continuous and $y \in Y$. Prove that the set $\{x \in X : f(x) = y\}$ is closed in $X$. 

\begin{proof}$ $
\\Note that the set we are interested in is simply $f^{-1}(\{y\})$. Recall that $f^{-1}(U^c) = f^{-1}(U)^c$ for all $U \subset Y$. Thus, $f^{-1}(\{y\})^c = f^{-1}(\{y\}^c) = f^{-1}(Y \backslash \{y\})$. Next, note that $Y\backslash \{y\}$ is an open set. To see this, let $y_0 \in Y \backslash \{y\}$. Let $r := d_Y(y_0, y) / 2$. Then it is clear that 
\begin{align*}
U_r(y_0) = \{x \in Y \; | \; d(x, y_0) < r\} &\not \ni \{y\}\\
\implies U_r(y_0) \subset Y \backslash \{y\}
\end{align*}
This means that $y_0$ is an interior point of $Y \backslash \{y\}$ and this can be done for all $y_0 \in Y \backslash \{y\}$, so $Y \backslash \{y\} \subset (Y \backslash \{y\})^\circ$ meaning the set is open. Furthermore, since $f$ is continuous, we know that $f^{-1}$ maps opens sets in $Y$ to open sets in $X$. In other words, $f^{-1}(Y \backslash \{y\})$ is open. By the equalities shown above, this means that $f^{-1}(\{y\})^c$ is open, or equivalently, that $f^{-1}(\{y\})$ is closed. Therefore, we have proven that the given set is closed. 
\end{proof}

\section*{Question 2}
Let $X = \mathbb{R}^n$, $Y = \mathbb{R}^m$, $Z = \mathbb{R}^{n+m}$ each equipped with the appropriate Euclidean metric. Let $K \subset X$ be compact and let $f: K \to Y$ be continuous. Prove that the graph of $f$, $\Gamma(f) = \{(x, f(x)) : x \in K\} \subset Z$ is compact. 

\begin{proof}$ $
\\Since $f$ is continuous and $K$ is compact, then we know that $f(K) \subset Y$ is compact. In particular, since $X = \mathbb{R}^n$ and since $Y = \mathbb{R}^m$, then we know that $K$ and $f(K)$ are both closed and bounded. Thus, our goal is simply to show that $\Gamma(f) \subset Z = \mathbb{R}^{n + m}$ is closed and bounded. 
\\
\\Since $K$ and $f(K)$ are both bounded, they both have finite diameters, say diam$(K) = r$ and diam$(f(K)) = \rho$. Since we are using the Euclidean metric, this means that $[(w_1 - x_1)^2 + (w_2 - x_2)^2 + \cdots + (w_n - x_n)^2]^{1/2} < r$ for all $w, x \in K$ with $w = (w_1, w_2, \ldots, w_n)$ and $x = (x_1, x_2, \ldots, x_n)$. Similarly, $[(y_1 - z_1)^2 + (y_2 - z_2)^2 + \cdots + (y_m - z_m)^2]^{1/2} < \rho$ for all $y, z \in f(K)$ with $y = (y_1, y_2, \ldots, y_m)$ and $z = (z_1, z_2, \ldots, z_n)$. Thus, let $(w, y), (x, z)$ be two arbitrary elements of $\Gamma(f)$. By examining their distance, we have
\begin{align*}
|(w, y) - (x, z)| &= \sqrt{(w_1 - x_1)^2 + (w_2 - x_2)^2 + \cdots + (w_n - x_n)^2 + (y_1 - z_1)^2 + (y_2 - z_2)^2 + \cdots + (y_m - z_m)^2}\\
&< \sqrt{r^2 + \rho^2}
\end{align*}
Thus, the diameter of $\Gamma(f)$ is at most $\sqrt{r^2 + \rho^2}$, so $\Gamma(f)$ must also be bounded. Next, I will show that $\Gamma(f)$ is closed. 
\\
\\Let $(x_n, y_n)$ be a sequence in $\Gamma(f)$ that converges to the point $(x, y) \in Z$. Since the sequence is contained in $\Gamma(f)$, then the point $(x, y)$ is either in $\Gamma(f)$ or in $\Gamma(f)'$, so that $(x, y) \in \overline{\Gamma(f)}$. Furthermore, since $\{(x_n, y_n)\} \subset \Gamma(f)$, then the sequence $(x_n) \subset K$ and $(y_n) \subset f(K)$. In particular, by construction of $\Gamma(f)$, $y_n = f(x_n)$. Thus, since $K$ is closed, $x \in K$ and since $f(K)$ is closed, then $y \in f(K)$. Thus, since $y_n = f(x_n)$ and $y_n \to y$ and $x_n \to x$, then by continuity of $f$, we have $f(x_n) \to f(x) = y$. Thus, we have $(x, y) = (x, f(x)) \in K \times f(K) = \Gamma(f)$. Therefore, $\Gamma(f)$ contains its limit points, so it is closed. 
\\
\\We have now shown that $\Gamma(f)$ is both closed and bounded. Since $\Gamma(f) \subset Z = \mathbb{R}^{n+m}$, we have that $\Gamma(f)$ is compact. 
\end{proof}

\section*{Question 3}
Let $(X, d_X), (Y, d_Y), (Z, d_Z)$ be metric spaces and let $f: X \to Y$ and $g: Y \to Z$ be uniformly continuous. Prove that $g \circ f: X \to Z$ is uniformly continuous as well. 

\begin{proof}$ $
\\Fix an $r > 0$, then since $g$ is uniformly continuous, we know there exists some $s_0 > 0$ such that for all $y, z \in Y$ such that $d_Y(y, z) < s_0$ we have $d_Z(g(y), g(z)) < r$. 
\\
\\Next, fix some $w, x \in X$. Then, since $f$ is uniformly continuous, we know that there exists some $s > 0$ such that $d_X(w, x) < s$ implies $d_Y(f(w), f(x)) < s_0$. 
\\
\\Thus, since $f(w), f(x) \in Y$ and satisfy $d_Y(f(w), f(x)) < s_0$, then we can conclude that $d_Z(g(f(w)), g(f(x))) < r$. In total, we have concluded the following: for all $r > 0$, there exist some $s > 0$ such that for all $w, x \in X$, the inequality $d_X(w, x) < s$ implies that $d_Z(g \circ f(w), g \circ f(x)) < r$ which means that $g \circ f$ is uniformly continuous. 
\end{proof}

\section*{Question 4}
Let $(X, d_X(x, y)) = (\mathbb{R}^n, |x - y|)$ and $(Y, d_Y)$ be a complete metric space. Let $B \subset X$ be bounded and let $f: B \to Y$ be uniformly continuous. Prove that $f(B)$ is bounded. 

\begin{proof}$ $
\\My idea is to show that the extension of $f$ given as $g: \overline{B} \to Y$ is continuous, so that $g(\overline{B})$ is compact (since $\overline{B} \subset \mathbb{R}^n$ is closed and $B$ bounded implies $\overline{B}$ bounded). We have proven in class that all compact spaces are bounded, so this would show that $g(\overline{B})$ is bounded. Thus, since $f(B) \subset g(\overline{B})$, then $f(B)$ is bounded. 
\\
\\More formally, define $g: \overline{B} \to Y$ as
\begin{align*}
g(x) = \begin{cases}
f(x) &\text{ if } x \in B\\
\lim_{n \to \infty} f(x_n) &\text{ if } x \in B' \text{ and } x = \lim_{n \to \infty} x_n \text{ for } x_n \in B
\end{cases}
\end{align*}
Thus, it is clear that $g$ is continuous on $B$ since $f$ is assumed to be (uniformly) continuous on $B$. Therefore, I simply need to show that $g$ is continuous at each point $x \in B'$. Let $x \in B'$, then there is a sequence $(x_n)$ in $B$ such that $\lim_{n \to \infty} x_n = x$. In particular, since the sequence is convergent, it must be Cauchy. The first thing I will prove is that the sequence $g(x_n) = f(x_n)$ is also Cauchy. Since $f$ is uniformly continuous, then for every $r > 0$, we can find some $s > 0$ such that for any $x, y \in B$ with $|x - y| < s$, we have $d_Y(f(x), f(y)) < r$. Thus, since $(x_n)$ is Cauchy, then there exists some $N$ such that $|x_n - x_m| < s$ for all $n, m \geq N$. Therefore, if we fix some $r > 0$, then there exists some $N$ (the same as from before) such that $d_Y(f(x_n), f(x_m)) < r$ for all $n, m \geq N$ since $|x_n - x_m| < s$ for all $n, m \geq N$.  Thus, $f(x_n)$ is indeed a Cauchy sequence. 
\begin{itemize}
\item Note if we only had continuity and not uniform continuity, then we may have a different $s$ for each $x \in B$. Therefore, saying that $|x_n - x_m| < s$ only guarantees that $d_Y(f(x_n), f(x_m)) < r$ for those fixed $n$ and $m$, not necessarily all $n, m \geq N$. 
\end{itemize}
Since we have that $f(x_n)$ is Cauchy, then we know that the sequence $(f(x_n))$ converges in $Y$ since $Y$ is a complete metric space. This means that $g$ is well defined since for every $x \in B'$, $\lim_{n \to \infty} f(x_n)$ exists and we know that if limits exist, then they are unique. Now I will show that $g$ is continuous at $x$. In fact, it may be more direct to show $g$ is uniformly continuous on $B'$.
\\
\\Let $x = \lim_{n \to \infty} x_n$ and $y = \lim_{n \to \infty} y_n$ be two points in $B'$. Let us fix some $r > 0$, then we have the following facts we can work with:
\begin{enumerate}[label = (\alph*)]
\item As we have shown, $f(x_n)$ and $f(y_n)$ should both converge to $f(x)$ and $f(y)$ respectively. In particular, this means we can find an $N_1$ and an $N_2$ such that $d_Y(f(x_n), f(x)) < r/3$ for all $n \geq N_1$ and that $d_Y(f(y_n), f(y)) < r/3$ for all $n \geq N_2$.
\item Since $f$ is uniformly continuous, then there exists some $s > 0$ such that if $|x_n - y_n| < 3s$, then $d_Y(f(x_n), f(y_n)) < r/3$. 
\item Since $x_n \to x$, then there exists some $N_3$ such that $|x_n - x| < s$ for all $n \geq N_3$. Similarly, there exists some $N_4$ such that $|y_n - y| < s$ for all $n \geq N_4$. 
\item If we choose $x$ and $y$ close enough so that $|x - y| < s$, then using (c), we get that for all $n \geq \max\{N_3, N_4\}$,
\begin{align*}
|x_n - y_n| = |x_n - x + x - y + y - y_n| \leq |x_n - x| + |x - y| + |y - y_n| < s + s + s = 3s
\end{align*}
\end{enumerate}
Therefore, taking $n \geq \max\{N_1, N_2, N_3, N_4\}$, then for all $x, y \in B'$ such that $|x - y| < s$, we have
\begin{align*}
d_Y(f(x), f(y)) &\leq d_Y(f(x), f(x_n)) + d_Y(f(x_n), f(y_n)) + d_Y(f(y_n), f(y)) &\text{by Triangle Inequality}\\
&< r/3 + d_Y(f(x_n), f(y_n)) + r/3 &\text{by (a)}\\
&< r/3 + r/3 + r/3 &\text{by (b) since we have (d)}\\
&= r
\end{align*}
In total, this proves that for any $r > 0$, there exists some $s > 0$ such that for any $x, y \in B'$ such that $|x - y| < s$ it implies that $d_Y(f(x), f(y)) < r$. Therefore, $g$ (which is the extension of $f$ to $\overline{B}$) is uniformly continuous on $B'$. We already knew $g$ was uniformly continuous on $B$, therefore $g$ is uniformly continuous on $\overline{B} = B' \cup B$. Therefore, as it was discussed at the beginning of the proof, this means that $f(B)$ is bounded. 
\end{proof}


\end{document}