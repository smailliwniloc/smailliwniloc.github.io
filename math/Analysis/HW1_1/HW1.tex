\documentclass[10pt,a4paper]{article}
\usepackage[utf8]{inputenc}
\usepackage[english]{babel}
\usepackage{csquotes}
\usepackage{amsmath}
\usepackage{amsfonts}
\usepackage{amssymb}
\usepackage{graphicx}
\usepackage[margin=0.5in]{geometry}
\usepackage{amsthm}
\usepackage{enumitem}
\usepackage{tikz}
\usetikzlibrary{calc}
\newtheorem{question}{Question}
\newtheorem*{question*}{Question}
\newtheorem{theorem}{Theorem}
\newtheorem*{theorem*}{Theorem}
\newtheorem{lemma}{Lemma}

\theoremstyle{definition}
\newtheorem{answer}{Answer}
\newtheorem*{answer*}{Answer}

\theoremstyle{definition}
\newtheorem{verify}{Verification}
\newtheorem*{verify*}{Verification}


\title{Analysis Homework 1}
\author{Colin Williams}

\begin{document}
\maketitle

\section*{Question 1}
Which of the following are metrics on $\mathbb{R}$?
\begin{enumerate}[label = (\alph*)]
\item $d(x, y) = (x - y)^2$;
	\begin{itemize}
	\item Note that $d(x, y) \geq 0$ for all $x, y$ since $(\cdot)^2$ is a non-negative function on the real numbers. Next, assume that $d(x, y) = (x - y)^2 = 0$ for some $x, y \in \mathbb{R}$. From this, we obtain $x - y = 0 \implies x = y$. Similarly, if $x = y \in \mathbb{R}$, then $d(x, x) = (x - x)^2 = 0^2 = 0$ which satisfies the first property of a metric. 
	\item Next, if $x, y \in \mathbb{R}$, note that $d(x, y) = (x - y)^2 = (-1[y - x])^2 = (-1)^2 (y - x)^2 = (y - x)^2 = d(y, x)$ meaning it satisfies the second property of a metric. 
	\item Lastly, let $x = 1, y = -1$, and $z = 0$, then it is clear to see that $d(x, y) \not \leq d(x, z) + d(z, y)$ since $4 \not \leq 1 + 1$ so the triangle inequality does not apply to this function, so $\boxed{d(x, y) \text{ is not a metric}}$
	\end{itemize}
\item $d(x, y) = |x^2 - y^2|$;
	\begin{itemize}
	\item Note that the first property of a metric does not hold for this function since for $x = 1$ and $y = -1$, we have $d(x, y) = |1^2 - (-1)^2| = 0$ but $1 \neq -1$ meaning $\boxed{d(x, y) \text{ is not a metric}}$
	\end{itemize}
\item $\displaystyle d(x, y) = \frac{|x - y|}{1 + |x - y|}$
	\begin{itemize}
	\item Note that $d(x, y) \geq 0$ for all $x, y \in \mathbb{R}$ since the numerator and denominator are both non-negative functions on $\mathbb{R}$. Furthermore, it is clear that if $x = y$, then $d(x, y) = 0$ since the numerator would be zero. On the other hand, assume that $d(x, y) = 0$:
	\begin{align*}
	\frac{|x - y|}{1 + |x - y|} &= 0\\
	\implies |x - y| &= 0(1 + |x - y|)\\
	\implies |x - y| &= 0\\
	\implies x &= y &\text{since $|\cdot, \cdot |$ is a metric}
	\end{align*}
	Thus, the first property of a metric is satisfied. 
	\item Next, let $x, y \in \mathbb{R}$, then note:
	\begin{align*}
	d(x, y) = \frac{|x - y|}{1 + |x - y|} = \frac{|(-1)(y - x)}{1 + |(-1)(y - x)|} = \frac{|-1||y - x|}{1 + |-1||y - x|} = \frac{|y - x|}{1 + |y - x|} = d(y, x)
	\end{align*}
	which satisfies the second property of a metric. 
	\end{itemize}
	\item Lastly, we need to deal with the Triangle Inequality, I will do so in a reverse fashion: let $x, y, z \in \mathbb{R}$, then we have:
	\begin{align*}
	d(x, z) + d(z, y) &= \frac{|x - z|}{1 + |x - z|} + \frac{|z - y|}{1 + |z - y|}\\
	&\geq \frac{|x - z|}{1 + |x - z| + |z - y|} + \frac{|z - y|}{1 + |x - z| + |z - y|}\\
	&= \frac{|x - z| + |z - y|}{1 + |x - z| + |z - y|}\\
	&= \frac{1 + |x - z| + |z - y| - 1}{1 + |x - z| + |z - y|}\\
	&= 1 - \frac{1}{1 + |x - z| + |z - y|}\\
	&\geq 1 - \frac{1}{1 + |x - y|} &\text{since $|x - y| \leq |x - z| + |z - y|$}\\
	&= \frac{1 + |x - y| - 1}{1 + |x - y|}\\
	&= \frac{|x - y|}{1 + |x - y|}\\
	&= d(x, y)
	\end{align*}
	Altogether, this states that $d(x, y) \leq d(x,z) + d(z, y)$ which is precisely the Triangle Inequality. Thus, \\$\boxed{d(x, y) \text{ is a metric.}}$
\end{enumerate}

\section*{Question 2}
Let $(X, d)$ be a metric space, and let $Y \subset X$. Prove the following:
\begin{enumerate}[label = (\alph*)]
\item $Y^\circ$ is open;
	\begin{proof}$ $
	\\Recall that a set $E \subset X$ is open if $E \subset E^\circ$. Thus, I need only check that $Y^\circ \subset (Y^\circ)^\circ$. Let $x \in Y^\circ$. This means that there exists some $r > 0$ such that $U_r(x) \subset Y$. Furthermore, let $z \in U_r(x)$, then if we take $\rho = r - d(x, z)$, it is clear that $U_\rho(z) \subset U_r(x) \subset Y$ since for any $y \in U_\rho(z)$, $d(x, y) \leq d(x, z) + d(z, y) < d(x, z) + r - d(x, z) = r$. Thus, $z$ is an interior point of $Y$ as well. In fact, since $z$ was chosen arbitrarily from $U_r(x)$, we can conclude that $U_r(x) \subset Y^\circ$. 
	\\
	\\Thus, we have shown that for any $x \in Y^\circ$, there is some radius $r$ where $U_r(x) \subset Y^\circ$. By definition, this means that $x \in (Y^\circ)^\circ$ so that $Y^\circ \subset (Y^\circ)^\circ$, i.e. $Y^\circ$ is an open set. 
	\end{proof}
\item if $Z \subset Y$ and $Z$ is open, then $Z \subset Y^\circ$;
	\begin{proof}$ $
	\\Let $x \in Z$. Since $Z$ is open, we know that $x$ is an interior point of $Z$, i.e. there exists some $r > 0$ such that $U_r(x) \subset Z$. However, since $Z \subset Y$, this means that $U_r(x) \subset Y$, so $x$ is indeed an interior point of $Y$ as well. Thus, we have some arbitrary point in $Z$ must be an interior point of $Y$, so $Z \subset Y^\circ$. 
	\end{proof}
\item $(Y^\circ)^c = \overline{Y^c}$.
	\begin{proof}$ $
	\\Let $x \in (Y^\circ)^c$. In other words, $x \in X$, but $x \not \in Y^\circ$. Since $x \not \in Y^\circ$, then for any $r > 0$, we know that $U_r(x) \not \subset Y$. Thus, there is some element in $U_r(x)$ which is not in $Y$, i.e. $U_r(x) \cap Y^c \neq \emptyset$. If the point $x \not \in Y$, then $x \in Y^c$ and obviously $x \in \overline{Y^c} = Y^c \cup (Y^c)'$. If $x \in Y$, we can improve the previous result about the intersection to say that $\stackrel{\circ}{U}_r(x) \cap Y^c \neq \emptyset$ since $x$ is not in $Y^c$, it is clearly not in the intersection. However, we have just shown that $x$ is a limit point of $Y^c$ which means that $x \in (Y^c)' \subset \overline{Y^c}$. Thus, whether $x \in Y$ or $x \not \in Y$, we can say that $x \in \overline{Y^c}$ meaning that $(Y^\circ)^c \subset \overline{Y^c}$.
	\\
	\\Next, let $x \in \overline{Y^c}$. If $x \in Y^c$, then clearly $x \in (Y^\circ)^c$ since $Y^\circ \subset Y$. Thus, let $x \in (Y^c)'$. By definition, this means that for all $r > 0$, $\stackrel{\circ}{U}_r(x) \cap Y^c \neq \emptyset$. In particular, since the punctured ball with the intersection is nonempty, so too must the standard ball with the intersection be nonempty. Thus, $U_r(x) \cap Y^c \neq \emptyset$. This means there is some element in $U_r(x)$ which is not in $Y^c$. In other words, $U_r(x) \not \subset Y$. Since this argument was made in terms of all $r > 0$, then we can say that there is no ball around $x$ of any radius that is contained in $Y$, i.e. $x$ is not an interior point of $Y$ which means $x \in (Y^\circ)^c$ showing that $\overline{Y^c} \subset (Y^\circ)^c$. 
	\\
	\\Together, this shows that $(Y^\circ)^c = \overline{Y^c}$. 
	\end{proof}
\end{enumerate}

\section*{Question 3}
Let $X$ be an infinite set. Define $d: X \times X \to \mathbb{R}$ as $$
d(x, y) = \begin{cases}
1 &\text{ if } x \neq y, \\
0 &\text{ if } x = y.
\end{cases}$$
Prove that $d$ is a metric. Which subsets of $X$ are open? Closed? Compact?

\begin{proof}$ $
\\The first property of a metric is trivial based off of the definition of $d$. Clearly $d(x, y) \geq 0$ for all $x, y \in X$ and $d(x, y) = 0 \iff x = y$. The second property is also trivial to show since $x = y \iff y = x$ and $x \neq y \iff y \neq x$ which clearly shows that $d(x, y) = d(y, x)$. 
\\
\\The Triangle Inequality takes a little bit more work. Let $x, y, z \in X$ and consider the following cases:
\begin{itemize}
\item $x = y = z$, then $d(x, y) \leq d(x, z) + d(z, y)$ since $0 \leq 0 + 0$. 
\item $x = y$ and $x \neq z$, then $d(x, y) \leq d(x, z) + d(z, y)$ since $0 \leq 1 + 1$. 
\item $x \neq y$ and $x = z$, then $d(x, y) \leq d(x, z) + d(z, y)$ since $1 \leq 0 + 1$. 
\item $x \neq y$ and $y = z$, then $d(x, y) \leq d(x, z) + d(z, y)$ since $1 \leq 1 + 0$. 
\item $x \neq y$, $y \neq z$, and $x \neq z$, then $d(x, y) \leq d(x, z) + d(z, y)$ since $1 \leq 1 + 1$. 
\end{itemize}
Since this exhausts the list of possibilities, we have the Triangle Inequality holds as well Thus, $d$ is a metric. 
\end{proof}

\begin{answer*}$ $
\\To determine which subsets are open, closed, or compact, I will first establish those definitions using this metric. 
\\
\\A set $S \subset X$ will be open if for every $x \in S$, there exists some $r > 0$ such that $U_r(x) \subset S$. More precisely, there exists some $r > 0$ such that $\{y \in X \; | \; d(x, y) < r\} \subset S$. Note that if $r > 1$, then $U_r(x) = X$ and if $r < 1$, then $U_r(x) = \{y \; | \; y = x\} = \{x\}$. Thus, $U_r(x) \subset S$ for all $r < 1$ since $x \in S$. Thus, any subset $S$ must be open since every point is an interior point.  
\\
\\A set $S \subset X$ will be closed if every point $x \in X$ satisfying $\stackrel{\circ}{U}_r(x) \cap S \neq \emptyset$ is also a point in $S$ for any $r > 0$. In particular, this must be true for $r = 0.5$, but $\stackrel{\circ}{U}_{0.5}(x) = \emptyset$. This means there are no points $x$ satisfying that criteria. Thus, there are no limits points in this metric space, so every subset $S$ must be closed as they trivially contain all of their limit points. 
\\
\\Lastly, recall that for a subset $S$ to be compact, that \emph{any} open cover of $S$ must admit a finite subcover. However, note that for each $x \in S$, $\{x\}$ is an open set. Therefore $\{\{x\}\}_{x \in S}$ is an open cover for $S$ since
$$
S \subset \bigcup_{x \in S} \{x\}
$$
However, this cover has no strict subcover since removing any element from the set no longer makes it a cover. Therefore, the only \enquote{subcover} is the cover itself. Therefore, a finite subcover only exists whenever the set $S$ itself is a finite set. Consequently, each finite set $S$ is compact by the same construction as above, and we can conclude that the compact subsets are precisely the finite subsets of $X$. 
\end{answer*}

\section*{Question 4}
Consider the metric space $(X, d)$, where $X = \mathbb{Q}$, $d(x, y) = |x - y|$. Prove that the subset $Y = \{x \in X \; | \; 2 < x^2 < 3\}$ is closed and bounded, but not compact. 

\begin{proof}$ $
\\First, it is clear that $Y$ is bounded since $Y \subset \{x \in X \; | \; x^2 < 4 \} = U_2(0)$. Next, I will use the following two theorems that we have proven in class:

\begin{theorem}$ $
\\Let $(A, d)$ be a metric space where $B \subseteq A$ and let $C \subseteq B$, then $C$ is open in $B \iff C = B \cap G$ and $G$ is open in $A$. 
\end{theorem}

\begin{theorem}$ $
\\Let $(A, d)$ be a metric space where $B \subseteq A$, then $B$ is closed if and only if $B^c$ is open. 
\end{theorem}
$ $
\\First, I will show that $Y^c = \{x \in X \; | \; x^2 \leq 2 \text{ or } x^2 \geq 3\}$ is an open set. I will do this by using Theorem 1 with $A = \mathbb{R}$, $B = \mathbb{Q}$ ( $= X$ ), $C = Y^c$, and $G = \{x \in \mathbb{R} \; | \; x^2 < 2 \text{ or } x^2 > 3\}$. To verify that the hypotheses of the Theorem hold, I simply need to realize that $G$ is open in $\mathbb{R}$ since it is the union of three open sets in $\mathbb{R}$: namely, $(-\infty, -\sqrt{3}), (-\sqrt{2}, \sqrt{2})$, and $(\sqrt{3}, \infty)$. Note that since $\sqrt{2}$ and $\sqrt{3}$ are not rational, then $Y^c$ is equivalent to $Y^c = \{x \in X \; | \; x^2 < 2 \text{ or } x^2 > 3\}$. Thus, $Y^c = \mathbb{Q} \cap G$, so by Theorem 1, $Y^c$ is open in $\mathbb{Q}$. 
\\
\\Now, by using Theorem 2, I can say that $Y$ must be closed in $\mathbb{Q}$ since we have established that $Y^c$ is open in $\mathbb{Q}$. 
\\
\\Finally, I need to show that $Y$ is not compact. To do this, I will find a cover for $Y$ that does not admit a finite subcover. Consider the intervals:
\begin{align*}
U_n &= \left(\sqrt{3} + \frac{1}{n}(\sqrt{2} - \sqrt{3}), \sqrt{3} + \frac{1}{n+1}(\sqrt{2} - \sqrt{3})\right)\\
V_n &= \left(-\sqrt{2} + \frac{1}{n}(\sqrt{2} - \sqrt{3}), -\sqrt{2} + \frac{1}{n+1}(\sqrt{2} - \sqrt{3})\right)
\end{align*}
Notice that $U_n$ simply partitions the interval $(\sqrt{2}, \sqrt{3})$ into distinct intervals where the $n$th interval is of length $(\sqrt{2} - \sqrt{3})(1/(n+1) - 1/n)$. Similarly, $V_n$ covers the interval $(-\sqrt{3}, -\sqrt{2})$. Furthermore, we can see that each of these individual intervals is open in $\mathbb{Q}$ since they are equal to the same (open) interval in $\mathbb{R}$ intersected with $\mathbb{Q}$ which makes them each open. Thus, the set $\{W_n\}_{n = 1}^\infty$ for $W_n = U_n \cup V_n$ is an open cover for $Y$. However, there does not exist any finite subcover because $W_i \cap W_j = \emptyset$ for all $i \neq j$. Thus, removing any $W_i$ from the set no longer makes it a cover, since the elements in that set is not included in any other sets. Thus, $Y$ is not compact. 
\end{proof}


\end{document}