\documentclass[10pt,a4paper]{article}
\usepackage[utf8]{inputenc}
\usepackage[english]{babel}
\usepackage{csquotes}
\usepackage{amsmath}
\usepackage{amsfonts}
\usepackage{amssymb}
\usepackage{graphicx}
\usepackage[margin=0.5in]{geometry}
\usepackage{amsthm}
\usepackage{enumitem}
\usepackage{tikz}
\usetikzlibrary{calc}
\newtheorem{question}{Question}
\newtheorem*{question*}{Question}
\newtheorem{theorem}{Theorem}
\newtheorem*{theorem*}{Theorem}
\newtheorem{lemma}{Lemma}

\theoremstyle{definition}
\newtheorem{answer}{Answer}
\newtheorem*{answer*}{Answer}

\theoremstyle{definition}
\newtheorem{verify}{Verification}
\newtheorem*{verify*}{Verification}

\numberwithin{equation}{section}


\title{Analysis HW 12}
\author{Colin Williams}

\begin{document}
\maketitle

In what follows, $(X, \Sigma, \mu)$ is a measure space. 

\section*{Question 1}
Let $Y_n$ be a sequence in $\Sigma$ such that $Y_1 \supset Y_2 \supset Y_3 \supset \cdots $. Prove that if $\mu(Y_1) < +\infty$, then $\mu\left(\bigcap_{n \in \mathbb{N}} Y_n\right) = \lim_{n \to \infty} \mu(Y_n)$. Construct an example where $X = \mathbb{N}, \Sigma = 2^X, \mu$ the counting measure and $\forall \; n \in \mathbb{N}, \mu(Y_n) = +\infty$, but $\mu\left(\bigcap_{n \in \mathbb{N}} Y_n \right) = 0$. 

\begin{proof}$ $
\\Denote $Y = \bigcap_{n \in \mathbb{N}} Y_n$ We clearly have that $Y \subset Y_n$ for all $n \in \mathbb{N}$. Therefore, $\mu(Y) \leq \mu(Y_n)$ for all $n \in \mathbb{N}$ and so by taking limits we have in fact 
\begin{align*}
\mu(Y) \leq \lim_{n \to \infty} \mu(Y_n)
\end{align*}
This doesn't prove the desired statement; though, so we will need to be more clever. Define $Z_n = Y_n \backslash Y_{n + 1}$. Then, notice the following:
\begin{align*}
Y_k = Y \cup \bigcup_{n = k}^\infty Z_n
\end{align*}
It is easy to see the $\supseteq$ inclusion as $Y \subset Y_k$ and $Z_n \subset Y_n \subset Y_k$ for all $n \geq k$. For the other inclusion, let $y \in Y_k$. If $y \in \bigcap_{n \in \mathbb{N}} Y_n$, then by definition $y \in Y$, so $y \in Y \cup \bigcup_{n = k}^\infty Z_n$. If $y \not \in \bigcap_{n \in \mathbb{N}} Y_n$, then there exists some $n \geq k$ such that $y \in Y_n$ but $y \not \in Y_{n+1}$. In other words, $y \in Z_n$ for some $n \geq k$ so that in fact $y \in Y \cup \bigcup_{n = k}^\infty Z_n$. Furthermore, each $Z_n$ is disjoint from the others which can be seen in general for arbitrary $i > j$ since we have $Z_j \subset Y_{j+1}^c \subset Y_i^c$ and $Z_i \subset Y_i$ so in fact they are disjoint. Furthermore, $Y$ is disjoint from all $Z_n$ since if $y \in Z_n$, then $y \not \in Y_{n+1} \supset Y$. With this in mind, we can use the additivity of the measure to say
\begin{align*}
\mu(Y_k) &= \mu(Y) + \sum_{n = k}^\infty \mu(Z_n)\\
&= \mu(Y) + \sum_{n = 1}^\infty \mu(Z_n) - \sum_{n = 1}^{k-1} \mu(Z_n)\\
&= \mu(Y_1) - \sum_{n = 1}^{k-1} \mu(Z_n)\\
&= \mu(Y_1) - \left( \mu(Y) + \sum_{n = 1}^{k-1} \mu(Z_n) \right) + \mu(Y) &(1)\\
\implies \lim_{k \to \infty} \mu(Y_k) &= \lim_{k \to \infty} \left[\mu(Y_1) - \left( \mu(Y) + \sum_{n = 1}^{k-1} \mu(Z_n) \right) + \mu(Y)\right]\\
&= \mu(Y_1) - \left( \mu(Y) + \sum_{n = 1}^{\infty} \mu(Z_n) \right) + \mu(Y)\\
&= \mu(Y_1) - \mu(Y_1) + \mu(Y)\\
&= \mu(Y) &(2)\\
&= \mu\left(\bigcap_{n \in \mathbb{N}} Y_n\right)
\end{align*}
In line (1), I added and subtracted $\mu(Y)$ which is valid since $\mu(Y) \leq \mu(Y_1) < +\infty$ so I added and subtracted a finite value. Similarly, in line (2), I noticed that $\mu(Y_1) - \mu(Y_1) = 0$ which is again only valid since $\mu(Y_1) < +\infty$. Therefore, we have the desired equality. 
\end{proof}

\begin{answer*}$ $
\\For a counterexample where the measure of every subset is infinite, consider $Y_n = [n, +\infty) \cap \mathbb{N} \subset X$. This is measurable since $\Sigma = 2^X$ and has an infinite number of elements, so $\mu(Y_n) = +\infty$ for all $n \in \mathbb{N}$. Therefore, we clearly have that
\begin{align*}
\lim_{n \to \infty} \mu(Y_n) = +\infty
\end{align*}
However, when looking at the intersection, 
\begin{align*}
\bigcap_{n \in \mathbb{N}} Y_n = \emptyset
\end{align*}
Since for every $x \in X$, $x \not \in Y_{x + 1}$, so it is not in the intersection. Therefore, the intersection is empty. Thus, 
\begin{align*}
\mu\left(\bigcap_{n \in \mathbb{N}} Y_n\right) = 0
\end{align*}
which means we do NOT have the stated equality in the question above. 
\end{answer*}

\section*{Question 2}
Let $(X, \Sigma, \mu)$ be as in Exercise 4 of HW11 (namely, $X$ is uncountable, $\Sigma$ is all sets $Y \subset X$ such that either $Y$ or $Y^c$ is at most countable, and $\mu(Y) = 1$ if $Y$ is uncountable and $\mu(Y) = 0$ otherwise). Describe all measurable functions $f: X \to \mathbb{R}$. 

\begin{answer*}$ $
\\Let $f: X \to \mathbb{R}$ be a measurable function. Denote $Y = f(X) \subset \mathbb{R}$. I claim that $Y$ must be (at most) countable. 
\\
\\If $Y$ is uncountable, then we can divide $Y$ into two uncountable disjoint subsets. More formally, if we denote $Y_1(x) = Y \cap (-\infty, x)$ and $Y_2(x) = Y \cap [x, +\infty)$, then there must be some $x \in X$ such that $Y_1(x)$ and $Y_2(x)$ are both uncountable. It is clear that at least one of them must be uncountable for if they were both countable, then their union would be countable (which it is not). As $x \to \infty$, $Y_1(x) \to Y$ which means for some $x$ large enough $Y_1(x)$ is uncountable. Similarly, for some $x$ as $x \to -\infty$,  $Y_2(x)$ must be uncountable. Therefore, as we discussed before since there is not a point where both sets are countable, there must be a point where both sets are uncountable. 
\\
\\Then, let $x_0$ be a point where $Y_1(x_0)$ and $Y_2(x_0)$ are both uncountable. Then, since $Y_1(x_0), Y_2(x_0) \subset f(X)$ and are both uncountable, then $f^{-1}(Y_1(x_0))$ and $f^{-1}(Y_2(x_0))$ must both be uncountable since for every $y \in Y_1(x_0)$, there must exist at least one $x \in f^{-1}(Y_1(x_0))$ as $Y_1(x_0)$ is a subset of the image of $f$ (and similarly for $Y_2(x_0)$). Thus, $f^{-1}\big((-\infty, x_0)\big) = f^{-1}(Y_1(x_0))$ is uncountable and $f^{-1}\big((-\infty, x_0)\big)^c = f^{-1}(Y_2(x_0))$ is also uncountable. Therefore, $f$ would not be measurable which is a contradiction. Therefore, we must have that $Y$ is at most countable. 
\\
\\Let $y \neq z \in Y$ be two distinct points in the image of $f$. I claim that only one of $f^{-1}(\{y\})$ or $f^{-1}(\{z\})$ can be uncountable. WLOG, assume that $y < z$. If they were both uncountable, then 
\begin{align*}
f^{-1}(\{y\}) &\subset f^{-1}\left(\left(-\infty, \frac{y + z}{2}\right)\right)\\
f^{-1}(\{z\}) &\subset f^{-1}\left(\left(-\infty, \frac{y + z}{2}\right)\right)^c = f^{-1}\left(\bigg[ \frac{y + z}{2}, + \infty \bigg) \right)
\end{align*}
both of the above sets would be uncountable which would means $f$ is not measurable, a contradiction. So far, we have determined that the image of $f$ is at most countable (i.e. has measure zero). Similarly, of the points in the image of $f$, at most one of them can have an uncountable pre-image. In fact, we must have that at least one of the points in the image has an uncountable pre-image since $X$ itself is uncountable and there are only (at most) countably many points in the image. Thus, we see there is EXACTLY one point in the image of $f$ which has an uncountable pre-image. Therefore,
\begin{align*}
\boxed{\text{$f$ must be a constant function except possibly on a set of measure zero}}
\end{align*}
\end{answer*}

\section*{Question 3}
Let $f, g: X \to [-\infty, +\infty]$ be two measurable functions. Prove that $\{x: f(x) = g(x)\} \in \Sigma$. 

\begin{proof}$ $
\\For $f$ and $g$ to be measurable as a mapping into the extended real numbers, we must have that $f^{-1}((\alpha, +\infty])$ and $g^{-1}((\alpha, +\infty])$ are measurable for any $\alpha \in \mathbb{R}$. With this, we can also say that $f^{-1}([-\infty, \beta))$ must also be measurable since
\begin{align*}
f^{-1}([-\infty, \beta)) = f^{-1}([\beta, +\infty])^c = f^{-1}\left( \bigcap_{n = 1}^\infty \bigg( \beta - \frac{1}{n}, +\infty\bigg] \right)^c = f^{-1} \left( \bigcup_{n = 1}^\infty \bigg( \beta - \frac{1}{n}, +\infty\bigg]^c \right) = \bigcup_{n = 1}^\infty f^{-1} \bigg( \beta - \frac{1}{n}, +\infty\bigg]^c
\end{align*}
and the last expression is the countable union of the complement of measurable sets, so it is measurable. Then, using the density of the rational numbers, notice the following:
\begin{align*}
\{x : f(x) < g(x)\} = \bigcup_{r \in \mathbb{Q}} \{x : f(x) < r < g(x)\} = \bigcup_{r \in \mathbb{Q}} \{x : f(x) < r \} \cap \{x : r < g(x)\} = \bigcup_{r \in \mathbb{Q}} f^{-1}([-\infty, r)) \cap g^{-1}((r, +\infty])
\end{align*}
As discussed before, both of the sets being intersected are measurable. Furthermore, the intersection of measurable sets is measurable (using De Morgan's Laws with closure under union and complement). Thus, we see that $\{x : f(x) < g(x)\}$ can be realized as the \textit{countable} union of measurable sets, so it is indeed measurable. In a similar fashion, we can show that $\{x : f(x) > g(x)\}$ is also countable. Then, notice we have
\begin{align*}
\bigg(\{x : f(x) < g(x)\} \cup \{x : f(x) > g(x)\}\bigg)^c = \{x : f(x) \geq g(x)\} \cap \{x : f(x) \leq g(x)\} = \{x : f(x) = g(x)\}
\end{align*}
Therefore, using closure under union and complement, we have that $\{x : f(x) = g(x)\}$ is measurable. 
\end{proof}

\section*{Question 4}
Let $f_n : X \to [-\infty, +\infty]$ be a sequence of measurable functions. Prove that the set of points $x \in X$ such that the sequence $f_n(x)$ is convergent (in a wide sense), is a measurable set. 

\begin{proof}$ $
\\Denote $L = \{x : f_n(x)$ is convergent in a wide sense$\}$. Let me first consider a subset of $L$, $L_1 = \{x : f_n(x)$ is convergent to a finite value$\}$. In this case for some $x_0 \in L_1$, the sequence $\{f_n(x_0)\}_{n \in \mathbb{N}}$ is a convergent sequence of real numbers. Since the real numbers are complete, a sequence is convergent if and only if it is Cauchy. Therefore, $L_1$ can also be expressed as $L_1 = \{x : \{f_n(x)\} \subset \mathbb{R}$ is a Cauchy Sequence$\}$. Recall that a sequence is Cauchy if 
\begin{align*}
\forall \; \varepsilon > 0, \; \exists \; N \in \mathbb{N} \text{ such that } m, n \geq N \implies |f_n(x) - f_m(x)| < \varepsilon
\end{align*}
With this definition, we can express $L_1$ as 
\begin{align*}
L_1 &= \{x : \{f_n(x)\} \subset \mathbb{R} \text{ is a Cauchy Sequence}\}\\
&= \{x : \forall \; \varepsilon > 0, \; \exists \; N \in \mathbb{N} \text{ such that } m, n \geq N \implies |f_n(x) - f_m(x)| < \varepsilon \}\\
&= \bigcap_{\varepsilon > 0} \{x : \exists \; N \in \mathbb{N} \text{ such that } m, n \geq N \implies |f_n(x) - f_m(x)| < \varepsilon \}\\
&= \bigcap_{\varepsilon > 0} \bigcup_{N \in \mathbb{N}} \{x : m, n \geq N \implies |f_n(x) - f_m(x)| < \varepsilon \}\\
&= \bigcap_{\varepsilon > 0} \bigcup_{N \in \mathbb{N}} \bigcap_{m, n \geq N} \{x : |f_n(x) - f_m(x)| < \varepsilon \}\\
&= \bigcap_{\varepsilon > 0} \bigcup_{N \in \mathbb{N}} \bigcap_{m, n \geq N} |f_n - f_m|^{-1}\big([-\infty, \varepsilon) \big)
\end{align*}
Notice that $f_n$ and $f_m$ are both measurable, so in fact $f_n - f_m$ is also measurable as proven in class. Thus, since measurability is also closed under absolute value, $|f_n - f_m|$ is measurable. Then, by the property of measurable functions, $|f_n - f_m|^{-1}\big((-\infty, \varepsilon)\big)$ must be a measurable set. Thus, since measurable sets are preserved under countable intersection and union, we know that 
\begin{align*}
\bigcup_{N \in \mathbb{N}} \bigcap_{m, n \geq N} |f_n - f_m|^{-1}\big([-\infty, \varepsilon) \big)
\end{align*}
is measurable. However, we have trouble with the first intersection since $\{\varepsilon : \varepsilon > 0\}$ is not a countable set. However, we can modify the definition of Cauchy, to say instead that 
\begin{align*}
\forall \; k \in \mathbb{N}, \; \exists \; N \in \mathbb{N} \text{ such that } m, n \geq N \implies |f_n(x) - f_m(x)| < \frac{1}{k}
\end{align*}
Then, in fact, we have
\begin{align*}
L_1 = \bigcap_{k \in \mathbb{N}} \bigcup_{N \in \mathbb{N}} \bigcap_{m, n \geq N} |f_n - f_m|^{-1}\bigg(\bigg[-\infty, \frac{1}{k}\bigg) \bigg)
\end{align*}
which as discussed above, must be a measurable set since now the left-most intersection is countable. However, since $L$ is actually the set of points in which $f_n(x)$ converges in a wide sense, we also need to consider when $f_n(x) \to \pm \infty$ as $n \to \infty$. This is equivalent; however, to saying $|f_n(x)| \to +\infty$ as $n \to \infty$. Then, we can define $L_2 = \{x : |f_n(x)| \to +\infty$ as $n \to \infty\}$. Notice, we can write $L_2$ as 
\begin{align*}
L_2 &= \{x : |f_n(x)| \to +\infty \text{ as } n \to \infty\}\\
&= \{x : \forall \; M \in \mathbb{N}, \; \exists \; N \in \mathbb{N} \text{ such that } n \geq N \implies |f_n(x)| > M\}\\
&= \bigcap_{M \in \mathbb{N}} \{x : \exists \; N \in \mathbb{N} \text{ such that } n \geq N \implies |f_n(x)| > M\}\\
&= \bigcap_{M \in \mathbb{N}} \bigcup_{N \in \mathbb{N}} \{x : n \geq N \implies |f_n(x)| > M\}\\
&= \bigcap_{M \in \mathbb{N}} \bigcup_{N \in \mathbb{N}} \bigcap_{n \geq N} \{x : |f_n(x)| > M\}\\
&= \bigcap_{M \in \mathbb{N}} \bigcup_{N \in \mathbb{N}} \bigcap_{n \geq N} |f_n|^{-1}\big((M, +\infty]\big)
\end{align*}
In a similar reasoning as before, we have that this set is measurable, so $L_2$ is measurable. Then, since we have that $L = L_1 \cup L_2$, we can finally conclude that $L$ is measurable. 
\end{proof}



\end{document}