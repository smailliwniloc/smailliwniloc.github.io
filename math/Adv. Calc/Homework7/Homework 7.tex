\documentclass[10pt,a4paper]{article}
\usepackage[utf8]{inputenc}
\usepackage[english]{babel}
\usepackage{csquotes}
\usepackage{amsmath}
\usepackage{amsfonts}
\usepackage{amssymb}
\usepackage{graphicx}
\usepackage[margin=0.5in]{geometry}
\usepackage{amsthm}
\usepackage{enumitem}
\usepackage{tikz}
\usetikzlibrary{calc}
\newtheorem{question}{Question}
\newtheorem*{question*}{Question}
\newtheorem{theorem}{Theorem}
\newtheorem*{theorem*}{Theorem}
\newtheorem{lemma}{Lemma}

\theoremstyle{definition}
\newtheorem{answer}{Answer}
\newtheorem*{answer*}{Answer}


\title{Advanced Calc. Homework 7}
\author{Colin Williams}

\begin{document}
\maketitle

\section*{10.1}
Which of the following sequences are increasing? decreasing? bounded?
\begin{enumerate}[label = (\alph*)]
\item $\displaystyle \frac{1}{n}$
\item $\displaystyle \frac{(-1)^n}{n^2}$
\item $\displaystyle n^5$
\item $\displaystyle \sin\left(\frac{n\pi}{7}\right)$
\item $\displaystyle (-2)^n$
\item $\displaystyle \frac{n}{3^n}$
\end{enumerate}

\begin{answer*}{$ $}
\\Note that (b), (d), and (e) all alternate between positive and negative; thus, none of them can be increasing nor decreasing.
\\Next, see that \boxed{\text{(c) is increasing}} because $n^5 < (n + 1)^5$ for all $n \in \mathbb{N}$.
\\Similarly, \boxed{\text{(a) and (f) are decreasing}} since $\frac{1}{n} > \frac{1}{n + 1}$ for all $n \in \mathbb{N}$ and $\frac{n}{3^n} > \frac{n + 1}{3^{n + 1}}$ for all $n \in \mathbb{N}$.
\\Lastly \boxed{\text{(a), (b), (d), and (f) are all bounded}} since $\left|\frac{1}{n}\right| \leq 1$; $\left|\frac{(-1)^n}{n^2}\right| \leq 1$; $\left|\sin\left(\frac{n\pi}{7}\right)\right| \leq 1$; and $\left|\frac{n}{3^n}\right| \leq \frac{1}{3}$ for all $n \in \mathbb{N}$.
\end{answer*}

\section*{10.2}
Prove that bounded decreasing sequences converge.

\begin{proof}{$ $}
\\Let $(s_n)$ be a bounded decreasing sequence. Let $S$ denote the set $\{s_n : n \in \mathbb{N}\}$. Then by Lemma 4.5 of the Completeness Axiom, we know that $u = \inf(S)$ exists and is a Real Number since $S$ is bounded (in particular, bounded below). I will show that $\lim(s_n) = u$. Let $\varepsilon > 0$, then $u < u + \varepsilon$, so $u + \varepsilon$ is not a lower bound for $S$ since $u$ is, by definition, the greatest lower bound for $S$. Thus, since $u + \varepsilon$ is not a lower bound, there exists some element in $S$ that is smaller than $u + \varepsilon$, i.e., there exists some $N$ such that $s_N < u + \varepsilon$. Furthermore, since $(s_n)$ is decreasing, we have that $s_n \leq s_N$ for all $n \geq N$. Thus, $s_n < u + \varepsilon$ for all $n \geq N$. Furthermore, since $u$ is the infimum of $S$ we know that $u \leq s_n$ for all $n$ and we also know that $u - \varepsilon < u$. Putting this all together, we get that $u - \varepsilon < s_n < u + \varepsilon$ for all $n \geq N \implies |s_n - u| < \varepsilon$ for all $n \geq N$ -- showing that that $(s_n)$ converges and $\lim(s_n) = u$.
\end{proof}

\section*{10.6}
\begin{enumerate}[label = (\alph*)]
\item Let $(s_n)$ be a sequence such that 
\[|s_{n + 1} - s_n| < 2^{-n} \quad \text{for all} \quad n \in \mathbb{N}.\]
Prove $(s_n)$ is a Cauchy sequence and hence a convergent sequence
\item Is the result in (a) true if we only assume $|s_{n + 1} - s_{n}| < \frac{1}{n}$ for all $n \in \mathbb{N}$?
\end{enumerate}

\begin{proof}{\textbf{(a)}}
\\I will start by considering some $m,n \in \mathbb{N}$ with the condition that $n > m$. Using this, I will evaluate $|s_n - s_m|$:
\begin{align*}
|s_n - s_m| &= |(s_n - s_{n - 1}) + (s_{n-1} - s_{n-2}) + \cdots + (s_{m+1} - s_m)|\\
&\leq |s_n - s_{n-1}| + |s_{n-1} - s_{n-2}| + \cdots + |s_{m+1} - s_m| &\text{by Triangle Inequality}\\
&< 2^{-(n-1)} + 2^{-(n-2)} + \cdots + 2^{-m} &\text{by the definition of the sequence}\\
&= \sum_{j = 0}^{n - 1} 2^{-j} - \sum_{j = 0}^{m - 1} 2^{-j}\\
\end{align*}
To evaluate this last term, let's consider $\displaystyle S_k = \sum_{j = 0}^{k}2^{-j} = (1 + 2^{-1} + 2^{-2} + \cdots + 2^{-k})$. Next, note that $S_k - S_k \cdot 2^{-1} = (1 + 2^{-1} + 2^{-2} + \cdots + 2^{-k}) - (2^{-1} + 2^{-2} + 2^{-3} \cdots + 2^{-k - 1}) = 1 - 2^{-k-1}$. Thus, $\displaystyle S_k = \frac{1 - 2^{-k-1}}{1 - 2^{-1}} = 2(1 - 2^{-k-1})$. Using this on the estimate for $|s_n - s_m|$ we get
\begin{align*}
|s_n - s_m| &< 2(1 - 2^{-n}) - 2(1 - 2^{-m}) &\text{by the analysis of $S_k$}\\
&= 2(1 - 2^{-n} - 1 + 2^{-m})\\
&= 2(2^{-m} - 2^{-n})\\
&= 2^{-(m - 1)} - 2^{-(n-1)}\\
&< 2^{-(m-1)} + 2^{-(n-1)} &\text{since $a - b < a + b$ for $a,b$ positive numbers}\\
\end{align*}
Now that we have analyzed $|s_n - s_m|$, we will let $\varepsilon > 0$ be some arbitrary (small) number. Then, let $\displaystyle N = -\log_2(\varepsilon) + 2$. Thus, for $n,m > N$ (and $n > m$) we have:
\begin{align*}
|s_n - s_m| &< 2^{-(m-1)} + 2^{-(n-1)}\\
&< 2^{-(N-1)} + 2^{-(N-1)} &\text{since $n,m > N$}\\
&= 2^{-(N-2)}\\
&= 2^{-(-\log_2(\varepsilon) + 2 - 2)}\\
&= 2^{\log_2(\varepsilon)}\\
&= \varepsilon
\end{align*}
Thus, for $n,m >N$ (and $n > m$) we know that $|s_n - s_m| < \varepsilon$. However, if $n = m$, then $|s_n - s_m| = 0 < \varepsilon$ and if $n < m$ we can relabel the terms appropriately to get the same results as above. Thus, we know that for \textit{any} $n,m > N$ that $|s_n - s_m| < \varepsilon$, proving that $\{s_n\}$ is Cauchy.

\end{proof}

\begin{answer*}{\textbf{(b)}}
\\By following the example from part (a) we can get the following bound on $|s_n - s_m|$ (for $n > m$):
\begin{align*}
|s_n - s_m| &= |(s_n - s_{n - 1}) + (s_{n-1} - s_{n-2}) + \cdots + (s_{m+1} - s_m)|\\
&\leq |s_n - s_{n-1}| + |s_{n-1} - s_{n-2}| + \cdots + |s_{m+1} - s_m| &\text{by Triangle Inequality}\\
&< \frac{1}{n - 1} + \frac{1}{n - 2} + \cdots + \frac{1}{m} &\text{by the definition of the sequence}\\
\end{align*}
This is the best that our estimate can get by using the provided information. However, we can see that this is not enough to show that the sequence is Cauchy. For example, let $n = 2m$ and $\varepsilon = \frac{1}{2}$. Then we have a sum of $m$ terms that are all greater than $\frac{1}{2m}$. Thus, the whole sum is greater than $\frac{1}{2m} \cdot m = \frac{1}{2}$. Therefore, with $n = 2m$, no matter how large we choose our $N$, $|s_n - s_m|$ will never be less than $\varepsilon = \frac{1}{2}$. Thus, this sequence is not Cauchy, so \boxed{\text{the result is not true when we assume }|s_n - s_m| < \frac{1}{n}}
\end{answer*}

\section*{10.7}
Let $S$ be a bounded nonempty subset of $\mathbb{R}$ such that $\sup(S)$ is not in $S$. Prove there is a sequence $(s_n)$ of points in $S$ such that $\lim(s_n) = \sup(S)$. 

\begin{proof}{$ $}
\\Let $S$ be a bounded nonempty subset of $\mathbb{R}$ with $t = \sup(S) \not\in S$. Thus, if we consider $t - \frac{1}{n}$, then note that $t - \frac{1}{n} < t$ for all $n \in \mathbb{N}$, so $t - \frac{1}{n}$ is never an upper bound for $S$ since $t$ is the \underline{least} upper bound. Thus, since $t - \frac{1}{n}$ is not an upper bound, there exists some $s_n \in S$ such that $t - \frac{1}{n} < s_n$. We also know that since $s_n \in S$ that $s_n < t$ as $t$ is the supremum. Thus, we have that for this $s_n$ that $t - \frac{1}{n} < s_n < t$. This same inequality holds for all $n$. Furthermore, we know that $\lim_{n \to \infty}(t - \frac{1}{n}) = t$ and $\lim_{n \to \infty}(t) = t$. Therefore, we can use the Exercise 8.5(a) (the "Squeeze Lemma") to conclude that $\lim_{n \to \infty}(s_n) = t$ just as desired. 
\end{proof}
\section*{10.8}
Let $(s_n)$ be an increasing sequence of positive numbers and define $\displaystyle \sigma_n = \frac{1}{n}(s_1 + s_2 + \cdots + s_n).$ Prove that $(\sigma_n)$ is an increasing sequence.

\begin{proof}{$ $}
\\First, I claim that $s_{k + 1} > \frac{1}{k}(s_1 + s_2 + \cdots + s_k)$ for all $k \in \mathbb{N}$. This is fairly easy to prove since we have the following inequalities which follow from $(s_n)$ being an increasing sequence.
\begin{align*}
s_{k + 1} &> s_k\\
s_k > s_{k-1} \implies s_{k+1} &> s_{k-1}\\
s_{k-1} > s_{k-2} \implies s_{k+1} &> s_{k-2}\\
\vdots\\
s_2 > s_1 \implies s_{k+1} &> s_1
\end{align*}
Summing these $k$ inequalities together, we get $k\cdot s_{k+1} > s_1 + s_2 + \cdots + s_k \implies s_{k+1} > \frac{1}{k}(s_1 + s_2 + \cdots + s_k)$. From here, I will prove the desired result, in other words I will prove that $\sigma_{n + 1} > \sigma_n$ for all $n \in \mathbb{N}$:
\begin{align*}
\sigma_{n+1} &= \frac{1}{n+1}(s_1 + s_2 + \cdots + s_n + s_{n+1}) &\text{by definition}\\
&> \frac{1}{n+1}(s_1 + s_2 + \cdots + s_n + \frac{1}{n}(s_1 + s_2 + \cdots + s_{n})) &\text{by the discussion above}\\
&= \frac{1}{n+1}\left(1 + \frac{1}{n}\right)(s_1 + s_2 + \cdots + s_{n})\\
&= \frac{1}{n+1}\left(\frac{n + 1}{n}\right)(s_1 + s_2 + \cdots + s_{n})\\
&= \frac{1}{n}(s_1 + s_2 + \cdots + s_{n})\\
&= \sigma_{n}
\end{align*}
Thus, $\sigma_{n+1} > \sigma_n$ for all $n \in \mathbb{N}$, so $(\sigma_n)$ is, by definition, an increasing sequence.
\end{proof}

\section*{10.9}
Let $s_1 = 1$ and $\displaystyle s_{n + 1} = \left(\frac{n}{n+1}\right) s_n^2$ for $n \geq 1$.
\begin{enumerate}[label = (\alph*)]
\item Find $s_2$, $s_3$, $s_4$.
\item Show that $\lim(s_n)$ exists.
\item Prove $\lim(s_n) = 0$.
\end{enumerate}

\begin{answer*}{\textbf{(a)}}
\begin{itemize}
\item $\displaystyle s_2 = \left(\frac{1}{2}\right)\cdot 1^2 = \frac{1}{2}$
\item $\displaystyle s_3 = \left(\frac{2}{3}\right)\cdot \left(\frac{1}{2}\right)^2 = \frac{2}{3} \cdot \frac{1}{4} = \frac{1}{6}$
\item $\displaystyle s_4 = \left(\frac{3}{4}\right)\cdot \left(\frac{1}{6}\right)^2 = \frac{3}{4} \cdot \frac{1}{36} = \frac{1}{48} $
\end{itemize}
\end{answer*}

\begin{proof}{\textbf{(b)}}
\\First, I will show that $(s_n)$ is bounded. I claim that $0 < s_n \leq 1$ for all $n \in \mathbb{N}$. I will prove this by induction:
\\\underline{Base Case:}
\\The first few values for $(s_n)$ were already found in part (a) and they all satisfy this bound.
\\\underline{Inductive Step:}
\\Assume that $0 < s_k \leq 1$ for some $k$ and then examine $s_{k+1}$. I will start with the left inequality:
\begin{align*}
0 &< s_k &\text{by the inductive assumption}\\
\implies 0 &< s_k^2 &\text{since the product of 2 positive numbers is positive}\\
\implies \left(\frac{n}{n+1}\right) \cdot 0 &< \left(\frac{n}{n+1}\right) s_k^2 &\text{since we can multiply an inequality by a positive number}\\
\implies 0 &< s_{k+1} &\text{by definition of $s_{k+1}$}
\end{align*}
Next, I will examine the right inequality:
\begin{align*}
s_{k+1} &= \left(\frac{n}{n+1}\right) s_n^2 &\text{by definition of $s_{k+1}$}\\
&< s_n^2 &\text{since $\frac{n}{n+1} < 1$}\\
&\leq 1^2 &\text{by the inductive assumption}\\
&= 1
\end{align*}
Thus, we have shown that $0 < s_n \leq 1$ for all $n$, so $(s_n)$ is bounded. Next, I will show that $(s_n)$ is monotone decreasing:
\begin{align*}
s_{n + 1} &= \left(\frac{n}{n+1}\right) s_n^2\\
&< s_n^2 &\text{since $\frac{n}{n+1} < 1$}\\
&\leq s_n &\text{since $s_n \leq 1$ as shown above}\\
\end{align*}
Thus, $s_{n+1} < s_n$ so $(s_n)$ is monotone decreasing (and bounded as shown before). Therefore, by Theorem 10.2, $(s_n)$ converges. 
\end{proof}

\begin{proof}{\textbf{(c)}}
\\As we have already shown in part (b), $(s_n)$ converges. In other words, we can say that $\lim(s_n) = s$ for $s \in \mathbb{R}$. Similarly, $\lim(s_{n+1}) = s$. From this, we can limits of both sides of the defining equation for $s_{n+1}$:
\begin{align*}
\lim_{n \to \infty}(s_{n+1}) &= \lim_{n \to \infty}\left(\left(\frac{n}{n+1}\right) s_n^2\right)\\
\implies \lim_{n \to \infty}(s_{n+1}) &= \lim_{n \to \infty}\left(\frac{n}{n+1}\right) \cdot \lim_{n \to \infty}(s_n^2) &\text{by the Limit Theorem on Produts}\\
\implies s &= \lim_{n \to \infty}\left(\frac{1}{1 + \frac{1}{n}}\right) \cdot s^2 &\text{by using the convergence of $(s_n)$}\\
\implies s &= \frac{\lim(1)}{\lim(1 + \frac{1}{n})} \cdot s^2 &\text{by the Limit Theorem on Quotients}\\
\implies s &= s^2\\
\implies s = 0 \quad &\text{OR} \quad s = 1
\end{align*} 
However, it is impossible for $s = 1$ since we have shown earlier that $s_n \leq \frac{1}{2}$ for all $n \geq 2$. Thus, \boxed{\lim(s_n) = 0}.
\end{proof}

\section*{10.10}
Let $s_1 = 1$ and $\displaystyle s_{n + 1} = \frac{1}{3}(s_n + 1)$ for $n \geq 1$.
\begin{enumerate}[label = (\alph*)]
\item Find $s_2$, $s_3$, $s_4$.
\item Use induction to show that $s_n > \frac{1}{2}$ for all $n$.
\item Show $(s_n)$ is a decreasing sequence.
\item Show $\lim(s_n)$ exists and find $\lim(s_n)$. 
\end{enumerate}

\begin{answer*}{\textbf{(a)}}
\begin{itemize}
\item $\displaystyle s_2 = \frac{1}{3}(1 + 1) = \frac{2}{3}$
\item $\displaystyle s_3 = \frac{1}{3}\left(\frac{2}{3} + 1\right) = \frac{5}{9}$
\item $\displaystyle s_4 = \frac{1}{3}\left(\frac{5}{9} + 1\right) = \frac{14}{27}$
\end{itemize}
\end{answer*}

\begin{proof}{\textbf{(b)}}
\\I have already covered the base case for induction in part (a) since all of the values I found were greater than $\frac{1}{2}$. Therefore, I will proceed with the inductive step of the proof: Assume that $s_k > \frac{1}{2}$ for some $k \in \mathbb{N}$. Next, I will examine $s_{k+1}$:
\begin{align*}
s_{k+1} &= \frac{1}{3}(s_k + 1)\\
&> \frac{1}{3}\left(\frac{1}{2} + 1\right) &\text{by the inductive assumption}\\
&= \frac{1}{3} \cdot \frac{3}{2}\\
&= \frac{1}{2}
\end{align*}
Thus, we have shown that $s_k > \frac{1}{2} \implies s_{k+1} > \frac{1}{2}$, so by the principle of mathematical induction, $s_n > \frac{1}{2}$ for all $n$. 
\end{proof}

\begin{proof}{\textbf{(c)}}
\\To do this, I will examine $s_{n+1} - s_n$:
\begin{align*}
s_{n+1} - s_n &= \frac{1}{3}(s_n + 1) - s_n &\text{by the definition of $s_{n+1}$}\\
&= \frac{1}{3} - \frac{2}{3}s_n\\
&< \frac{1}{3} - \frac{2}{3}\cdot \frac{1}{2} &\text{since we have shown $s_n > \frac{1}{2}$}\\
&= \frac{1}{3} - \frac{1}{3}\\
&= 0
\end{align*}
Thus, we have shown that $s_{n+1} - s_n < 0 \implies s_{n+1} < s_n$ which means $(s_n)$ is a decreasing sequence. 
\end{proof}

\begin{proof}{\textbf{(d)}}
\\We have shown that $(s_n)$ is a decreasing sequence which means that $(s_n)$ is bounded above by its first term, 1. Also, we have shown that $s_n > \frac{1}{2}$ for all $n$, so $(s_n)$ is bounded below as well. Thus, we know that $(s_n)$ is a bounded monotone sequence, so by Theorem 10.2, $(s_n)$ converges, i.e. $\lim(s_n) = s$. Similarly, we can say $\lim(s_{n+1}) = s$. Furthermore, by taking the limit of both sides of the defining equation for $s_{n+1}$ we get:
\begin{align*}
\lim(s_{n+1}) &= \lim\left(\frac{1}{3}(s_n + 1)\right)\\
\implies s &= \frac{1}{3}(s + 1) &\text{by the Limit Theorems}\\
\implies \frac{2}{3}s &= \frac{1}{3}\\
\implies &\boxed{s = \frac{1}{2}}
\end{align*}
\end{proof}

\section*{10.11}
Let $t_1 = 1$ and $\displaystyle t_{n + 1} = \left[1 - \frac{1}{4n^2}\right] \cdot t_n$ for $n \geq 1$. 
\begin{enumerate}[label = (\alph*)]
\item Show that $\lim(t_n)$ exists.
\item What do you think $\lim(t_n)$ is?
\end{enumerate}

\begin{proof}{\textbf{(a)}}
\\First, it is easy to see that $(t_n)$ is decreasing since $\displaystyle \left[1 - \frac{1}{4n^2}\right] < 1 \implies \left[1 - \frac{1}{4n^2}\right]\cdot t_n < t_n \implies t_{n+1} < t_n$ for all $n$. This clearly means $(t_n)$ is bounded above by its first term, 1. I will next show that it is bounded below by 0 using induction. We already know that the first term is greater than 0 because we are given it is 1. Just to be safe we could calculate $t_2 = \frac{3}{4} > 0$. Next, assume $t_k > 0$ for some $k \in \mathbb{N}$. Then look at $t_{k+1}$:
\begin{align*}
t_{k + 1} &= \left[1 - \frac{1}{4k^2}\right] \cdot t_k\\
&> \left[1 - \frac{1}{4k^2}\right] \cdot 0 &\text{by the inductive assumption and that the left term is positve}\\
&= 0
\end{align*}
Thus, $t_k > 0 \implies t_{k+1} > 0$ and $t_1 > 0$ so by the principle of Mathematical induction, we know that $t_n > 0$ for all $n$. Thus, $(t_n)$ is bounded above and below and is a decreasing sequence. Therefore, by Theorem 10.2, $(t_n)$ converges, i.e. $\lim(t_n)$ exists. 
\end{proof}

\begin{answer*}{\textbf{(b)}}
\\\boxed{\text{I think the limit is } \frac{2}{\pi}} since this resembles the reciprocal of the famous Wallis Product which converges to $\displaystyle \frac{\pi}{2}$. This guess also seems reasonably close to the first few terms. 
\end{answer*}

\section*{10.12}
Let $t_1 = 1$ and $\displaystyle t_{n + 1} = \left[1 - \frac{1}{(n + 1)^2}\right]\cdot t_n$ for $n \geq 1$.
\begin{enumerate}[label = (\alph*)]
\item Show that $\lim(t_n)$ exists.
\item What do you think $\lim(t_n)$ is?
\item Use induction to show that $\displaystyle t_n = \frac{n + 1}{2n}$.
\item Repeat part (b)
\end{enumerate}

\begin{proof}{\textbf{(a)}}
\\First, it is easy to see that $(t_n)$ is decreasing since $\displaystyle \left[1 - \frac{1}{(n + 1)^2}\right] < 1 \implies \left[1 - \frac{1}{(n+1)^2}\right]\cdot t_n < t_n \implies t_{n+1} < t_n$ for all $n$. This clearly means $(t_n)$ is bounded above by its first term, 1. I will next show that it is bounded below by 0 using induction. We already know that the first term is greater than 0 because we are given it is 1. Just to be safe we could calculate $t_2 = \frac{3}{4} > 0$. Next, assume $t_k > 0$ for some $k \in \mathbb{N}$. Then look at $t_{k+1}$:
\begin{align*}
t_{k + 1} &= \left[1 - \frac{1}{(k+1)^2}\right] \cdot t_k\\
&> \left[1 - \frac{1}{(k+1)^2}\right] \cdot 0 &\text{by the inductive assumption and that the left term is positve}\\
&= 0
\end{align*}
Thus, $t_k > 0 \implies t_{k+1} > 0$ and $t_1 > 0$ so by the principle of Mathematical induction, we know that $t_n > 0$ for all $n$. Thus, $(t_n)$ is bounded above and below and is a decreasing sequence. Therefore, by Theorem 10.2, $(t_n)$ converges, i.e. $\lim(t_n)$ exists.
\end{proof}

\begin{answer*}{\textbf{(b)}}
\\The terms appear to be approaching $\frac{1}{2}$ so \boxed{\text{I think the limit is }\frac{1}{2}}
\end{answer*}

\begin{proof}{\textbf{(c)}}
\\\underline{Base Case:}
\\We are given that $t_1 = 1$. Also, $\displaystyle \frac{1 + 1}{2(1)} = 1$.
\\For $n = 2$, we found earlier that $t_2 = \frac{3}{4}$. Also, $\displaystyle \frac{2 + 1}{2(2)} = \frac{3}{4}$.
\\\underline{Inductive step:}
\\Assume that $\displaystyle t_k = \frac{k+1}{2k}$ for some $k \in \mathbb{N}$. Then examine $t_{k+1}$:
\begin{align*}
t_{k+1} &= \left[1 - \frac{1}{(k + 1)^2}\right]\cdot t_k &\text{by the definition of $t_{k+1}$}\\
&= \left[1 - \frac{1}{(k + 1)^2}\right]\cdot \frac{k + 1}{2k} &\text{by the inductive assumption}\\
&= \frac{(k + 1)^2 - 1}{(k + 1)^2} \cdot \frac{k + 1}{2k}\\
&= \frac{k^2 + 2k + 1 - 1}{2k(k + 1)}\\
&= \frac{k^2 + 2k}{2k(k+1)}\\
&= \frac{k + 2}{2(k+1)} = \frac{(k + 1) + 1}{2(k+1)}
\end{align*}
Thus, we have shown that the formula holding for $k$ implies that it holds for $k + 1$ and we have shown that it holds for $n = 1$, so by the principle of mathematical induction, the formula must hold for all $n$. 
\end{proof}

\begin{answer*}{\textbf{(d)}}
\\I agree with my previous assessment that \boxed{\text{the limit is } \frac{1}{2}} which can be checked by using basic limit theorems on $\displaystyle \frac{n + 1}{2n}$
\end{answer*}

\section*{11.1}
Let $a_n = 3 + 2(-1)^n$ for $n \in \mathbb{N}$.
\begin{enumerate}[label = (\alph*)]
\item List the first eight terms of the sequence $(a_n)$.
\item Give a subsequence that is constant [takes a single value]. Specify the selection function $\sigma$. 
\end{enumerate}

\begin{answer*}{\textbf{(a)}}
\\The first 8 terms are: $1, 5, 1, 5, 1, 5, 1, 5$
\end{answer*}

\begin{answer*}{\textbf{(b)}}
\\Choose $\sigma(k) = n_k = 2k$, then $(a_{n_k})$ is the constant value 5.
\\Could also choose $\sigma(k) = n_k = 2k + 1$ to make $(a_{n_k})$ the constant value 1.
\end{answer*}

\section*{11.2}
Consider the sequences defined as follows:
\begin{align*}
a_n = (-1)^n, && b_n = \frac{1}{n}, && c_n = n^2, && d_n = \frac{6n + 4}{7n - 3}.
\end{align*}
\begin{enumerate}[label = (\alph*)]
\item For each sequence, give an example of a monotone subsequence.
\item For each sequence, give its set of subsequential limits.
\item For each sequence, give its $\lim \sup$ and $\lim \inf$.
\item Which of the sequences converges? diverges to $+\infty$? diverges to $-\infty$?
\item Which of the sequences is bounded?
\end{enumerate}

\begin{answer*}{$a_n$}
\begin{enumerate}[label = (\alph*)]
\item A monotone subsequence could utilize \boxed{\sigma(k) = n_k = 2k} which then makes $a_{n_k} =(-1)^{2k} = 1$ for all $k$. Thus, $a_{n_k}$ is a monotone subsequence.
\item Let $S$ be the set of subsequential limits for $a_n$. Since $a_n = 1$ or $-1$ for all $n$, \boxed{S = \{-1,1\}}.
\item I will use the fact that $\lim \sup(a_n) = \sup(S)$ and $\lim \inf(a_n) = \inf(S)$. Thus, \boxed{\lim \sup(a_n) = 1 \text{ and } \lim \inf(a_n) = -1}.
\item This sequence \boxed{\text{does not converge nor diverge to }\pm \infty}
\item \boxed{\text{This sequence is bounded}} since $-1 \leq a_n \leq 1$ for all $n$.
\end{enumerate}
\end{answer*}

\begin{answer*}{$b_n$}
\begin{enumerate}[label = (\alph*)]
\item A monotone subsequence could utilize \boxed{\sigma(k) = n_k = k} which then makes $b_{n_k} = \frac{1}{k} > \frac{1}{k + 1} = b_{n_{k+1}}$ for all $k$. Thus, $b_{n_k}$ is a monotone decreasing subsequence.
\item Let $S$ be the set of subsequential limits for $b_n$. Since $b_n$ converges, then all subsequential limits will be the same, so \boxed{S = \{0\}}.
\item I will use the fact that $\lim \sup(b_n) = \sup(S)$ and $\lim \inf(b_n) = \inf(S)$. Thus, \boxed{\lim \sup(b_n) = 0 \text{ and } \lim \inf(b_n) = 0}.
\item This sequences \boxed{\text{converges to 0}}
\item \boxed{\text{This sequence is bounded}} since $0 < b_n \leq 1$ for all $n$.
\end{enumerate}
\end{answer*}

\begin{answer*}{$c_n$}
\begin{enumerate}[label = (\alph*)]
\item A monotone subsequence could utilize \boxed{\sigma(k) = n_k = k} which then makes $c_{n_k} = k^2 < (k+1)^2 = c_{n{_k+1}}$ for all $k$. Thus, $c_{n_k}$ is a monotone increasing subsequence.
\item Let $S$ be the set of subsequential limits for $c_n$. Since $c_n$ diverges to $+\infty$, \boxed{S = \{+\infty\}}.
\item I will use the fact that $\lim \sup(a_n) = \sup(S)$ and $\lim \inf(a_n) = \inf(S)$. Thus, \boxed{\lim \sup(c_n) = +\infty \text{ and } \lim \inf(c_n) = +\infty}.
\item This sequence \boxed{\text{diverges to }+\infty}
\item \boxed{\text{This sequence is not bounded}} since $c_n = n^2 > M$ for any $M$ at some point $n > N$.
\end{enumerate}
\end{answer*}

\begin{answer*}{$d_n$}
\begin{enumerate}[label = (\alph*)]
\item A monotone subsequence could utilize \boxed{\sigma(k) = n_k = k} which then makes $d_{n_k} = \frac{6k + 4}{7k - 3} > \frac{6(k+1) + 4}{7(k + 1) - 3} = d_{n_{k+1}}$ for all $k$. Thus, $d_{n_k}$ is a monotone decreasing subsequence.
\item Let $S$ be the set of subsequential limits for $d_n$. Since $d_n$ converges, all subsequential limits will be the same, so \boxed{S = \bigg\{\frac{6}{7}\bigg\}}.
\item I will use the fact that $\lim \sup(d_n) = \sup(S)$ and $\lim \inf(d_n) = \inf(S)$. Thus, \boxed{\lim \sup(d_n) = \frac{6}{7} \text{ and } \lim \inf(d_n) = \frac{6}{7}}.
\item This sequence \boxed{\text{converges to }\frac{6}{7}}
\item \boxed{\text{This sequence is bounded}} since $\frac{6}{7} < d_n \leq \frac{10}{4}$ for all $n$.
\end{enumerate}
\end{answer*}



\section*{11.3}
Consider the sequences defined as follows:
\begin{align*}
s_n = \cos\left(\frac{n\pi}{3}\right), && t_n = \frac{3}{4n + 1}, && u_n = \left(-\frac{1}{2}\right)^n, && v_n = (-1)^n + \frac{1}{n}.
\end{align*}
\begin{enumerate}[label = (\alph*)]
\item For each sequence, give an example of a monotone subsequence.
\item For each sequence, give its set of subsequential limits.
\item For each sequence, give its $\lim \sup$ and $\lim \inf$.
\item Which of the sequences converges? diverges to $+\infty$? diverges to $-\infty$?
\item Which of the sequences is bounded?
\end{enumerate}

\begin{answer*}{$s_n$}
\begin{enumerate}[label = (\alph*)]
\item A monotone subsequence could utilize \boxed{\sigma(k) = n_k = 6k} which then makes $s_{n_k} = \cos\left(\frac{6k \pi}{3}\right) = \cos(2k \pi) = 1$ for all $k$. Thus, $s_{n_k}$ is a monotone subsequence.
\item Let $S$ be the set of subsequential limits for $s_n$. Next, notice $s_n$ only has distinct values for $n = 6k, 6k + 1, 6k + 2$, or $6k + 3$. Thus, using these respective values, we get \boxed{S = \bigg\{1, \frac{1}{2}, -\frac{1}{2}, -1\bigg\}}.
\item I will use the fact that $\lim \sup(s_n) = \sup(S)$ and $\lim \inf(s_n) = \inf(S)$. Thus, \boxed{\lim \sup(s_n) = 1 \text{ and } \lim \inf(s_n) = -1}.
\item This sequence \boxed{\text{does not converge nor diverge to }\pm \infty}
\item \boxed{\text{This sequence is bounded}} since $-1 \leq s_n \leq 1$ for all $n$.
\end{enumerate}
\end{answer*}

\begin{answer*}{$t_n$}
\begin{enumerate}[label = (\alph*)]
\item A monotone subsequence could utilize \boxed{\sigma(k) = n_k = k} which then makes $t_{n_k} = \frac{3}{4k + 1} > \frac{3}{4(k+1) + 1} = t_{n_{k+1}}$ for all $k$. Thus, $t_{n_k}$ is a monotone decreasing subsequence.
\item Let $S$ be the set of subsequential limits for $t_n$. Since $t_n$ converges, all subsequential limits will be the same, so \boxed{S = \{0\}}.
\item I will use the fact that $\lim \sup(t_n) = \sup(S)$ and $\lim \inf(t_n) = \inf(S)$. Thus, \boxed{\lim \sup(t_n) = 0 \text{ and } \lim \inf(t_n) = 0}.
\item This sequence \boxed{\text{converges to 0}}
\item \boxed{\text{This sequence is bounded}} since $0 < t_n \leq \frac{3}{5}$ for all $n$.
\end{enumerate}
\end{answer*}

\begin{answer*}{$u_n$}
\begin{enumerate}[label = (\alph*)]
\item A monotone subsequence could utilize \boxed{\sigma(k) = n_k = 2k} which then makes $u_{n_k} = (-\frac{1}{2})^{2k} = \frac{1}{4^k} > \frac{1}{4^{k+1}} = u_{n_{k+1}}$ for all $k$. Thus, $u_{n_k}$ is a monotone decreasing subsequence.
\item Let $S$ be the set of subsequential limits for $u_n$. Since $u_n$ converges, all subsequential limits will be the same, so \boxed{S = \{0\}}.
\item I will use the fact that $\lim \sup(u_n) = \sup(S)$ and $\lim \inf(u_n) = \inf(S)$. Thus, \boxed{\lim \sup(u_n) = 0 \text{ and } \lim \inf(u_n) = 0}.
\item This sequence \boxed{\text{converges to 0}}
\item \boxed{\text{This sequence is bounded}} since $-\frac{1}{2} \leq u_n \leq \frac{1}{4}$ for all $n$.
\end{enumerate}
\end{answer*}

\begin{answer*}{$v_n$}
\begin{enumerate}[label = (\alph*)]
\item A monotone subsequence could utilize \boxed{\sigma(k) = n_k = 2k} which then makes $v_{n_k} = (-1)^{2k} + \frac{1}{2k} = 1 + \frac{1}{2k} > 1 + \frac{1}{2(k+1)} = v_{n_{k+1}}$ for all $k$. Thus, $v_{n_k}$ is a monotone decreasing subsequence.
\item Let $S$ be the set of subsequential limits for $v_n$. Since all subsequences of $v_n$ converge to either $-1$ or $1$, we know that \boxed{S = \{-1 , 1\}}.
\item I will use the fact that $\lim \sup(v_n) = \sup(S)$ and $\lim \inf(v_n) = \inf(S)$. Thus, \boxed{\lim \sup(v_n) = 1 \text{ and } \lim \inf(v_n) = -1}.
\item This sequence \boxed{\text{does not converge nor diverge to }\pm \infty}
\item \boxed{\text{This sequence is bounded}} since $-1 < s_n \leq 1.5$ for all $n$.
\end{enumerate}
\end{answer*}

\section*{11.5}
Let $(q_n)$ be an enumeration of all the rationals in the interval $(0,1]$.
\begin{enumerate}[label = (\alph*)]
\item Give the set of subsequential limits for $(q_n)$.
\item Give the values of $\lim \sup(q_n)$ and $\lim \inf(q_n)$.
\end{enumerate}

\begin{answer*}{\textbf{(a)}}
\\Let $S$ be the set of all subsequential limits for $(q_n)$. Since we can find any subsequence of $(q_n)$ that converges to any real number in $[0,1]$, then \boxed{S = [0,1]}
\end{answer*}

\begin{answer*}{\textbf{(b)}}
\\Based off of the result from part (a) and that $\lim \sup (q_n) = \sup(S)$ and $\lim \inf (q_n) = \inf(S)$, then we have that \boxed{\lim \sup (q_n) = 1 \text{ and } \lim \inf (q_n) = 0}
\end{answer*}
\section*{11.7}
Let $(r_n)$ be an enumeration of the set $\mathbb{Q}$ of all rational numbers. Show that there exists a subsequence $(r_{n_k})$ such that $\lim_{k \to \infty}(r_{n_k}) = +\infty$. 

\begin{proof}{$ $}
\\Theorem 11.2 says that if some sequence $(s_n)$ is unbounded above, then it has a subsequence with limit $+\infty$. Thus, since $(r_n)$ is an enumeration of $\mathbb{Q}$ and $\mathbb{Q}$ is unbounded above, then $(r_n)$ is also unbounded above. This shows that there exists some subsequence $(r_{n_k})$ such that $\lim_{k \to \infty}(r_{n_k}) = +\infty$.
\end{proof}

\section*{11.8}
Use the definition of $\lim \sup$ and $\lim \inf$ and the result from Exercise 5.4 to prove that $\lim \inf(s_n) = -\lim \sup (-s_n)$ for every sequence $(s_n)$. 

\begin{proof}{$ $}
\\Exercise 5.4 tells us that for any nonempty subset $S$ of $\mathbb{R}$, that $\inf(S) = -\sup(-S)$. The definition of $\lim \sup$ and $\lim \inf$ are:
\begin{equation}
\lim \sup (s_n) = \lim_{N \to \infty} \sup\{s_n : n > N\}
\end{equation}
\begin{equation}
\lim \inf (s_n) = \lim_{N \to \infty} \inf\{s_n : n > N\}
\end{equation}
Thus, by starting with the definition of $\lim \inf(s_n)$ we can get:
\begin{align*}
\lim \inf (s_n) &= \lim_{N \to \infty} \inf\{s_n : n > N\} &\text{by (2)}\\
&= \lim_{N \to \infty} (-\sup\{-s_n : n > N\}) &\text{by Exercise 5.4}\\
&= - \lim_{N \to \infty} \sup\{-s_n : n > N\} &\text{by a basic Limit Theorem}\\
&= -\lim \sup(-s_n) &\text{by (1)}
\end{align*}
Thus, proving the desired equality.
\end{proof}

\end{document}