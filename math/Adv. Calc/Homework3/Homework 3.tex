\documentclass[10pt,a4paper]{article}
\usepackage[utf8]{inputenc}
\usepackage[english]{babel}
\usepackage{amsmath}
\usepackage{amsfonts}
\usepackage{amssymb}
\usepackage{graphicx}
\usepackage[margin=0.5in]{geometry}
\usepackage{amsthm}
\usepackage{enumitem}
\usepackage{tikz}
\usetikzlibrary{calc}
\newtheorem{question}{Question}
\newtheorem*{question*}{Question}
\newtheorem{theorem}{Theorem}
\newtheorem*{theorem*}{Theorem}
\newtheorem{lemma}{Lemma}

\theoremstyle{definition}
\newtheorem{answer}{Answer}
\newtheorem*{answer*}{Answer}


\title{Advanced Calc. Homework 3}
\author{Colin Williams}

\begin{document}
\maketitle

\section*{Prove Theorem 3.5}
\begin{theorem*}{\textbf{3.5}}
\\The following holds $\forall \; a,b \in \mathbb{R}$:
\begin{enumerate}[label = (\roman*)]
\item $|a| \geq 0$
\item $|ab| = |a| \cdot |b|$
\item $|a + b| \leq |a| + |b|$
\end{enumerate}
\end{theorem*}

\begin{proof}{\textbf{(i)}}
\\\underline{Case 1:} If $a \geq 0$, then $|a| = a$, so clearly $|a| \geq 0$.
\\\underline{Case 2:} If $a \leq 0$, then $|a| = -a$, so by Theorem 3.2 (i), we can say $-a \geq -0$. Since we've already shown that $0 = -0$ and we know that $|a| = -a$, it is clear that this implies $|a| \geq 0$.
\\Since $a \leq 0$ or $a \geq 0$ constitute all possible cases for $a$, we have shown that $|a| \geq 0$ for all $a$. 
\end{proof}

\begin{proof}{\textbf{(ii)}}
\\\underline{Case 1:} If $a \geq 0$ and $b \geq 0$, then $ab \geq 0$ by Theorem 3.2 (iii). Thus, $|a| = a$, $|b| = b$, and $|ab| = ab$. So,
\begin{align*}
ab &= a \cdot b\\
\implies |ab| &= |a| \cdot |b|
\end{align*}
\\\underline{Case 2:} If $a \geq 0$ and $b \leq 0$, then $ab \leq 0 \cdot b$ by Theorem 3.2 (ii) $\implies ab \leq 0$ by Theorem 3.1 (ii) and the commutative law. Thus, $|a| = a$, $|b| = -b$, and $|ab| = -ab$. So,
\begin{align*}
ab &= a \cdot b\\
\implies (ba) &= b \cdot a &\text{by the commutative law}\\
\implies -(ba) &= (-b) \cdot a &\text{by Theorem 3.1 (iii)}\\
\implies -(ab) &= a \cdot (-b) &\text{by the commutative law}\\
\implies |ab| &= |a| \cdot |b| &\text{by above comments}
\end{align*}
\\\underline{Case 3:} If $a \leq 0$ and $b \geq 0$, then $ba \leq 0 \cdot a$ by Theorem 3.2 (ii) $\implies ab \leq 0$ by Theorem 3.1 (ii) and the commutative law. Thus, $|a| = -a$, $|b| = b$, and $|ab| = -ab$. So, 
\begin{align*}
ab &= a \cdot b\\
\implies -ab &= (-a) \cdot b &\text{by Theorem 3.1 (iii)}\\
\implies |ab| &= |a| \cdot |b| &\text{by above comments}
\end{align*}
\\\underline{Case 4:} If $a \leq 0$ and $b \leq 0$, then $-a \geq 0$ and $-b \geq 0$ by Theorem 3.2 (i) and the fact that $0 = -0$. Thus, by Theorem 3.2 (iii) $(-a)(-b) \geq 0 \implies ab \geq 0$ by Theorem 3.1 (iv). Thus, $|a| = -a$, $|b| = -b$, and $|ab| = ab$. So,
\begin{align*}
ab &= a \cdot b\\
\implies ab &= (-a) \cdot (-b) &\text{by Theorem 3.1 (iv)}\\
\implies |ab| &= |a| \cdot |b| &\text{by above comments}
\end{align*}
\end{proof}

\begin{proof}{\textbf{(iii)}}
\\Since part (i) of this proof tells us that $|a| \geq 0$ for all $a$, we can also say that $-|a| \leq 0$ by Theorem 3.2 (i) and the fact that $0 = -0$. Also, since $a = |a|$ or $a = -|a|$ for every $a$, we can conclude that $-|a| \leq a \leq |a|$. Similarly, $-|b| \leq b \leq |b|$ for any $b$ as well. Starting with $-|a| \leq a$, we can get:
\begin{align*}
-|a| + (-|b|) &\leq a + (-|b|) &\text{by O4}\\
\implies -(|a| + |b|) &\leq a + b &\text{by DL on the left, and the fact that }-|b| \leq b\\
\implies -(a + b) &\leq |a| + |b| &\text{by Theorem 3.2 (i) and the fact that } -(-c) = c\\
\end{align*}
Similarly, if we start with $a \leq |a|$, we can get:
\begin{align*}
a + b &\leq |a| + b &\text{by O4}\\
&\leq |a| + |b| &\text{since } b \leq |b|\\
\end{align*}
Thus, since $|a + b|$ either equals $a + b$ or $-(a + b)$ and both of those we have shown are less than or equal to $|a| + |b|$, we can conclude that $|a + b| \leq |a| + |b|$. 
\end{proof}

\section*{4.1, 4.2, 4.3, 4.4}

\begin{question*}
For each of the following sets:
\begin{itemize}
\item \textbf{Question 4.1} If it is bounded above, list 3 upper bounds; otherwise, write Not Bounded Above.
\item \textbf{Question 4.2} If it is bounded below, list 3 lower bounds; otherwise, write Not Bounded Below.
\item \textbf{Question 4.3} Give its supremum if it has one; otherwise, write No Supremum.
\item \textbf{Question 4.4} Give its infimum if it has one; otherwise, write No Infimum.
\end{itemize}
\end{question*}
\begin{enumerate}[label = (\alph*)]
\item $S = [0,1]$ 
\begin{itemize}
\item $1, 1.5, \sqrt{2}$
\item $0, -0.5, -2$
\item $\sup(S) = 1$
\item $\inf(S) = 0$
\end{itemize}
\item $S = (0,1)$
\begin{itemize}
\item $1, 17, \pi$
\item $0, -5, -22$
\item $\sup(S) = 1$
\item $\inf(S) = 0$
\end{itemize}
\item $S = \{2,7\}$
\begin{itemize}
\item $7, 11, 8.8$
\item $2, 1, 0$
\item $\sup(S) = 7$
\item $\inf(S) = 2$
\end{itemize}
\item $S = \{\pi, e\}$
\begin{itemize}
\item $4, 5, 6$
\item $0, 1, 2$
\item $\sup(S) = \pi$
\item $\inf(S) = e$
\end{itemize}
\item $S = \{\frac{1}{n} : n \in \mathbb{N}\}$
\begin{itemize}
\item $2, 3, 4$
\item $0, -1, -2$
\item $\sup(S) = 1$ since the largest element occurs when $n = 1$ and corresponds to $\frac{1}{1} = 1$
\item $\inf(S) = 0$ since any $r > 0$ also satisfies $r \geq \frac{1}{N}$ for $N \geq \lceil\frac{1}{r}\rceil$, so $r$ cannot be a lower bound.
\end{itemize}
\item $S = \{0\}$
\begin{itemize}
\item $1, 2, 3$
\item $-1, -2, -3$
\item $\sup(S) = 0$
\item $\inf(S) = 0$
\end{itemize}
\item $S = [0,1] \cup [2,3]$
\begin{itemize}
\item $3, 4, 5$
\item $0, -1, -2$
\item $\sup(S) = 3$
\item $\inf(S) = 0$
\end{itemize}
\item $S = \cup_{n = 1}^{\infty}[2n, 2n + 1]$
\begin{itemize}
\item Not Bounded Above
\item $2, 1, 0$
\item No Supremum
\item $\inf(S) = 2$ since the interval with the smallest elements occurs when $n = 1$ and corresponds to $[2,3]$ with minimum = 2.
\end{itemize}
\item $S = \cap_{n = 1}^{\infty}[-\frac{1}{n}, 1 + \frac{1}{n}]$
\\First, note that $S$ simply equals $[0,1]$ since all $r \in [0,1]$ are in $S$ and any $r < 0$ or $> 1$ of the form $r = -\varepsilon$ or $r = 1 + \varepsilon$ is not contained in the set $[-\frac{1}{N}, 1 + \frac{1}{N}]$ for $N \geq \lceil\frac{1}{\varepsilon}\rceil$ which is a requirement to belong to $S$. 
\begin{itemize}
\item $2, 3, 4$
\item $-1, -2, -3$
\item $\sup(S) = 1$
\item $\inf(S) = 0$
\end{itemize}
\item $S = \{1 - \frac{1}{3^n} : n \in \mathbb{N}\}$
\begin{itemize}
\item $1, 2, 3$
\item $0, -1, -2$
\item $\sup(S) = 1$ since any $r < 1$ cannot be an upper bound since $r < 1 - \frac{1}{3^N}$ for $N \geq \lceil\log_3(\frac{1}{1-r})\rceil$. This value of $N$ was found by solving $r < 1 - \frac{1}{3^N}$ for $N$. 
\item $\inf(S) = \frac{2}{3}$ since the smallest element in the list occurs at $n = 1$ and corresponds to $1 - \frac{1}{3} = \frac{2}{3}$.
\end{itemize}
\item $S = \{n + \frac{(-1)^n}{n} : n \in \mathbb{N}\}$
\begin{itemize}
\item Not Bounded Above
\item $0, -1, -2$
\item No Supremum
\item $\inf(S) = 0$ since the smallest element occurs at $n = 1$ and corresponds to $1 + \frac{(-1)^1}{1} = 1 - 1 =0$.
\end{itemize}
\item $S = \{r \in \mathbb{Q} : r < 2\}$
\begin{itemize}
\item $3, 4, 5$
\item Not Bounded Below
\item $\sup(S) = 2$ since 2 is greater than all elements of $S$ and any $r < 2$ also has some $r < \frac{p}{q} < 2$
\item No Infimum
\end{itemize}
\item $S = \{r \in \mathbb{Q} : r^2 < 4\}$
\begin{itemize}
\item $3, 4, 5$
\item $-3, -4, -5$
\item $\sup(S) = 2$ by the same argument as before
\item $\inf(S) = -2$ by a similar argument as before
\end{itemize}
\item $S = \{r \in \mathbb{Q} : r^2 < 2\}$
\begin{itemize}
\item $2, 3, 4$
\item $-2, -3, -4$
\item $\sup(S) = \sqrt{2}$ by the same argument as (m) but with $\sqrt{2}$ instead of 2
\item $\inf(S) = -\sqrt{2}$ by the same argument as (m) but with $\sqrt{2}$ instead of 2
\end{itemize}
\item $S = \{x \in \mathbb{R} : x < 0\}$
\begin{itemize}
\item $0, 1, 2$
\item Not Bounded Below
\item $\sup(S) = 0$ since 0 is greater than all elements in $S$ and any $r < 0$ also satisfies $r < s < 0$ for $s \in S$. In particular, choose $s = \frac{r}{2}$.
\item No Infimum
\end{itemize}
\item $S = \{1, \frac{\pi}{3}, \pi^2, 10\}$
\begin{itemize}
\item $11, 12, 13$
\item $0, -1, -2$
\item $\sup(S) = 10$
\item $\inf(S) = 1$
\end{itemize}
\item $S = \{0, 1, 2, 4, 8, 16\}$
\begin{itemize}
\item $32, 64, 128$
\item $-1, -2, -4$
\item $\sup(S) = 16$
\item $\inf(S) = 0$
\end{itemize}
\item $S = \cap_{n = 1}^{\infty} (1 - \frac{1}{n}, 1 + \frac{1}{n})$
\\First, note that $S = {1}$. This is because 1 is in every set we are taking the intersection of and any number $r$ of the form $r = 1 \pm \varepsilon \not\in S$ because that $r$ does not belong in the set $(1 - \frac{1}{N}, 1 + \frac{1}{N})$ for $N \geq \lceil\frac{1}{\varepsilon}\rceil$.
\begin{itemize}
\item $2, 3, 4$
\item $0, -1, -2$
\item $\sup(S) = 1$
\item $\inf(S) = 1$
\end{itemize}
\item $S = \{\frac{1}{n} : n \in \mathbb{N}$ and $n$ is prime$\}$
\begin{itemize}
\item $2, 3, 4$
\item $-1, -2, -3$
\item $\sup(S) = \frac{1}{2}$ since the largest element of the set occurs when $n = 2$ and corresponds to $\frac{1}{2}$
\item $\inf(S) = 0$ since 0 is less than all elements of $S$ and since there are infinitely many primes, for any $r > 0$ we can also find a prime $N$ such that $r > \frac{1}{N}$.
\end{itemize}
\item $S = \{x \in \mathbb{R} : x^3 < 8\}$
\begin{itemize}
\item $3, 4, 5$
\item Not Bounded Below
\item $\sup(S) = 2$ by similar reasoning as part (l).
\item No Infimum
\end{itemize}
\item $S = \{x^2 : x \in \mathbb{R}\}$
\begin{itemize}
\item Not Bounded Above
\item $-1, -2, -3$
\item No Supremum
\item $\inf(S) = 0$ since the smallest element of this set happens when $x = 0$ and corresponds to $0^2 = 0$. 
\end{itemize}
\item $S = \{\cos(\frac{n\pi}{3}) : n \in \mathbb{N}\}$
\begin{itemize}
\item $2, 3, 4$
\item $-2, -3, -4$
\item $\sup(S) = 1$ since the largest element of this set happens when $n$ is of the form $n = 6k$ for $k \in \mathbb{N}$ and corresponds to $\cos(\frac{6k\pi}{3}) = \cos(2k\pi) = 1$.
\item $\inf(S) = -1$ since the smallest element of this set happens when $n$ is of the form $n = 6k + 3$ for $k \in \mathbb{N}$ and corresponds to $\cos(\frac{(6k + 3)\pi}{3}) = \cos(2k\pi + \pi) = \cos(\pi) = -1$.
\end{itemize}
\item $S = \{\sin(\frac{n\pi}{3}) : n \in \mathbb{N}\}$
\begin{itemize}
\item $1, 2, 3$
\item $-1, -2, -3$
\item $\sup(S) = \frac{\sqrt{3}}{2}$ since the largest element of this set happens when $n$ is of the form $n = 6k + 1$ or $6k + 2$ for $k \in \mathbb{N}$ and this corresponds to $\sin(\frac{(6k + 1)\pi}{3}) = \sin(2k\pi + \frac{\pi}{3}) = \sin(\frac{\pi}{3}) = \frac{\sqrt{3}}{2}$ or $\sin(\frac{(6k + 2)\pi}{3}) = \sin(2k\pi + \frac{2\pi}{3}) = \sin(\frac{2\pi}{3}) = \frac{\sqrt{3}}{2}$
\item $\inf(S) = -\frac{\sqrt{3}}{2}$ since the smallest element of this set happens when $n$ is of the form $n = 6k + 4$ or $6k + 5$ for $k \in \mathbb{N}$ and this corresponds to $\sin(\frac{(6k + 4)\pi}{3}) = \sin(2k\pi + \frac{4\pi}{3}) = \sin(\frac{4\pi}{3}) = -\frac{\sqrt{3}}{2}$ or $\sin(\frac{(6k + 5)\pi}{3}) = \sin(2k\pi + \frac{5\pi}{3}) = \sin(\frac{5\pi}{3}) = -\frac{\sqrt{3}}{2}$
\end{itemize}
\end{enumerate}

\section*{4.5}

\begin{question*}
Let $S$ be a nonempty subset of $\mathbb{R}$ that is bounded above. Prove that if $\sup(S)$ belongs to $S$, then $\sup(S) = \max(S)$. 
\end{question*}

\begin{proof}{$ $}
\\Since $\sup(S)$ must be an upper bound of $S$, then by definition of an upper bound, $s \leq \sup(S)$ for all $s \in S$. Also since, by assumption, $\sup(S) \in S$, then $\sup(S)$ satisfies the definition of $\max(S)$, so $\sup(S) = \max(S)$.
\end{proof}

\section*{4.7}

\begin{question*}
Let $S$ and $T$ be nonempty bounded subsets of $\mathbb{R}$.
\begin{enumerate}[label = (\alph*)]
\item Prove that if $S \subseteq T$, then $\inf(T) \leq \inf(S) \leq \sup(S) \leq \sup(T)$.
\item Prove that $\sup(S \cup T) = \max\{\sup(S), \sup(T)\}$.
\end{enumerate}
\end{question*}

\begin{proof}{\textbf{(a)}}
\\By definition of $\inf(T)$, we know that $\inf(T) \leq t$ for all $t \in T$. In particular, since $S \subseteq T$, we know that $s \in T$ for all $s \in S$. This means $\inf(T) \leq s$ for all $s \in S$, telling us that $\inf(T)$ is a lower bound of $S$. Furthermore since $\inf(S)$ is the \underline{Greatest} Lower Bound of $S$, we know that $\inf(T) \leq \inf(S)$, the first of these inequalities. 
\\
\\Next, we know that $\inf(S) \leq s$ for all $s \in S$ and that $s \leq \sup(S)$ for all $s \in S$. Thus, fix some $s_0 \in S$ to give us that $\inf(S) \leq s_0$ and $s_0 \leq \sup(S) \implies \inf(S) \leq \sup(S)$ by O3. Therefore, we have our second of the desired inequalities. 
\\
\\Lastly, we know that $t \leq \sup(T)$ for all $t \in T$. Therefore, since $S \subseteq T$ tells us that $s \in T$ for all $s \in S$, we can conclude that $s \leq \sup(T)$ for all $s \in S$. This tells us that $\sup(T)$ is an upper bound of $S$. However, $\sup(S)$ is the \underline{Least} Upper Bound, so we must have that $\sup(S) \leq \sup(T)$, the last of the desired inequalities.
\\
\\We now know that $\inf(T) \leq \inf(S)$, $\inf(S) \leq \sup(S)$, and $\sup(S) \leq \sup(T)$. More compactly, this is equivalent to $\inf(T) \leq \inf(S) \leq \sup(S) \leq \sup(T)$ exactly what we wished to prove. 
\end{proof}

\begin{proof}{\textbf{(b)}}
\\It is clear that $\sup(S) \leq \max\{\sup(S), \sup(T)\}$ and $\sup(T) \leq \max\{\sup(S), \sup(T)\}$. By definition, we know that $s \leq \sup(S)$ for all $s \in S$ and $t \leq \sup(T)$ for all $t \in T$. We can use O3 to deduce that $s \leq \max\{\sup(S), \sup(T)\}$ and $t \leq \max\{\sup(S), \sup(T)\}$ for all $s \in S$ and $t \in T$. This means that for any $r \in S \cup T$, we have that $r \leq \max\{\sup(S), \sup(T)\}$, i.e. $\max\{\sup(S), \sup(T)\}$ is an upper bound for $S \cup T$. Furthermore, we know that $\sup(S \cup T) \leq \max\{\sup(S), \sup(T)\}$ since $\sup(S \cup T)$ is the \underline{Least} Upper Bound of $S \cup T$.
\\
\\On the other hand, since $S \subseteq S \cup T$ and $T \subseteq S \cup T$, then by part (a) of this questions, we know that $\sup(S) \leq \sup(S \cup T)$ and $\sup(T) \leq \sup(S \cup T)$. Also, since $\max\{\sup(S), \sup(T)\}$ equals either $\sup(S)$ or $\sup(T)$, we know that $\max\{\sup(S), \sup(T)\} \leq \sup(S \cup T)$.\\
\\Since we have shown this previous inequality holds in both directions, then by O2, we know that $\sup(S \cup T) = \max\{\sup(S), \sup(T)\}$.
\end{proof}

\section*{4.9}

\begin{question*}{$ $}
\\Let $S$ be a nonempty subset of $\mathbb{R}$ that is bounded below. Define $-S$ to be the set $\{-s: s \in S\}$ and let $s_0 = \sup(-S)$. Prove the following:
\begin{enumerate}[label = (\arabic*)]
\item $-s_0 \leq s$ for all $s \in S$.
\item If $t \leq s$ for all $s \in S$, then $t \leq -s_0$. 
\end{enumerate}

\begin{proof}{\textbf{(1)}}
\\Since $s_0 = \sup(-S)$, then by the definition of Supremum, we know that $\tilde{s} \leq s_0$ for every $\tilde{s} \in -S$. By the definition of $-S$, we know this is equivalent to saying $-s \leq s_0$ for every $s \in S$. Lastly, by Theorem 3.2 (i), we can say that $-s_0 \leq -(-s)$ for every $s \in S$ which is equivalent to $-s_0 \leq s$ for every $s \in S$ since $-(-c) = c$ for all $c \in \mathbb{R}$. Thus, we have proven the desired inequality. 
\end{proof}

\begin{proof}{\textbf{(2)}}
\\Let $t$ be such that $t \leq s$ for all $s \in S$. Then, by Theorem 3.2 (i), we can say that $-s \leq -t$ for all $s \in S$. However, by the definition above of $-S$, this is the same as saying $\tilde{s} \leq -t$ for all $\tilde{s} \in -S$. Therefore, this shows that $-t$ is an upper bound for $-S$. However, since $s_0$ is the \underline{Least} Upper Bound of $-S$, we know that $s_0 \leq -t$. Then, again by Theorem 3.2 (i), we can say that $-(-t) \leq -s_0$ which is the same as saying $t \leq -s_0$, exactly what we wanted to show. 
\end{proof}

\end{question*}

\section*{4.14}

\begin{question*}{$ $}
\\Let $A$ and $B$ be nonempty bounded subsets of $\mathbb{R}$ and define $A + B = \{a + b : a \in A, b \in B\}$.
\begin{enumerate}[label = (\alph*)]
\item Prove that $\sup(A + B) = \sup(A) + \sup(B)$.
\item Prove that $\inf(A + B) = \inf(A) + \inf(B)$.
\end{enumerate}

\begin{proof}{\textbf{(a)}}
\\For any $x \in A + B$, $x$ is of the form $x = a + b$ for $a \in A$ and $b \in B$. Clearly, $a + b \leq \sup(A) + \sup(B)$ since $a \leq \sup(A)$ and $b \leq \sup(B)$. Thus, for any $x \in A + B$, $x \leq \sup(A) + \sup(B)$. In other words, $\sup(A) + \sup(B)$ is an upper bound for $A+B$. However, since $\sup(A+B)$ is the \underline{Least} Upper Bound of $A + B$, we have that $\sup(A+B) \leq \sup(A) + \sup(B)$.
\\
\\To show the inequality in the other direction, fix $b_0$ as some arbitrary element of $B$. Next, note that $x \leq \sup(A + B)$ for all $x \in A + B$, i.e. $a + b \leq \sup(A + B)$ for all $a \in A$ and $b \in B$. In particular, $a + b_0 \leq \sup(A+B)$ for all $a \in A$. Thus, $a + b_0 - b_0 \leq \sup(A+B) - b_0$ for all $a \in A$ by O4. Using A4 and A3, we can finally conclude that $a \leq \sup(A + B) - b_0$ for all $a \in A$. In other words, $\sup(A + B) - b_0$ is an upper bound for $A$. However, since $\sup(A)$ is the \underline{Least} Upper Bound for $A$, we have that $\sup(A) \leq \sup(A+B) - b_0$. From this we can get:
\begin{align*}
\sup(A) &\leq \sup(A+B) - b_0\\
\implies -(\sup(A+B) - b_0) &\leq -\sup(A) &\text{by Theorem 3.2 (i)}\\
\implies -\sup(A+B) + b_0 &\leq -\sup(A) &\text{by DL}\\
\implies -\sup(A+B) + b_0 + \sup(A + B) &\leq -\sup(A) + \sup(A+B) &\text{by O4}\\
\implies b_0 -\sup(A+B) + \sup(A + B) &\leq \sup(A+B) - \sup(A) &\text{by A2}\\
\implies b_0 &\leq \sup(A+B) - \sup(A) &\text{by A2, A4, and A3 in that order}
\end{align*}
Therefore, since this $b_0$ can be any arbitrary element of $B$, we have shown that $\sup(A+B) - \sup(A)$ is an upper bound for $B$. However, since $\sup(B)$ is the \underline{Least} Upper Bound for $B$, we have that $\sup(B) \leq \sup(A + B) - \sup(A)$. Thus, by applying O4 with $\sup(A)$, then applying A4 and A3 on the right and A2 on the left, we get precisely $\sup(A) + \sup(B) \leq \sup(A + B)$
\\
\\Therefore, we have shown that $\sup(A+B) \leq \sup(A) + \sup(B)$ and $\sup(A) + \sup(B) \leq \sup(A+B)$. Therefore, by O2, we have that $\sup(A+B) = \sup(A) + \sup(B)$, exactly what we wanted to prove.
\end{proof}

\begin{proof}{\textbf{(b)}}
\\Similarly to in (a), we know that for any $x = a + b \in A+B$, we have that $\inf(A) + \inf(B) \leq a + b$ since $\inf(A) \leq a$ and $\inf(B) \leq b$. Thus, $\inf(A) + \inf(B) \leq x$ for all $x \in A+B$ meaning that $\inf(A) + \inf(B)$ is a lower bound for $A+B$. However, since $\inf(A+B)$ is the \underline{Greatest} Lower Bound, we know that $\inf(A) + \inf(B) \leq \inf(A+B)$.
\\
\\Motivated from above, we will fix $b_0$ as some arbitrary element of $B$ and notice that $\inf(A+B) \leq x$ for all $x \in A+B$. In particular, $\inf(A+B) \leq a + b_0$ for all $a \in A$. By adding $-b_0$ to both sides according to O4 and then using A4 and A3, we can see that $\inf(A+B) - b_0 \leq a$ for all $a \in A$. In other words, $\inf(A+B) - b_0$ is a lower bound for $A$. However, since $\inf(A)$ is the \underline{Greatest} Lower Bound, we see that $\inf(A+B) - b_0 \leq \inf(A)$. Following this we can see that:
\begin{align*}
\inf(A+B) - b_0 &\leq \inf(A)\\
\implies \inf(A+B) - \inf(A) &\leq b_0 &\text{by using a nearly identical set of steps used in part (a)}
\end{align*}
Therefore, since $b_0$ can be any arbitrary element of $B$, this shows us that $\inf(A+B) - \inf(A)$ is a lower bound for $B$. However, since $\inf(B)$ is the \underline{Greatest} lower bound for $B$, we get that $\inf(A+B) - \inf(A) \leq \inf(B)$. By applying O4, A4, A3, and A2, we immediately get that $\inf(A+B) \leq \inf(A) + \inf(B)$.
\\
\\Therefore, we have shown that $\inf(A) + \inf(B) \leq \inf(A+B)$ and $\inf(A+B) \leq \inf(A) + \inf(B)$. Thus, by O2, we get $\inf(A+B) = \inf(A) = \inf(B)$, exactly what we wanted to prove. 
\end{proof}

\end{question*}

\end{document}