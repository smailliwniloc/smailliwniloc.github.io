\documentclass[10pt,a4paper]{article}
\usepackage[utf8]{inputenc}
\usepackage[english]{babel}
\usepackage{amsmath}
\usepackage{amsfonts}
\usepackage{amssymb}
\usepackage{graphicx}
\usepackage[margin=0.5in]{geometry}
\usepackage{amsthm}
\usepackage{enumitem}
\newtheorem{question}{Question}
\newtheorem*{question*}{Question}
\newtheorem{theorem}{Theorem}
\newtheorem{lemma}{Lemma}

\title{Advanced Calculus Homework 1}
\author{Colin Williams}

\begin{document}
\maketitle
\section*{1.1}
\begin{question*}{$ $}
\\Prove that 
\[P_n: 1^2 + 2^2 + \cdots + n^2 = \frac{n(n+1)(2n+1)}{6}\]
is true for all positive integers $n$.
\end{question*}

\begin{proof}{(Using induction)}
\\\underline{Base Case: $n = 1$,}
\[1^2 = 1 \quad \quad \text{also, } \quad \frac{1(1+1)(2(1)+1)}{6} = \frac{(2)(3)}{6} = 1.\]
\\\underline{Inductive Hypothesis:}
\\Assume that $P_k$ is true for some integer $k \geq 1$. We want to show that $P_{k+1}$ must also be true. i.e., we want to show that:
\[1^2 + 2^2 + \cdots + k^2 + (k+1)^2 = \frac{(k+1)(k+2)(2k+3)}{6}.\]
Starting with the left hand side, we have:
\begin{align*}
1^2 + 2^2 + \cdots + k^2 + (k+1)^2 &= \frac{k(k+1)(2k+1)}{6} + (k+1)^2 \quad \text{by Inductive Hypothesis}\\
&= \frac{k(k+1)(2k+1) + 6(k+1)^2}{6}\\
&= \frac{(k+1)(k(2k+1) + 6(k+1))}{6}\\
&= \frac{(k+1)(2k^2 + k + 6k + 6)}{6}\\
&= \frac{(k+1)(2k^2 + 7k + 6)}{6}\\
&= \frac{(k+1)(k+2)(2k+3)}{6}
\end{align*}
This is our desired equality. Therefore, by the principle of Mathematical Induction, $P_n$ must be true for all $n \in \mathbb{N}$.
\end{proof}

\section*{1.3}
\begin{question*}{$ $}
\\Prove that 
\[P_n: 1^3 + 2^3 + \cdots + n^3 = (1 + 2 + \cdots + n)^2\]
is true for all positive integers $n$.
\end{question*}

\begin{proof}{(Using induction)}
\\\underline{Base Case: $n = 1$,}
\[1^3 = 1 \quad \quad \text{also, } \quad 1^2 = 1.\]
\\\underline{Inductive Hypothesis:}
\\Assume that $P_k$ is true for some integer $k \geq 1$. We want to show that $P_{k+1}$ must also be true. i.e., we want to show that:
\[1^3 + 2^3 + \cdots + k^3 + (k+1)^3 = (1 + 2 + \cdots + k + (k+1))^2\]
Starting with the left hand side, we have:
\begin{align}
1^3 + 2^3 + \cdots + k^3 + (k+1)^3 &= (1 + 2 + \cdots + k)^2 + (k+1)^3 &\quad \text{by Inductive Hypothesis}\nonumber\\
&= \left(\frac{k(k+1)}{2}\right)^2 + (k+1)^3 &\quad \text{by Lemma 1 below}\nonumber\\
&= \frac{k^2(k+1)^2 + 4(k+1)^3}{4}\nonumber\\
&= \frac{(k+1)^2(k^2 + 4(k+1))}{4}\nonumber\\
&= \frac{(k+1)^2(k^2 + 4k + 4)}{4}\nonumber\\
&= \frac{(k+1)^2(k+2)^2}{4}\nonumber\\
&= \left(\frac{(k+1)(k+2)}{2}\right)^2\nonumber\\
&= (1 + 2 + \cdots + k + (k+1))^2 &\quad \text{again by Lemma 1 below}
\end{align}

\begin{lemma}
\[P_n: 1 + 2 + \cdots + n = \frac{n(n+1)}{2} \quad \quad \text{is true for all n} \in \mathbb{N}\]
\end{lemma}
\begin{proof}{Proof of Lemma 1, using induction}
\\\underline{Base Case: $n = 1$,}
\[1 = 1 \quad \quad \text{also, } \quad \frac{1(1+1)}{2} = \frac{2}{2} = 1.\]
\\\underline{Inductive Hypothesis}
\\Assume that $P_k$ is true for some integer $k \geq 1$. We want to show that $P_{k+1}$ must also be true. i.e., we want to show that:
\[1 + 2 + \cdots + k + (k+1) = \frac{(k+1)(k+2)}{2}\]
Starting with the left hand side, we have:
\begin{align*}
1 + 2 + \cdots + k + (k+1) &= \frac{k(k+1)}{2} + k+1 \quad \text{by Inductive Hypothesis}\\
&= \frac{k(k+1) + 2(k+1)}{2}\\
&= \frac{(k+1)(k+2)}{2}
\end{align*}
This is the desired equality; thus, proving the Lemma by the Principle of Mathematical Induction. 
\end{proof}
With the above Lemma proven, and the desired equality in line (1), we have proven the given statement due to the Principle of Mathematical Induction.
\end{proof}

\section*{1.5}
\begin{question*}{$ $}
\\Prove that
\[P_n: 1 + \frac{1}{2} + \frac{1}{4} + \cdots + \frac{1}{2^n} = 2 - \frac{1}{2^n}\]
is true for all positive integers $n$.
\end{question*}

\begin{proof}{(Using indution)}
\\\underline{Base Case: $n = 1$,}
\[1 + \frac{1}{2^1} = \frac{3}{2} \quad \quad \text{also, } \quad 2 - \frac{1}{2^1} = \frac{3}{2}\]
\\\underline{Inductive Hypothesis}
\\Assume that $P_k$ is true for some integer $k \geq 1$. We want to show that $P_{k+1}$ must also be true. i.e., we want to show that:
\[1 + \frac{1}{2} + \cdots + \frac{1}{2^k} + \frac{1}{2^{k+1}} = 2 - \frac{1}{2^{k+1}}\]
Starting with the left hand side, we have:
\begin{align*}
1 + \frac{1}{2} + \cdots + \frac{1}{2^k} + \frac{1}{2^{k+1}} &= 2 - \frac{1}{2^k} + \frac{1}{2^{k+1}} \quad \text{by the Inductive Hypothesis}\\
&= 2 - \frac{1}{2^k}\left(1 - \frac{1}{2}\right)\\
&= 2 - \frac{1}{2^k}\left(\frac{1}{2}\right)\\
&= 2 - \frac{1}{2^{k+1}}
\end{align*}
This is our desired equality. Thus, by the Principle of Mathematical induction, $P_n$ must be true for all positive integers $n$. 
\end{proof}

\section*{2.1}
\begin{question*}{$ $}
\\Show that:
\begin{enumerate}[label=\alph*)]
\item $\sqrt{3}$
\item $\sqrt{5}$
\item $\sqrt{7}$
\item $\sqrt{24}$
\item $\sqrt{31}$
\end{enumerate}
are all not rational numbers.
\end{question*}

For the following proofs, I will use the "Rational Zeros Theorem" as stated below:
\begin{theorem}
Suppose $c_0, c_1, \dots, c_n$ are integers and $r = \frac{c}{d}$ is a rational number (with $d \neq 0$ and $c$ and $d$ sharing no common factors) satisfying the following polynomial equation :
\[c_nx^n + c_{n-1}x^{n-1} + \cdots + c_1x + c_0 = 0\]
where $n \geq 1, c_n \neq 0$. Then, $c$ divides $c_0$ and $d$ divides $c_n$
\end{theorem}

\begin{proof}{$a.)$}
\\I will apply the Rational Zeros Theorem to the polynomial $f$ with $n = 2, c_2 = 1, c_1 = 0$, and $c_0 = -3$ because $\sqrt{3}$ is clearly a zero of $f(x) = x^2 - 3$. By the Rational Zeros Theorem, if this polynomial has any rational roots, they must be of the form $r = \frac{c}{d}$ where $c|(-3)$ and $d|1$. Since 3 is prime, and 1 only has itself and -1 as factors, we know that $r \in \{\pm 1,\pm 3\}$. However, $f(\pm 1) = -2$ and $f(\pm 3) = 6$, so none of those are roots of $f$. Therefore, $f$ has no rational roots and since $\sqrt{3}$ is a root of $f$, it cannot be rational. 
\end{proof}

\begin{proof}{$b.)$}
\\I will apply the Rational Zeros Theorem to the polynomial $f$ with $n = 2, c_2 = 1, c_1 = 0$, and $c_0 = -5$ because $\sqrt{5}$ is clearly a zero of $f(x) = x^2 - 5$. By the Rational Zeros Theorem, if this polynomial has any rational roots, they must be of the form $r = \frac{c}{d}$ where $c|(-5)$ and $d|1$. Since 5 is prime, and 1 only has itself and -1 as factors, we know that $r \in \{\pm 1,\pm 5\}$. However, $f(\pm 1) = -4$ and $f(\pm 5) = 20$, so none of those are roots of $f$. Therefore, $f$ has no rational roots and since $\sqrt{5}$ is a root of $f$, it cannot be rational. 
\end{proof}

\begin{proof}{$c.)$}
\\I will apply the Rational Zeros Theorem to the polynomial $f$ with $n = 2, c_2 = 1, c_1 = 0$, and $c_0 = -7$ because $\sqrt{7}$ is clearly a zero of $f(x) = x^2 - 7$. By the Rational Zeros Theorem, if this polynomial has any rational roots, they must be of the form $r = \frac{c}{d}$ where $c|(-7)$ and $d|1$. Since 7 is prime, and 1 only has itself and -1 as factors, we know that $r \in \{\pm 1,\pm 7\}$. However, $f(\pm 1) = -6$ and $f(\pm 7) = 42$, so none of those are roots of $f$. Therefore, $f$ has no rational roots and since $\sqrt{7}$ is a root of $f$, it cannot be rational. 
\end{proof}

\begin{proof}{$d.)$}
\\I will apply the Rational Zeros Theorem to the polynomial $f$ with $n = 2, c_2 = 1, c_1 = 0$, and $c_0 = -24$ because $\sqrt{24}$ is clearly a zero of $f(x) = x^2 - 24$. By the Rational Zeros Theorem, if this polynomial has any rational roots, they must be of the form $r = \frac{c}{d}$ where $c|(-24)$ and $d|1$. Since 24 has 1, 2, 3, 4, 6, 8, 12, and 24 as its positive factors, and 1 only has itself and -1 as factors, we know that $r \in \{\pm 1,\pm 2, \pm 3, \pm 4, \pm 6, \pm 8, \pm 12, \pm 24\}$. However, $f(\pm 1) = -23$, $f(\pm 2) = -20$, $f(\pm 3) = -15$, $f(\pm 4) = -8$, $f(\pm 6) = 12$, $f(\pm 12) = 120$, and $f(\pm 24) = 552$, so none of those are roots of $f$. Therefore, $f$ has no rational roots and since $\sqrt{24}$ is a root of $f$, it cannot be rational. 
\end{proof}

\begin{proof}{$e.)$}
\\I will apply the Rational Zeros Theorem to the polynomial $f$ with $n = 2, c_2 = 1, c_1 = 0$, and $c_0 = -31$ because $\sqrt{31}$ is clearly a zero of $f(x) = x^2 - 31$. By the Rational Zeros Theorem, if this polynomial has any rational roots, they must be of the form $r = \frac{c}{d}$ where $c|(-31)$ and $d|1$. Since 31 is prime, and 1 only has itself and -1 as factors, we know that $r \in \{\pm 1,\pm 31\}$. However, $f(\pm 1) = -30$ and $f(\pm 31) = 930$, so none of those are roots of $f$. Therefore, $f$ has no rational roots and since $\sqrt{31}$ is a root of $f$, it cannot be rational. 
\end{proof}

\section*{2.2}
\begin{question*}{$ $}
\\Show that:
\begin{enumerate}[label=\alph*)]
\item $\sqrt[3]{2}$
\item $\sqrt[7]{5}$
\item $\sqrt[4]{13}$
\end{enumerate}
are all not rational numbers. 
\end{question*}
For the following proofs, I will once again use the "Rational Zeros Theorem" from above. 

\begin{proof}{$a.)$}
\\I will apply the Rational Zeros Theorem to the polynomial $f$ with $n = 3, c_3 = 1, c_2 = c_1 = 0$, and $c_0 = -2$ because $\sqrt[3]{2}$ is clearly a zero of $f(x) = x^3 - 2$. By the Rational Zeros Theorem, if this polynomial has any rational roots, they must be of the form $r = \frac{c}{d}$ where $c|(-2)$ and $d|1$. Since 2 is prime, and 1 only has itself and -1 as factors, we know that $r \in \{\pm 1,\pm 2\}$. However, $f(\pm 1) = -1$ and $f(\pm 2) = 6$, so none of those are roots of $f$. Therefore, $f$ has no rational roots and since $\sqrt[3]{2}$ is a root of $f$, it cannot be rational. 
\end{proof}

\begin{proof}{$b.)$}
\\I will apply the Rational Zeros Theorem to the polynomial $f$ with $n = 7, c_7 = 1, c_6 = c_5 = c_4 = c_3 = c_2 = c_1 = 0$, and $c_0 = -5$ because $\sqrt[7]{5}$ is clearly a zero of $f(x) = x^7 - 5$. By the Rational Zeros Theorem, if this polynomial has any rational roots, they must be of the form $r = \frac{c}{d}$ where $c|(-5)$ and $d|1$. Since 5 is prime, and 1 only has itself and -1 as factors, we know that $r \in \{\pm 1,\pm 5\}$. However, $f(\pm 1) = -4$ and $f(\pm 5) = 78120$, so none of those are roots of $f$. Therefore, $f$ has no rational roots and since $\sqrt[7]{5}$ is a root of $f$, it cannot be rational. 
\end{proof}

\begin{proof}{$c.)$}
\\I will apply the Rational Zeros Theorem to the polynomial $f$ with $n = 4, c_4 = 1, c_3 = c_2 = c_1 = 0$, and $c_0 = -13$ because $\sqrt[4]{13}$ is clearly a zero of $f(x) = x^4 - 13$. By the Rational Zeros Theorem, if this polynomial has any rational roots, they must be of the form $r = \frac{c}{d}$ where $c|(-13)$ and $d|1$. Since 13 is prime, and 1 only has itself and -1 as factors, we know that $r \in \{\pm 1,\pm 13\}$. However, $f(\pm 1) = -12$ and $f(\pm 13) = 28548$, so none of those are roots of $f$. Therefore, $f$ has no rational roots and since $\sqrt[4]{13}$ is a root of $f$, it cannot be rational. 
\end{proof}

\section*{2.3}
\begin{question*}{$ $}
\\Show that $\sqrt{2 + \sqrt{2}}$ is not a rational number. 
\end{question*}

\begin{proof}{$ $}
\\Here, I would again like to use the "Rational Zeros Theorem," but it is not quite as obvious as to what my polynomial should be. I will begin by setting $a = \sqrt{2 + \sqrt{2}}$ and manipulating the equation.
\begin{align*}
a &= \sqrt{2 + \sqrt{2}}\\
a^2 &= 2 + \sqrt{2}\\
a^2 - 2 &= \sqrt{2}\\
(a^2 - 2)^2 &= 2\\
a^4 - 4a^2 + 4 &= 2\\
a^4 - 4a^2 + 2 &= 0
\end{align*}
From this, it is clear that I can apply the Rational Zeros Theorem to the polynomial $f$ with $n = 4, c_4 = 1, c_3 = 0, c_2 = -4, c_1 = 0$, and $c_0 = 2$ because we've just shown that $\sqrt{2 + \sqrt{2}}$ is a zero of $f(x) = x^4 - 4x^2 + 2$. By the Rational Zeros Theorem, if this polynomial has any rational roots, they must be of the form $r = \frac{c}{d}$ where $c|2$ and $d|1$. Since 2 is prime, and 1 only has itself and -1 as factors, we know that $r \in \{\pm 1,\pm 2\}$. However, $f(\pm 1) = -1$ and $f(\pm 2) = 2$, so none of those are roots of $f$. Therefore, $f$ has no rational roots and since $\sqrt{2 + \sqrt{2}}$ is a root of $f$, it cannot be rational. 
\end{proof}

\section*{2.8}
\begin{question*}{$ $}
\\Find all rational solutions of the equation $x^8 - 4x^5 + 13x^3 - 7x + 1 = 0$.
\end{question*}

\begin{proof}
I will once again apply the Rational Zeros Theorem on this polynomial. In this case, $n = 8, c_8 = 1, c_7 = c_6 = 0, c_5 = -4, c_4 = 0, c_3 = 13, c_2 = 0, c_1 = -7$, and $c_0 = 1$. By the Rational Zeros Theorem, if this polynomial has any rational roots, they must be of the form $r = \frac{c}{d}$ where $c|1$ and $d|1$. Since 1 only has itself and -1 as factors, we know that $r \in \{+1, -1\}$. However, $(1)^8 - 4(1)^5 + 13(1)^3 - 7(1) + 1 = 1 - 4 + 13 - 7 + 1 = 4$, so +1 is not a solution of the equation. Thus, -1 is the only option for a rational root for the polynomial, and, indeed, $(-1)^8 - 4(-1)^5 + 13(-1)^3 - 7(-1) + 1 = 1 + 4 - 13 + 7 + 1 = 0$. Therefore, -1 is the only rational solution to the equation.
\end{proof}
\end{document}