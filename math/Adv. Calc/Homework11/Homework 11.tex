\documentclass[10pt,a4paper]{article}
\usepackage[utf8]{inputenc}
\usepackage[english]{babel}
\usepackage{csquotes}
\usepackage{amsmath}
\usepackage{amsfonts}
\usepackage{amssymb}
\usepackage{graphicx}
\usepackage[margin=0.5in]{geometry}
\usepackage{amsthm}
\usepackage{enumitem}
\usepackage{tikz}
\usetikzlibrary{calc}
\newtheorem{question}{Question}
\newtheorem*{question*}{Question}
\newtheorem{theorem}{Theorem}
\newtheorem*{theorem*}{Theorem}
\newtheorem{lemma}{Lemma}

\theoremstyle{definition}
\newtheorem{answer}{Answer}
\newtheorem*{answer*}{Answer}


\title{Advanced Calc. Homework 11}
\author{Colin Williams}

\begin{document}
\maketitle
\section*{Notation}
For the entirety of this assignment, I will examine power series of the form $\displaystyle \sum a_nx^n$. This series converges for all $|x| < R$ and diverges for all $|x| > R$ with 
\begin{align*}
R = \frac{1}{\beta} && \beta = \begin{cases}
\limsup \sqrt[n]{|a_n|} &\text{or}\\
\lim \left|\frac{a_{n+1}}{a_n}\right| &\text{if it exists}
\end{cases}
\end{align*}
Furthermore, we define $R = 0$ if $\beta = \infty$ and $R = \infty$ if $\beta = 0$.

\section*{23.1}
For each of the following power series, find the radius of convergence and determine the exact interval of convergence.
\begin{enumerate}[label = (\alph*)]
\item $\displaystyle \sum n^2x^n$
	\begin{itemize}
	\item Here, our $a_n = n^2$. Thus,
	\[\beta = \lim \left|\frac{a_{n+1}}{a_n}\right| = \lim \left|\frac{(n+1)^2}{n^2}\right| = \lim \frac{n^2 + 2n + 1}{n^2} = \lim \left(1 + \frac{2}{n} + \frac{1}{n^2}\right) = 1\]
	\item Thus, the \boxed{\text{Radius of Convergence, $R$, is equal to 1}}. To find the exact interval of convergence, I will examine the series at $x = \pm 1$. However, in both cases $\lim(n^2)$ and $\lim((-1)^nn^2)$ do not equal zero, so those corresponding series cannot possibly converge. Thus, the \boxed{\text{Interval of Convergence is } (-1,1)}.
	\end{itemize}
\item $\displaystyle \sum \left(\frac{x}{n}\right)^n$
	\begin{itemize}
	\item Here, our $a_n = \frac{1}{n^n}$. Thus,
	\[\beta = \limsup \sqrt[n]{|a_n|} = \limsup \sqrt[n]{\frac{1}{n^n}} = \limsup \frac{1}{n} = 0\]
	\item Thus, the \boxed{\text{Radius of Convergence, $R$, is equal to $\infty$}}. With $R = \infty$, we are immediately given the interval of convergence is all real numbers. Thus, the \boxed{\text{Interval of Convergence is } (-\infty, \infty)}.
	\end{itemize}
\item $\displaystyle \sum \left(\frac{2^n}{n^2}\right)x^n$
	\begin{itemize}
	\item Here, our $\displaystyle a_n = \left(\frac{2^n}{n^2}\right)$. Thus,
	\[\beta = \lim \left|\frac{a_{n+1}}{a_n}\right| = \lim \left|\frac{2^{n+1}}{(n+1)^2} \cdot \frac{n^2}{2^n}\right| = \lim \frac{2n^2}{(n+1)^2} = 2\lim \frac{n^2}{n^2 + 2n + 1} = 2\lim \frac{1}{1 + \frac{2}{n} + \frac{1}{n^2}} = 2\]
	\item Thus, the \boxed{\text{Radius of Convergence, $R$, is equal to }\frac{1}{2}}. To find the exact interval of convergence, I will examine the series at $x = \pm \frac{1}{2}$. With $x = \frac{1}{2}$, we have the sum of terms in the form of $\displaystyle \left(\frac{2^n}{n^2}\right)\left(\frac{1}{2}\right)^n = \frac{1}{n^2}$ which we know converges by $p$-test. For $x = -\frac{1}{2}$, we have the sum of terms in the form of $\displaystyle \left(\frac{2^n}{n^2}\right)\left(-\frac{1}{2}\right)^n = \frac{(-1)^n}{n^2}$ which we know converges by the alternating series test. Thus, the \boxed{\text{Interval of Convergence is } \left[-\frac{1}{2},\frac{1}{2}\right]}.
	\end{itemize}
\item $\displaystyle \sum \left(\frac{n^3}{3^n}\right)x^n$
	\begin{itemize}
	\item Here, our $\displaystyle a_n = \left(\frac{n^3}{3^n}\right)$. Thus,
	\[\beta = \lim \left|\frac{a_{n+1}}{a_n}\right| = \lim \left|\frac{(n + 1)^3}{3^{n+1}} \cdot \frac{3^n}{n^3}\right| = \lim \left|\frac{(n+1)^3}{3n^3}\right| = \frac{1}{3} \lim \left(1 + \frac{1}{n}\right)^3 = \frac{1}{3}	
	\]
	\item Thus, the \boxed{\text{Radius of Convergence, $R$, is equal to 3}}. To find the exact interval of convergence, I will examine the series at $x = \pm 3$. However, in both cases $\lim(n^3)$ and $\lim((-1)^nn^3)$ do not equal zero, so those corresponding series cannot possibly converge. Thus, the \boxed{\text{Interval of Convergence is } (-3,3)}.
	\end{itemize}
\item $\displaystyle \sum \left(\frac{2^n}{n!}\right)x^n$
	\begin{itemize}
	\item Here, our $\displaystyle a_n = \left(\frac{2^n}{n!}\right)$. Thus,
	\[\beta = \lim \left|\frac{a_{n+1}}{a_n}\right| = \lim \left|\frac{2^{n+1}}{(n+1)!} \cdot \frac{n!}{2^n}\right| = \lim \frac{2}{n+1} = 0\]
	\item Thus, the \boxed{\text{Radius of Convergence, $R$, is equal to }\infty}. With $R = \infty$, we are immediately given the interval of convergence is all real numbers. Thus, the \boxed{\text{Interval of Convergence is } (-\infty, \infty)}.
	\end{itemize}
\item $\displaystyle \sum \left(\frac{1}{(n + 1)^2 \cdot 2^n}\right)x^n$
	\begin{itemize}
	\item Here, our $\displaystyle a_n = \left(\frac{1}{(n+1)^2\cdot 2^n}\right)$. Thus,
	\[\beta = \lim \left|\frac{a_{n+1}}{a_n}\right| = \lim \left|\frac{(n+1)^2 \cdot 2^n}{(n+2)^2 \cdot 2^{n+1}}\right| = \frac{1}{2} \lim \frac{n^2 + 2n + 1}{n^2 + 4n + 4} = \frac{1}{2} \lim \frac{1 + \frac{2}{n} + \frac{1}{n^2}}{1 + \frac{4}{n} + \frac{4}{n^2}} = \frac{1}{2}\]
	\item Thus, the \boxed{\text{Radius of Convergence, $R$, is equal to 2}}. To find the exact interval of convergence, I will examine the series at $x = \pm 2$. With $x = 2$, we get the summation of a series with terms of the form $\displaystyle \left(\frac{1}{(n+1)^2 \cdot 2^n}\right)\cdot 2^n = \frac{1}{(n+1)^2}$ which converges by comparison with $\sum \frac{1}{n^2}$ which converges by the $p$-series test. With $x = -2$, we get the summation of a series with terms of the form $\displaystyle \left(\frac{1}{(n+1)^2 \cdot 2^n}\right)\cdot (-2)^n = \frac{(-1)^n}{(n+1)^2}$ which converges by the Alternating Series test. Thus, the \boxed{\text{Interval of Convergence is } [-2,2]}.
	\end{itemize}
\item $\displaystyle \sum \left(\frac{3^n}{n\cdot 4^n}\right)x^n$
	\begin{itemize}
	\item Here, our $\displaystyle a_n = \left(\frac{3^n}{n\cdot 4^n}\right)$. Thus,
	\[\beta = \lim \left|\frac{a_{n+1}}{a_n}\right| = \lim \left|\frac{3^{n+1}}{(n+1)\cdot 4^{n+1}} \cdot \frac{n \cdot 4^n}{3^n}\right| = \frac{3}{4} \lim \frac{n}{n + 1} = \frac{3}{4} \lim \frac{1}{1 + \frac{1}{n}} = \frac{3}{4}\]
	\item Thus, the \boxed{\text{Radius of Convergence, $R$, is equal to }\frac{4}{3}}. To find the exact interval of convergence, I will examine the series at $x = \pm \frac{4}{3}$. When $x = \frac{4}{3}$, we get a series with terms of the form $\displaystyle \left(\frac{3^n}{n\cdot 4^n}\right)\left(\frac{4}{3}\right)^n = \frac{1}{n}$ which diverges due to the $p$ series test. However, with $x = -\frac{4}{3}$, we get a series with terms of the form $\displaystyle \left(\frac{3^n}{n \cdot 4^n}\right)\cdot \left(-\frac{4}{3}\right) = \frac{(-1)^n}{n}$ which converges due to the Alternating Series Test. Thus, the \boxed{\text{Interval of Convergence is } \left[-\frac{4}{3} ,\frac{4}{3}\right)}.
	\end{itemize}
\item $\displaystyle \sum \left(\frac{(-1)^n}{n^2\cdot 4^n}\right)x^n$
	\begin{itemize}
	\item Here, our $\displaystyle a_n = \left(\frac{(-1)^n}{n^2\cdot 4^n}\right)$. Thus,
	\[\beta = \lim \left|\frac{a_{n+1}}{a_n}\right| = \lim \left|\frac{(-1)^{n+1}}{(n+1)^2 \cdot 4^{n+1}} \cdot \frac{n^2 \cdot 4^n}{(-1)^n}\right| = \frac{1}{4} \lim \frac{n^2}{n^2 + 2n + 1} = \frac{1}{4}\lim \frac{1}{1 + \frac{2}{n} + \frac{1}{n^2}} = \frac{1}{4}\]
	\item Thus, the \boxed{\text{Radius of Convergence, $R$, is equal to 4}}. To find the exact interval of convergence, I will examine the series at $x = \pm 4$. For $x = 4$, we get a series with terms of the form $\displaystyle \left(\frac{(-1)^n}{n^2 \cdot 4^n}\right)\cdot 4^n = \frac{(-1)^n}{n^2}$ which converges by the Alternating Series Test. With $x = -4$, we get a series with terms of the form $\displaystyle \left(\frac{(-1)^n}{n^2 \cdot 4^n}\right)\cdot (-4)^n = \frac{1}{n^2}$ which converges by the $p$-series test. Thus, the \boxed{\text{Interval of Convergence is } [-4,4]}.
	\end{itemize}
\end{enumerate}


\section*{23.2}
For each of the following power series, find the radius of convergence and determine the exact interval of convergence.
\begin{enumerate}[label = (\alph*)]
\item $\displaystyle \sum \sqrt{n}x^n$
	\begin{itemize}
	\item Here, our $\displaystyle a_n = \sqrt{n}$. Thus,
	\[\beta = \lim \left|\frac{a_{n+1}}{a_n}\right| = \lim \left|\frac{\sqrt{n+1}}{\sqrt{n}}\right| = \lim \sqrt{1 + \frac{1}{n}} = 1\]
	\item Thus, the \boxed{\text{Radius of Convergence, $R$, is equal to 1}}. To find the exact interval of convergence, I will examine the series at $x = \pm 1$. For $x = 1$, we get a series with terms of the form $\displaystyle \sqrt{n}$ which diverges since $\lim(\sqrt{n}) \neq 0$. With $x = -1$, we get a series with terms of the form $\displaystyle (-1)^n\sqrt{n}$ which diverges as well since $\lim ((-1)^n\sqrt{n}) \neq 0$. Thus, the \boxed{\text{Interval of Convergence is } (-1,1)}.
	\end{itemize}
\item $\displaystyle \sum \left(\frac{1}{n^{\sqrt{n}}}\right)x^n$
	\begin{itemize}
	\item Here, our $\displaystyle a_n = \left(\frac{1}{n^{\sqrt{n}}}\right)$. Thus,
	\begin{align*}\beta = \limsup \sqrt[n]{|a_n|} = \limsup \sqrt[n]{\frac{1}{n^{\sqrt{n}}}} &= \limsup \frac{1}{n^{\frac{\sqrt{n}}{n}}} = \limsup \frac{1}{n^{1/\sqrt{n}}}\\
	&= \limsup n^{-1/\sqrt{n}}\\
	&= \limsup e^{\displaystyle \ln(n^{-1/\sqrt{n}})}\\
	&= \limsup e^{\displaystyle (-1/\sqrt{n})\cdot \ln(n)}\\
	&= e^{\displaystyle \limsup\left(\frac{-\ln(n)}{\sqrt{n}}\right)}\\
	&= e^{\displaystyle \limsup \left(\frac{-\frac{1}{n}}{\frac{1}{2}n^{-\frac{1}{2}}}\right) } &\text{by L'Hopital's Rule}\\
	&= e^{\displaystyle \limsup \left(\frac{-2\sqrt{n}}{n}\right) }\\
	&= e^{\displaystyle \limsup \left(\frac{-2}{\sqrt{n}}\right) }\\
	&= e^0 = 1
	\end{align*}
	\item Thus, the \boxed{\text{Radius of Convergence, $R$, is equal to 1}}. To find the exact interval of convergence, I will examine the series at $x = \pm 1$. For $x = 1$, we get a series with terms of the form $\displaystyle \left(\frac{1}{n^{\sqrt{n}}}\right)$ which converges by the Comparison Test with $\frac{1}{n^2}$ for all $n \geq 4$. With $x = -1$, we get a series with terms of the form $\displaystyle \left(\frac{(-1)^n}{n^{\sqrt{n}}}\right)$ which converges by the Alternating Series Test. Thus, the \boxed{\text{Interval of Convergence is } [-1,1]}.
	\end{itemize}
\item $\displaystyle \sum x^{n!}$
	\begin{itemize}
	\item Here, our $a_n$ needs to be defined carefully as: \[ a_n = \begin{cases}
	1 &\text{ if } n = k! \text{ for some $k \in \mathbb{N}$}\\
	0 &\text{ otherwise }
	\end{cases}\]
	\item Thus,
	\[\beta = \limsup \sqrt[n]{|a_n|} = \lim_{k \to \infty} \sqrt[k!]{|a_{k!}|} = \lim_{k \to \infty} \sqrt[k!]{1} = 1
	\]
	\item Thus, the \boxed{\text{Radius of Convergence, $R$, is equal to 1}}. To find the exact interval of convergence, I will examine the series at $x = \pm 1$. For $x = 1$, we get a series with terms of the form $\displaystyle 1^{n!} = 1$ which diverges since $\lim(1) \neq 0$. With $x = -1$, we get a series with terms of the form $\displaystyle (-1)^{n!}$ which diverges as well since $\lim ((-1)^{n!}) \neq 0$. Thus, the \boxed{\text{Interval of Convergence is } (-1,1)}.
	\end{itemize}
\item $\displaystyle \sum \left(\frac{3^n}{\sqrt{n}}\right)x^{2n + 1}$
	\begin{itemize}
	\item Again, our $a_n$ needs to be defined carefully as: \[ a_n = \begin{cases}
	\frac{3^k}{\sqrt{k}} &\text{ if } n = 2k + 1 \text{ for some $k \in \mathbb{N}$}\\
	0 &\text{ otherwise }
	\end{cases}\]
	\item Thus,
	\begin{align*}
	\beta = \limsup \sqrt[n]{|a_n|} = \lim_{k \to \infty} \sqrt[2k + 1]{|a_{2k + 1}|} &= \lim_{k \to \infty} \left|\frac{a_{2(k+1) + 1}}{a_{2k+1}}\right|\\
	&= \lim_{k \to \infty} \left|\frac{3^{k+1}}{\sqrt{k+1}} \cdot \frac{\sqrt{k}}{3^k}\right|\\
	&= 3 \lim_{k \to \infty} \sqrt{\frac{k}{k + 1}}\\
	&= 3
	\end{align*}
	\item Thus, the \boxed{\text{Radius of Convergence, $R$, is equal to } \frac{1}{3}}. To find the exact interval of convergence, I will examine the series at $x = \pm \frac{1}{3}$. For $x = \frac{1}{3}$, we get a series with terms of the form $\displaystyle \left(\frac{3^n}{\sqrt{n}}\right)\cdot \left(\frac{1}{3}\right)^{2n + 1} = \frac{1}{\sqrt{n}\cdot 3^{n+1}}$ which converges by a Comparison Test with $\sum \frac{1}{3^n}$ which converges as a geometric series. With $x = -\frac{1}{3}$, we get a series with terms of the form $\displaystyle \left(\frac{3^n}{\sqrt{n}}\right)\cdot \left(-\frac{1}{3}\right)^{2n + 1} = \frac{-1}{\sqrt{n}\cdot 3^{n+1}}$ which converges as well by the same Comparison Test. Thus, the \boxed{\text{Interval of Convergence is } \left[-\frac{1}{3},\frac{1}{3}\right]}.
	\end{itemize}
\end{enumerate}


\end{document}