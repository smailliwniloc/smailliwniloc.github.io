\documentclass[10pt,a4paper]{article}
\usepackage[utf8]{inputenc}
\usepackage[english]{babel}
\usepackage{csquotes}
\usepackage{amsmath}
\usepackage{amsfonts}
\usepackage{amssymb}
\usepackage{graphicx}
\usepackage[margin=0.5in]{geometry}
\usepackage{amsthm}
\usepackage{enumitem}
\usepackage{tikz}
\usetikzlibrary{calc}
\newtheorem{question}{Question}
\newtheorem*{question*}{Question}
\newtheorem{theorem}{Theorem}
\newtheorem*{theorem*}{Theorem}
\newtheorem{lemma}{Lemma}

\theoremstyle{definition}
\newtheorem{answer}{Answer}
\newtheorem*{answer*}{Answer}


\title{Advanced Calc. Homework 4}
\author{Colin Williams}

\begin{document}
\maketitle

\section*{7.1}
Write out the first five terms of the following sequences:
\begin{enumerate}[label = (\alph*)]
\item $\displaystyle s_n = \frac{1}{3n + 1}$
	\begin{itemize}
	\item $\displaystyle \frac{1}{3(1) + 1} = \boxed{\frac{1}{4}}, \quad \frac{1}{3(2) + 1} = \boxed{\frac{1}{7}}, \quad \frac{1}{3(3) + 1} = \boxed{\frac{1}{10}}, \quad \frac{1}{3(4) + 1} = \boxed{\frac{1}{13}}, \quad \frac{1}{3(5) + 1} = \boxed{\frac{1}{16}}$
	\end{itemize}
\item $\displaystyle b_n = \frac{3n + 1}{4n - 1}$
	\begin{itemize}
	\item $\displaystyle \frac{3(1) + 1}{4(1) - 1} = \boxed{\frac{4}{3}}, \quad \frac{3(2) + 1}{4(2) - 1} = \frac{7}{7} = \boxed{1}, \quad \frac{3(3) + 1}{4(3) - 1} = \boxed{\frac{10}{11}}, \quad \frac{3(4) + 1}{4(4) - 1} = \boxed{\frac{13}{15}}, \quad \frac{3(5) + 1}{4(5) - 1} = \boxed{\frac{16}{19}}$
	\end{itemize}
\item $\displaystyle c_n = \frac{n}{3^n}$
	\begin{itemize}
	\item $\displaystyle \frac{1}{3^1} = \boxed{\frac{1}{3}}, \quad \frac{2}{3^2} = \boxed{\frac{2}{9}}, \quad \frac{3}{3^3} = \frac{3}{27} = \boxed{\frac{1}{9}}, \quad \frac{4}{3^4} = \boxed{\frac{4}{81}}, \quad \frac{5}{3^5} = \boxed{\frac{5}{243}}$
	\end{itemize}
\item $\displaystyle \sin\left(\frac{n \pi}{4}\right)$
	\begin{itemize}
	\item $\displaystyle \sin\left(\frac{(1) \pi}{4}\right) = \boxed{\frac{\sqrt{2}}{2}}, \quad \sin\left(\frac{(2) \pi}{4}\right) = \boxed{1}, \quad \sin\left(\frac{(3) \pi}{4}\right) = \boxed{\frac{\sqrt{2}}{2}}, \quad \sin\left(\frac{(4) \pi}{4}\right) = \boxed{0}, \quad \sin\left(\frac{(5) \pi}{4}\right) = \boxed{\frac{-\sqrt{2}}{2}}$
	\end{itemize}
\end{enumerate}


\section*{7.2}
For each of sequences in the last question, determine (without formal proof) whether it converges and, if it converges, give its limit. 
\begin{enumerate}[label = (\alph*)]
\item Since the denominator is growing without bound and the numerator is constant, \boxed{\text{this sequence converges to 0}}
\item Since the sequence can be rewritten in the following way:
\[b_n = \frac{3 + \frac{1}{n}}{4 - \frac{1}{n}}\]
and both of the $\displaystyle \frac{1}{n}$ terms are going to zero, then \boxed{b_n\text{ must converge to }\frac{3}{4}}
\item Since $3^n$ grows much faster than $n$ does, the denominator is becoming much larger than the numerator, so \boxed{\text{this sequence converges to 0}}
\item Since the $\sin(\cdot)$ function is $2\pi$ periodic, we know that $\displaystyle \sin\left(\frac{n \pi}{4}\right)$ must repeat every 8 values of $n$ and these 8 values of $n$ do not give the same result, so \boxed{\text{this sequence does not converge}}
\end{enumerate}

\section*{7.3}
For each sequence below, determine (without formal proof) whether it converges and, if it converges, give its limit.
\begin{enumerate}[label = (\alph*)]
\item $\displaystyle a_n = \frac{n}{n + 1}$
	\begin{itemize}
	\item Note this can be rewritten as $\displaystyle a_n = 1 - \frac{1}{n + 1}$, and the second term goes to 0, so \boxed{a_n \text{ converges to 1}}
	\end{itemize}
\item $\displaystyle b_n = \frac{n^2 + 3}{n^2 - 3}$
	\begin{itemize}
	\item Note this can be rewritten as $\displaystyle b_n = \frac{1 + \frac{3}{n^2}}{1 - \frac{3}{n^2}}$ and both of the $\displaystyle \frac{3}{n^2}$ terms go to zero, so \boxed{b_n \text{ converges to 1}}
	\end{itemize}
\item $\displaystyle c_n = 2^{-n}$
	\begin{itemize}
	\item This is equivalent to $\displaystyle c_n = \frac{1}{2^n}$ and $2^n$ grows without bound, so \boxed{c_n\text{ converges to 0}}
	\end{itemize}
\item $\displaystyle t_n = 1 + \frac{2}{n}$
	\begin{itemize}
	\item $\displaystyle \frac{2}{n}$ goes to zero as $n$ gets large, so \boxed{t_n \text{ converges to 1}}
	\end{itemize}
\item $\displaystyle x_n = 73 + (-1)^n$
	\begin{itemize}
	\item This oscillates between equaling 72 and 74, so \boxed{\text{this sequence does not converge}}
	\end{itemize}
\item $\displaystyle s_n = (2)^{\frac{1}{n}}$
	\begin{itemize}
	\item $\displaystyle \frac{1}{n}$ goes to zero as $n$ gets large and $2^0 = 1$, so \boxed{s_n \text{ converges to 1}}
	\end{itemize}
\item $\displaystyle y_n = n!$
	\begin{itemize}
	\item The factorial function grows without bound, so \boxed{\text{this sequence does not converge}}
	\end{itemize}
\item $\displaystyle d_n = (-1)^nn$
	\begin{itemize}
	\item In absolute value, $|d_n| = n$ and this grows without bound, but any convergent series' absolute value must be bounded, so \boxed{\text{this sequence does not converge}}
	\end{itemize}
\item $\displaystyle \frac{(-1)^n}{n}$
	\begin{itemize}
	\item If $n$ is even, then $\displaystyle \frac{1}{n}$ converges to 0 and if $n$ is odd, then $\displaystyle \frac{-1}{n}$ converges to 0, so \boxed{\text{this sequence converges to 0}}
	\end{itemize}
\item $\displaystyle \frac{7n^3 + 8n}{2n^3 - 3}$
	\begin{itemize}
	\item This fraction can be rewritten as $\displaystyle \frac{7 + \frac{8}{n^2}}{2 - \frac{3}{n^3}}$ and the second terms in the numerator and the denominator both go to zero, so overall \boxed{\text{this sequence converges to }\frac{7}{2}}
	\end{itemize}
\item $\displaystyle \frac{9n^2 - 18}{6n + 18}$
	\begin{itemize}
	\item The numerator grows faster than the denominator, so \boxed{\text{this sequence does not converge}}
	\end{itemize}
\item $\displaystyle \sin\left(\frac{n \pi}{2}\right)$
	\begin{itemize}
	\item Since the $\sin(\cdot)$ function is $2\pi$ periodic, we know that $\displaystyle \sin\left(\frac{n \pi}{2}\right)$ must repeat every 4 values of $n$, but these 4 values of $n$ do not give the same result, so \boxed{\text{this sequence does not converge}}
	\end{itemize}
\item $\displaystyle \sin(n\pi)$
	\begin{itemize}
	\item For every $n \in \mathbb{N}$, $\sin(n\pi) = 0$, so \boxed{\text{this sequence converges to 0}}
	\end{itemize}
\item $\displaystyle \sin\left(\frac{2n\pi}{3}\right)$
	\begin{itemize}
	\item Since the $\sin(\cdot)$ function is $2\pi$ periodic, we know that $\displaystyle \sin\left(\frac{2n \pi}{3}\right)$ must repeat every 3 values of $n$, but these 3 values of $n$ do not give the same result, so \boxed{\text{this sequence does not converge}}
	\end{itemize}
\item $\displaystyle \frac{1}{n}\sin(n)$
	\begin{itemize}
	\item Since $\sin(n)$ is always between $-1$ and $1$, we know that $n$ in the denominator will eventually be much larger since it grows without bound, so \boxed{\text{this sequence converges to 0}}
	\end{itemize}
\item $\displaystyle \frac{2^{n+1} + 5}{2^n - 7}$
	\begin{itemize}
	\item We can rewrite this fraction as $\displaystyle \frac{2 + \frac{5}{2^n}}{1 - \frac{7}{2^n}}$ and both of the second terms in the numerator and the denominator go to zero, so \boxed{\text{this sequence converges to 2}}
	\end{itemize}
\item $\displaystyle \frac{3^n}{n!}$
	\begin{itemize}
	\item Eventually, $n!$ is bigger than $3^n$ and then it continues to grow faster, so \boxed{\text{this sequence converges to 0}}
	\end{itemize}
\item $\displaystyle \left(1 + \frac{1}{n}\right)^2$
	\begin{itemize}
	\item The stuff in parenthesis tend towards 1 since $\displaystyle \frac{1}{n}$ goes to 0. Since $1^2 = 1$, we can say \boxed{\text{this sequence converges to 1}}
	\end{itemize}
\item $\displaystyle \frac{4n^2 + 3}{3n^2 - 2}$
	\begin{itemize}
	\item We can rewrite this fraction as $\displaystyle \frac{4 + \frac{3}{n^2}}{3 - \frac{2}{n^2}}$ and both of the second terms in the numerator and the denominator go to 0, so we can say \boxed{\text{this sequence converges to } \frac{4}{3}}
	\end{itemize}
\item $\displaystyle \frac{6n + 4}{9n^2 + 7}$
	\begin{itemize}
	\item The denominator grows faster than the numerator, so we can say \boxed{\text{this sequence converges to 0}}
	\end{itemize}
\end{enumerate}


\section*{7.4}
Give examples of the following:
\begin{enumerate}[label = (\alph*)]
\item A sequence $(x_n)$ of irrational numbers having a limit $\lim(x_n)$ that is a rational number.
	\begin{itemize}
	\item Let $\displaystyle x_n = \frac{\sqrt{17}}{n}$, then the denominator of $x_n$ is growing without bound and the numerator is constant, so $\lim(x_n) = 0$, a rational number, but all $x_i$'s in the sequence are irrational due to the irrationality of $\sqrt{17}$
	\end{itemize}
\item A sequence $(r_n)$ of rational numbers having a limit $\lim(r_n)$ that is an irrational number.
	\begin{itemize}
	\item Let $r_n$ be a sequence whose $n$th term is the first $n$ decimal places of $\frac{1}{\sqrt{2}}$. i.e. $r_1 = 0.7, r_2 = 0.70, r_3 = 0.707, r_4 = 0.7071, \ldots$ then it is clear that $\lim(r_n) = \frac{1}{\sqrt{2}}$, an irrational number. However, each $r_i$ is a rational number since it is a finite decimal expansion, which must be rational. 
	\end{itemize}
\end{enumerate}


\section*{7.5}
Determine the following limits. No proofs are required but show the relevant algebra.
\begin{enumerate}[label = (\alph*)]
\item $\lim(s_n)$ where $s_n = \sqrt{n^2 + 1} - n$
	\begin{itemize}
	\item First, make the following algebraic manipulations:
	\begin{align*}
	s_n &= \sqrt{n^2 + 1} - n\\
	&= \left(\sqrt{n^2 + 1} - n\right)\cdot \frac{\sqrt{n^2 + 1}+n}{\sqrt{n^2 + 1}+n}\\
	&= \frac{(n^2 + 1) + n\sqrt{n^2 + 1} - n\sqrt{n^2 + 1} - n^2}{\sqrt{n^2 + 1}+n}\\
	&= \frac{n^2 + 1 - n^2}{\sqrt{n^2 + 1}+n}\\
	&= \frac{1}{\sqrt{n^2 + 1}+n}
	\end{align*}
	\item Thus, it is clear that $s_n$ in this form has a denominator that is growing without bound and a constant numerator; thus, \boxed{\lim(s_n) = 0}.
	\end{itemize}
\item $\lim(\sqrt{n^2 + n} - n)$
	\begin{itemize}
	\item I will make a similar algebraic manipulation with this expression as well:
	\begin{align*}
	\sqrt{n^2 + n}-n &= \left(\sqrt{n^2 + n}-n\right) \cdot \frac{\sqrt{n^2 + n}+n}{\sqrt{n^2 + n}+n}\\
	&= \frac{(n^2 + n) + n\sqrt{n^2 + n} - n\sqrt{n^2 + n} - n^2}{\sqrt{n^2 + n}+n}\\
	&= \frac{n^2 + n - n^2}{\sqrt{n^2 + n}+n}\\
	&= \frac{n}{\sqrt{n^2 + n}+n}\\
	&= \frac{n}{\sqrt{n^2 + n}+n} \cdot \frac{\frac{1}{n}}{\frac{1}{n}}\\
	&= \frac{1}{\sqrt{\frac{1}{n^2}(n^2 + n)} + 1}\\
	&= \frac{1}{\sqrt{1 + \frac{1}{n}} + 1}
	\end{align*}
	\item When this expression is written in this manner, we can see that all terms are constant except the $\displaystyle \frac{1}{n}$ term, which goes to zero. Thus, \boxed{\lim(\sqrt{n^2 + n}-n) = \frac{1}{\sqrt{1 + 0} + 1} = \frac{1}{2}}
	\end{itemize}
\item $\lim(\sqrt{4n^2 + n} - 2n)$
	\begin{itemize}
	\item I will once again do the following algebraic manipulations:
	\begin{align*}
	\sqrt{4n^2 + n} - 2n &= \left(\sqrt{4n^2 + n} - 2n\right)\cdot \frac{\sqrt{4n^2 + n} + 2n}{\sqrt{4n^2 + n} + 2n}\\
	&= \frac{(4n^2 + n) + 2n\sqrt{4n^2 + n} - 2n\sqrt{4n^2 + n} - 4n^2}{\sqrt{4n^2 + n} + 2n}\\
	&= \frac{4n^2 + n - 4n^2}{\sqrt{4n^2 + n} + 2n}\\
	&= \frac{n}{\sqrt{4n^2 + n} + 2n}\\
	&= \frac{n}{\sqrt{4n^2 + n} + 2n} \cdot \frac{\frac{1}{n}}{\frac{1}{n}}\\
	&= \frac{1}{\sqrt{\frac{1}{n^2}(4n^2 + n)} + 2}\\
	&= \frac{1}{\sqrt{4 + \frac{1}{n}} + 2}
	\end{align*}
	\item When this expression is written in this manner, we can see that all terms are constant except the $\displaystyle \frac{1}{n}$ term, which goes to zero. Thus, \boxed{\lim(\sqrt{4n^2 + n}-2n) = \frac{1}{\sqrt{4 + 0} + 2} = \frac{1}{4}}
	\end{itemize}
\end{enumerate}

\end{document}