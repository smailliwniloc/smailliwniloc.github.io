\documentclass[10pt,a4paper]{article}
\usepackage[utf8]{inputenc}
\usepackage[english]{babel}
\usepackage{csquotes}
\usepackage{amsmath}
\usepackage{amsfonts}
\usepackage{amssymb}
\usepackage{graphicx}
\usepackage[margin=0.5in]{geometry}
\usepackage{amsthm}
\usepackage{enumitem}
\usepackage{tikz}
\usetikzlibrary{calc}
\newtheorem{question}{Question}
\newtheorem*{question*}{Question}
\newtheorem{theorem}{Theorem}
\newtheorem*{theorem*}{Theorem}
\newtheorem{lemma}{Lemma}

\theoremstyle{definition}
\newtheorem{answer}{Answer}
\newtheorem*{answer*}{Answer}


\title{Advanced Calc. Homework 6}
\author{Colin Williams}

\begin{document}
\maketitle

\section*{9.1}
Using the Limit Theorems 9.2-9.7, prove the following. Justify all steps.
\begin{enumerate}[label = (\alph*)]
\item $\displaystyle \lim \frac{n + 1}{n} = 1$.
\item $\displaystyle \lim \frac{3n + 7}{6n - 5} = \frac{1}{2}$.
\item $\displaystyle \lim \frac{17n^5 + 73n^4 - 18n^2 + 3}{23n^5 + 13n^3} = \frac{17}{23}$.
\end{enumerate}

\begin{proof}{\textbf{(a)}}
\\First, I will note that $\displaystyle \frac{n + 1}{n} = 1 + \frac{1}{n}$. Therefore,
\begin{align*}
\lim \frac{n + 1}{n} &= \lim\left(1 + \frac{1}{n}\right)\\
&= \lim(1) + \lim\left(\frac{1}{n}\right) &\text{by Theorem 9.3}\\
&= 1 + \lim\left(\frac{1}{n}\right) &\text{since 1 converges to 1}\\
&= 1 + 0 &\text{by Theorem 9.7(a) with }p = 1\\
\implies &\boxed{\lim \frac{n + 1}{n} = 1}
\end{align*}
\end{proof}

\begin{proof}{\textbf{(b)}}
\\First, I will note that $\displaystyle \frac{3n + 7}{6n - 5} = \frac{3 + \frac{7}{n}}{6 - \frac{5}{n}}$ by multiplying the numerator and the denominator by $\frac{1}{n}$. Next, I will ensure that the numerator and denominator both converge (and that the denominator does not converge to 0) in order to use Theorem 9.6. With the numerator,
\begin{align*}
\lim\left(3 + \frac{7}{n}\right) &= \lim(3) + \lim\left(\frac{7}{n}\right) &\text{by Theorem 9.3}\\
&= 3 + 7\cdot \lim\left(\frac{1}{n}\right) &\text{by Theorem 9.2}\\
&= 3 + 0&\text{by applying Theorem 9.7(a) with } p = 1.
\end{align*}
Similarly, with the denominator we get,
\begin{align*}
\lim\left(6 - \frac{5}{n}\right) &= \lim(6) + \lim\left(-\frac{5}{n}\right) &\text{by Theorem 9.3}\\
&= 6 -5 \cdot \lim\left(\frac{1}{n}\right) &\text{by Theorem 9.2}\\
&= 6 + 0&\text{by Theorem 9.7(a) with } p = 1.
\end{align*}
Therefore, we can use Theorem 9.6 since both sequences converge, and the denominator does not converge to 0 nor does it equal 0 for any choice of $n$. Thus, 
\begin{align*}
\lim \frac{3n + 7}{6n - 5} &= \lim \frac{3 + \frac{7}{n}}{6 - \frac{5}{n}}\\
&= \frac{\lim\left(3 + \frac{7}{n}\right)}{\lim\left(6 - \frac{5}{n}\right)} &\text{by Theorem 9.6}\\
&= \frac{3}{6} &\text{by my calculations of these limits}\\
\implies &\boxed{\lim \frac{3n + 7}{6n - 5} = \frac{1}{2}}
\end{align*}
\end{proof}

\begin{proof}{\textbf{(c)}}
\\First, I will note that $\displaystyle \frac{17n^5 + 73n^4 - 18n^2 + 3}{23n^5 + 13n^3} = \frac{17 + \frac{73}{n} - \frac{18}{n^3} + \frac{3}{n^5}}{23 + \frac{13}{n^2}}$ by multiplying the numerator and the denominator by $\frac{1}{n^5}$. Next, I will ensure that the numerator and denominator both converge (and that the denominator does not converge to 0) in order to use Theorem 9.6. With the numerator,
\begin{align*}
\lim\left(17 + \frac{73}{n} - \frac{18}{n^3} + \frac{3}{n^5}\right) &= \lim(17) + \lim\left(\frac{73}{n}\right) + \lim\left(\frac{-18}{n^3}\right) + \lim\left(\frac{3}{n^5}\right) &\text{by Theorem 9.3}\\
&= 17 + 73\cdot \lim\left(\frac{1}{n}\right) - 18 \cdot \lim\left(\frac{1}{n^3}\right) + 3 \cdot \lim\left(\frac{1}{n^5}\right) &\text{by Theorem 9.2}\\
&= 17 + 0 + 0 + 0 &\text{by applying Theorem 9.7(a) with } p = 1, 3, 5.
\end{align*}
Similarly, with the denominator we get,
\begin{align*}
\lim\left(23 + \frac{13}{n^2}\right) &= \lim(23) + \lim\left(\frac{13}{n^2}\right) &\text{by Theorem 9.3}\\
&= 23 + 13 \cdot \lim\left(\frac{1}{n^2}\right) &\text{by Theorem 9.2}\\
&= 23 + 0 &\text{by Theorem 9.7(a) with } p = 2.
\end{align*}
Therefore, we can use Theorem 9.6 since both sequences converge, and the denominator does not converge to 0 nor does it equal 0 for any choice of $n$. Thus, 
\begin{align*}
\lim \frac{17n^5 + 73n^4 - 18n^2 + 3}{23n^5 + 13n^3} &= \lim \frac{17 + \frac{73}{n} - \frac{18}{n^3} + \frac{3}{n^5}}{23 + \frac{13}{n^2}}\\
&= \frac{\lim\left(17 + \frac{73}{n} - \frac{18}{n^3} + \frac{3}{n^5}\right)}{\lim\left(23 + \frac{13}{n^2}\right)} &\text{by Theorem 9.6}\\
&= \frac{17}{23} &\text{by my calculations of these limits}\\
\implies &\boxed{\lim \frac{17n^5 + 73n^4 - 18n^2 + 3}{23n^5 + 13n^3} = \frac{17}{23}}
\end{align*}
\end{proof}

\section*{9.2}
Suppose $\lim(x_n) = 3$, $\lim(y_n) = 7$, and $y_n \neq 0$ for all n. Determine the following limits:
\begin{enumerate}[label = (\alph*)]
\item $\displaystyle \lim(x_n + y_n)$.
\item $\displaystyle \lim \frac{3y_n - x_n}{y_n^2}$.
\end{enumerate}

\begin{answer*}{\textbf{(a)}}
\\Using Theorem 9.3, $\lim(x_n + y_n) = \lim(x_n) + \lim(y_n) = 3 + 7$ by the given conditions. Thus, \boxed{\lim(x_n + y_n) = 10}
\end{answer*}

\begin{answer*}{\textbf{(b)}}
\\First, I will ensure that the numerator and denominator both converge and that the denominator does not converge to 0 in order to use Theorem 9.6. For the numerator, we have,
\begin{align*}
\lim(3y_n - x_n) &= \lim(3y_n) + \lim(-x_n) &\text{by Theorem 9.3}\\
&= 3\cdot \lim(y_n) + (-1) \cdot \lim(x_n) &\text{by Theorem 9.2}\\
&= 3 (7) - 3 = 18
\end{align*}
Similarly, for the denominator, we have
\begin{align*}
\lim(y_n^2) &= \lim(y_n \cdot y_n)\\
&= \lim(y_n) \cdot \lim(y_n) &\text{by Theorem 9.4}\\
&= 7 \cdot 7 = 49
\end{align*}
Thus, we can apply since Theorem 9.6 since both sequences converge and the denominator does not converge to 0 nor does it equal 0 for any $n$ (since $y_n \neq 0$ for all n $\implies y_n^2 \neq 0$ for all n). Therefore, we have:
\begin{align*}
\lim \frac{3y_n - x_n}{y_n^2} &= \frac{\lim(3y_n - x_n)}{\lim(y_n^2)} &\text{by Theorem 9.6}\\
&= \frac{18}{49} &\text{by my calculations of these limits}\\
\implies &\boxed{\lim \frac{3y_n - x_n}{y_n^2} = \frac{18}{49}}
\end{align*}
\end{answer*}

\section*{9.3}
Suppose $\lim(a_n) = a$, $\lim(b_n) = b$, and $\displaystyle s_n = \frac{a_n^3 + 4a_n}{b_n^2 + 1}$. Carefully prove that $\displaystyle \lim(s_n) = \frac{a^3 + 4a}{b^2 + 1}$.

\begin{proof}{$ $}
\\First, I will note that since $b_n \in \mathbb{R}$, that $b_n^2 + 1 > 0$. Therefore, the denominator is never equal to 0 for any $n$. From this, I will attempt to use Theorem 9.6; however, I first need to show that the numerator and denominator both converge. I will first, look at the numerator,
\begin{align*}
\lim(a_n^3 + 4a_n) &= \lim(a_n^3) + \lim(4a_n) &\text{by Theorem 9.3}\\
&= \lim(a_n^3) + 4\cdot \lim(a_n) &\text{by Theorem 9.2}\\
&= \lim(a_n\cdot a_n^2) + 4 \cdot \lim(a_n)\\
&= \lim(a_n)\cdot \lim(a_n^2) + 4\cdot \lim(a_n) &\text{by Theorem 9.4}\\
&= \lim(a_n)\cdot \lim(a_n \cdot a_n) + 4\cdot \lim(a_n)\\
&= \lim(a_n) \cdot \lim(a_n) \cdot \lim(a_n) + 4\lim(a_n) &\text{by Theorem 9.4}\\
&= a \cdot a \cdot a + 4\cdot a &\text{by the given definition of }\lim(a_n)\\
&= a^3 + 4a
\end{align*}
Similarly, looking at the denominator we get,
\begin{align*}
\lim(b_n^2 + 1) &= \lim(b_n^2) + \lim(1) &\text{by Theorem 9.3}\\
&= \lim(b_n^2) + 1 &\text{since limit of 1 is 1}\\
&= \lim(b_n \cdot b_n) + 1\\
&= \lim(b_n) \cdot \lim(b_n) + 1 &\text{by Theorem 9.4}\\
&= b \cdot b + 1 &\text{by the given definition of }\lim(b_n)\\
&= b^2 + 1
\end{align*}
Just like we said earlier about each $b_n$, we also have that $b^2 + 1 > 0$, so this limit exists and is nonzero. Therefore, we can use Theorem 9.6 to get:
\begin{align*}
\lim(s_n) &= \lim \frac{a_n^3 + 4a_n}{b_n^2 + 1}\\
&= \frac{\lim(a_n^3 + 4a_n)}{\lim(b_n^2 + 1)} &\text{by Theorem 9.6}\\
&= \frac{a^3 + 4a}{b^2 + 1} &\text{by my computations of these limits above}\\
\implies &\boxed{\lim(s_n) = \frac{a^3 + 4a}{b^2 + 1}}
\end{align*}
\end{proof}

\section*{9.4}
Let $s_1 = 1$ and for $n\geq 1$ let $s_{n + 1} = \sqrt{s_n + 1}$.
\begin{enumerate}[label = (\alph*)]
\item List the first four terms of $(s_n)$.
\item It turns out that $(s_n)$ converges. Assume this fact and prove that the limit is $\frac{1}{2}(1 + \sqrt{5})$.
\end{enumerate}

\begin{answer*}{\textbf{(a)}}
\\The first four terms are 
\begin{itemize}
\item $s_1 = 1$
\item $s_2 = \sqrt{s_1 + 1} = \sqrt{1 + 1} = \sqrt{2}$
\item $s_3 = \sqrt{s_2 + 1} = \sqrt{\sqrt{2} + 1}$
\item $s_4 = \sqrt{s_3 + 1} = \sqrt{\sqrt{\sqrt{2} + 1} + 1}$
\end{itemize}
\end{answer*}

\begin{proof}{\textbf{(b)}}
\\Assume that $(s_n)$ converges to $s$. Notice that if $s_n$ converges to $s$, then $s_{n + 1}$ also converges to $s$. You can prove this by noting that the condition of convergence of $s_n$ to $s$ states that for every $\varepsilon > 0$ there exists some $N$ such $|s_n - s| < \varepsilon$ for all $n > N$. Therefore, we also have that $|s_{n + 1} - s| < \varepsilon$ for all $(n + 1) > N$, i.e. for all $n > N - 1$. Thus, $s_{n + 1}$ converges to $s$ with our \enquote{$N$} given by $N - 1$. Using this, I will square both sides of the equation and then take the limit of both sides.
\begin{align*}
s_{n + 1} &= \sqrt{s_n + 1}\\
\implies s_{n + 1}^2 &= s_n + 1 &\text{by squaring both sides}\\
\implies \lim(s_{n + 1}^2) &= \lim(s_n + 1) &\text{by taking the limit of both sides}\\
\implies \lim(s_{n + 1} \cdot s_{n + 1}) &= \lim(s_n) + \lim(1) &\text{by Theorem 9.3 on the right}\\
\implies \lim(s_{n + 1})\cdot \lim(s_{n + 1}) &= \lim(s_n) + 1 &\text{by Theorem 9.4 on the left}\\
\implies s \cdot s &= s + 1 &\text{by the discussion of the limits above}\\
\implies s^2 - s - 1 &= 0\\
\implies s &= \frac{1 \pm \sqrt{1 - 4(1)(-1)}}{2(1)} &\text{by the quadratic formula}\\
\implies s = \frac{1 + \sqrt{5}}{2} \quad &\text{OR} \quad s = \frac{1 - \sqrt{5}}{2}
\end{align*}
However, $\displaystyle \frac{1 - \sqrt{5}}{2} < 0$ and since $s_{n + 1}$ is generated from the square root of another function (and square roots are nonnegative), $s_{n + 1} \geq 0$ for all $n$ which implies that $s$ must also be greater than or equal to 0 due to Exercise 8.9(a). Thus, \boxed{\lim(s_n) = \frac{1 + \sqrt{5}}{2}}
\end{proof}

\section*{9.5}
Let $t_1 = 1$ and $\displaystyle t_{n + 1} = \frac{t_n^2 + 2}{2t_n}$ for $n \geq 1$. Assume $(t_n)$ converges and find the limit. 

\begin{answer*}{$ $}
\\Assume that $(t_n)$ converges to $t$. Thus, by the same argument used in Exercise 9.4(b), $t_{n + 1}$ also converges to $t$. I will eventually take the limit of both sides of the expression for $t_{n + 1}$. however, I will need to use Theorem 9.6, so I must first check that both the numerator and the denominator converge. First, the numerator:
\begin{align*}
\lim(t_n^2 + 2) &= \lim(t_n^2) + \lim(2) &\text{by Theorem 9.3}\\
&= \lim(t_n \cdot t_n) + 2\\
&= \lim(t_n) \cdot \lim(t_n) + 2 &\text{by Theorem 9.4}\\
&= t \cdot t + 2 &\text{by the convergence of }t_n\\
&= t^2 + 2
\end{align*}
Thus, the numerator converges to $t^2 + 2$. When looking at the denominator we get:
\begin{align*}
\lim(2t_n) &= 2\lim(t_n) &\text{by Theorem 9.2}\\
&= 2t &\text{by the convergence of }t_n
\end{align*}
Thus, the denominator converges to $2t$. Note that $t_n > 0$ for all $n$. This can be proven with induction since $t_1 > 0$ and if we assume that $t_k > 0$ for some $k \in \mathbb{N}$, then $\displaystyle t_{k +1} = \frac{t_k^2 + 2}{2t_k} > 0$ since the numerator and denominator are both positive. Thus, the denominator, $2t_n \neq 0$ for all n which implies that the limit of the denominator $2t$ also must be greater than 0 by Exercise 8.9(a). Thus, we can calculate the supposed value of $t$ in the following manner: 
\begin{align*}
t_{n + 1} &= \frac{t_n^2 + 2}{2t_n}\\
\implies \lim(t_{n + 1}) &= \lim \frac{t_n^2 + 2}{2t_n} &\text{by taking the limit of both sides}\\
\implies \lim(t_{n + 1}) &= \frac{\lim(t_n^2 + 2)}{\lim(2t_n)} &\text{by Theorem 9.6}\\
\implies t &= \frac{t^2 + 2}{2t} &\text{by the above discussions on the limit}\\
\implies 2t^2 &= t^2 + 2\\
\implies t^2 &= 2\\
\implies t = \sqrt{2} \quad &\text{OR} \quad t = -\sqrt{2}
\end{align*}
However, I already discussed the fact that $t > 0$; therefore, if $(t_n)$ converges, then it must converge to the value \boxed{\lim(t_n) = \sqrt{2}}
\end{answer*}
\section*{9.6}
Let $x_1 = 1$ and $x_{n+1} = 3x_n^2$ for $n \geq 1$. 
\begin{enumerate}[label = (\alph*)]
\item Show if $a = \lim(x_n)$, then $a = \frac{1}{3}$ or $a = 0$.
\item Does $\lim(x_n)$ exist? Explain.
\item Discuss the apparent contradiction between parts (a) and (b). 
\end{enumerate}

\begin{answer*}{\textbf{(a)}}
\\Assume $\lim(x_n) = a \in \mathbb{R}$. Thus, by the discussion in Exercise 9.4(b), we know that $\lim(x_{n+1}) = a$ as well. Thus, 
\begin{align*}
x_{n + 1} &= 3x_n^2\\
\implies \lim(x_{n+1}) &= \lim(3x_n^2) &\text{by taking the limit of both sides}\\
\implies \lim(x_{n + 1}) &= 3\lim(x_n \cdot x_n) &\text{by Theorem 9.2}\\
\implies \lim(x_{n+1}) &= 3\lim(x_n) \cdot \lim(x_n) &\text{by Theorem 9.4}\\
\implies a &= 3a \cdot a &\text{by the assumption that }\lim(x_n) = a\\
\implies a - 3a^2 &= 0\\
\implies a(1 - 3a) &= 0\\
\implies a = 0 \quad &\text{OR} \quad a = \frac{1}{3}
\end{align*}
Thus, we have shown exactly what we wished to show. 
\end{answer*}

\begin{answer*}{\textbf{(b)}}
\\$\lim(x_n)$ does exist, but it does not converge to a real number, instead, it diverges to $+\infty$.
\begin{proof}{$ $}
\\First, note that $x_n > n$ for all $n \geq 2$. I will show this by induction: it obviously holds for $n = 2$ since $x_2 = 3 > 2$. Next, assume $x_k > k$ for some $k \geq 2 \in \mathbb{N}$. Then $x_{k+1} = 3x_k^2 > 3k^2 > 3k > 2k = k + k > k + 1$ where the first inequality comes from the inductive hypothesis and some of these inequalities only hold for $k > 1$, but that is fine since we assume $k \geq 2$. Thus, $x_n > n$ for all $n \geq 2$. Therefore, we can show $\lim(x_n) = +\infty$ as follows:
\\
\\Let $M > 0$ be some large positive number. Then for $N = M$, we have $x_n > n > M$ which obviously holds for all $n > N$, so $x_n > M$ for all $n > N$ indicating that $\lim(x_n) = +\infty$.
\end{proof}
\end{answer*}

\begin{answer*}{\textbf{(c)}}
\\In part (a) we assumed that the sequence converged to some real number $a$ and continued to use Limit Theorems with this assumption. These limit theorems are only valid when the sequences they're used on converge to real numbers, so since our sequence does not converge, the use of the Limit Theorems was invalid; thus, explaining the contradictory results. 
\end{answer*}

\section*{9.7}
Complete the proof of Theorem 9.7(c), i.e. show that $\lim(s_n) = 0$.\\
\\We need to prove that for $s_n < \sqrt{\frac{2}{n-1}}$ we have $\lim(s_n) = 0$.
\begin{proof}{$ $}
\\We wish to show that for any $\varepsilon > 0$ that there exists some $N$ such that $|s_n - 0| < \varepsilon$ for all $n > N$. First, however, note that it is sufficient to use $\sqrt{\frac{2}{n-1}} < \varepsilon$ since $|s_n - 0| = s_n < \sqrt{\frac{2}{n-1}}$ where the first equality comes from the discussion at the beginning of the proof of the Theorem 9.7(c). If we fix some $\varepsilon$, we can do the following algebraic manipulations to get an idea for what our $N$ should be. 
\begin{align*}
\sqrt{\frac{2}{n-1}} &< \varepsilon\\
\iff \frac{2}{n-1} &< \varepsilon^2\\
\iff \frac{2}{\varepsilon^2} &< n-1\\
\iff 2\varepsilon^{-2} + 1 &< n
\end{align*}
Therefore, if we fix $\varepsilon>0$ and let $N = 2\varepsilon^{-2} + 1$, then for $n > N$ we have 
\begin{align*}
|s_n - 0| = s_n &< \sqrt{\frac{2}{n - 1}}\\
&< \sqrt{\frac{2}{2\varepsilon^{-2} + 1 - 1}}\\
&= \sqrt{\frac{2}{2}\varepsilon^2} = \varepsilon
\end{align*}
just as desired. Therefore, $\lim(s_n) = 0$.
\end{proof}

\section*{9.9}
Suppose there exists $N_0$ such that $s_n \leq t_n$ for all $n > N_0$. 
\begin{enumerate}[label = (\alph*)]
\item Prove that if $\lim(s_n) = +\infty$, then $\lim(t_n) = +\infty$.
\item Prove that if $\lim(t_n) = -\infty$, then $\lim(s_n) = -\infty$. 
\item Prove that if $\lim(s_n)$ and $\lim(t_n)$ exist, then $\lim(s_n) \leq \lim(t_n)$.
\end{enumerate}

\begin{proof}{\textbf{(a)}}
\\If $\lim(s_n) = +\infty$, then for every $M > 0$, we can find some $N \geq N_0$ such that $s_n > M$ for all $n > N$. Further, since $s_n \leq t_n$ for all $n > N_0$, then $M < s_n \leq t_n$ for all $n > N \implies t_n > M$ for all $n > N$. Thus, $\lim(t_n) = +\infty$. 
\end{proof}

\begin{proof}{\textbf{(b)}}
\\If $\lim(t_n) = -\infty$, then for every $M < 0$, we can find some $N \geq N_0$ such that $t_n < M$ for all $n > N$. Further, since $s_n \leq t_n$ for all $n > N_0$, then $s_n \leq t_n < M$ for all $n > N \implies s_n < M$ for all $n > N$. Thus, $\lim(s_n) = -\infty$. 
\end{proof}

\begin{proof}{\textbf{(c)}}
\\Assume that $\lim(s_n) = s$ and $\lim(t_n) = t$ where $s, t \in \mathbb{R}$. We need not consider $s$ or $t$ to be $\pm\infty$ since those cases were taken care of in parts (a) and (b). Next, note that $s_n \leq t_n \implies s_n - t_n \leq 0$ for all $n > N_0$. Thus, taking limits, we get $\lim(s_n - t_n) \leq \lim(0)$. Thus, using Theorem 9.3 and 9.2, we get $\lim(s_n) - \lim(t_n) \leq 0 \implies s \leq t$, just as desired. 
\end{proof}


\section*{9.12}
Assume that $s_n \neq 0$ for all n and that the limit $\displaystyle L = \lim \left|\frac{s_{n + 1}}{s_n}\right|$ exists.
\begin{enumerate}[label = (\alph*)]
\item Show that if $L < 1$, then $\lim(s_n) = 0$.
\item Show that if $L > 1$, then $\lim|s_n| = +\infty$.
\end{enumerate}

\begin{proof}{\textbf{(a)}}
\\Assume that $L < 1$, since we know that $\displaystyle L = \lim \left|\frac{s_{n + 1}}{s_n}\right|$, we can say that for every $\varepsilon > 0$ we can find some $N \in \mathbb{N}$ such that $n \geq N$ implies that 
\begin{align*}
\left|\frac{s_{n + 1}}{s_n} - L\right| &< \varepsilon\\
\implies L - \varepsilon < &\frac{s_{n + 1}}{s_n} < L + \varepsilon &\text{by Exercise 3.7(b)}\\
\implies -L - \varepsilon < &\frac{s_{n + 1}}{s_n} < L + \varepsilon &\text{since we know } -L < L \text{ since }L \geq 0\\
\text{Note that we know } L \geq 0 \text{ since } \left|\frac{s_{n + 1}}{s_n}\right| &\geq 0 \text{ so the result follows from Exercise 8.9(a)}\\
\implies \left|\frac{s_{n + 1}}{s_n}\right| &< L + \varepsilon &\text{by Exercise 3.7(a)}\\
\end{align*}
Since $L < 1$, we wish to choose $\varepsilon$ small enough such that $L + \varepsilon < 1$. One such possible choice for $\varepsilon$ would be $\varepsilon = \frac{1}{2}(1 - L)$. Regardless, define $a := L + \varepsilon$ then it is clear that $a \in (L, 1)$. From this, we can alter the last inequality we had to get for all $n \geq N$:
\[|s_{n + 1}| < a|s_n|\]
In particular, we know that
\[|s_{N + 1}| < a|s_N|\]
Iterating this process we get
\[|s_{N + 2}| < a|s_{N + 1}| < a^2 |s_N|\]
\[\implies |s_n| = |s_{N + (n - N)}| < a|s_{N + (n - N) - 1}| < a^2|s_{N + (n - N) - 2}| < \ldots < a^{n - N}|s_N| \]
\[\implies |s_n| < a^{n - N}|s_N| = a^n a^{-N}|s_N| \quad \text{for all } n > N\]
Thus, by noting that $a^{-N}|s_N|$ is simply just a constant and that $0 < a < 1$, we can use Theorem 9.7(b) to say that the right hand side of this inequality converges to 0. In other words, for $\varepsilon > 0$, there exists some $n > N_1$ such that 
\[|a^n - 0| = a^n < \frac{\varepsilon a^N}{|s_N|}\]
Using this, we get
\[|s_n - 0| = |s_n| < a^n a^{-N}|s_N| < \frac{\varepsilon a^N}{|s_N|} a^{-N}|s_N| = \varepsilon\]
Thus, $|s_n - 0| < \varepsilon$ for all $n > N_1$, so $\lim(s_n) = 0$, just as desired. 
\end{proof}

\begin{proof}{\textbf{(b)}}
\\Let $L > 1$ and $\displaystyle t_n = \frac{1}{|s_n|}$, then we have that 
\begin{align*}
\left|\frac{t_{n + 1}}{t_n}\right| = \left|\frac{\frac{1}{|s_{n+1}|}}{\frac{1}{|s_n|}}\right| = \left|\frac{|s_n|}{|s_{n + 1}|}\right| = \left|\frac{s_n}{s_{n+1}}\right|
\end{align*}
Thus, by Lemma 9.5 and the fact that $L > 1$ (more specifically $L \neq 0$) and $s_n \neq 0$ for all $n$. we get that 
\begin{align*}
\lim \left|\frac{t_{n+1}}{t_n}\right| = \frac{1}{\left|\frac{s_{n + 1}}{s_n}\right|} = \frac{1}{L}
\end{align*}
Furthermore, if $L > 1$, then $\frac{1}{L} < 1$. From this, we can use the result from part (a) of this question to conclude that $\lim(t_n) = 0$. However, by Theorem 9.10, this shows that $\lim|s_n| = +\infty$, exactly what we wished to show. 
\end{proof}

\section*{9.15}
Show that $\displaystyle \lim_{n \to \infty} \frac{a^n}{n!} = 0$ for all $a \in \mathbb{R}$

\begin{proof}{$ $}
\\Say $\displaystyle s_n = \frac{a^n}{n!}$, then I will consider the limit used in Exercise 9.12:
\begin{align*}
\lim \left|\frac{s_{n + 1}}{s_n}\right| &= \lim \left|\frac{\frac{a^{n + 1}}{(n + 1)!}}{\frac{a^n}{n!}}\right|\\
&= \lim \left|\frac{a^{n + 1}}{(n + 1)!} \cdot \frac{n!}{a^n}\right|\\
&= \lim \left|\frac{a}{n + 1}\right|\\
&= \lim \frac{|a|}{n + 1}\\
&= \lim \frac{\frac{|a|}{n}}{1 + \frac{1}{n}}
\end{align*}
I will now consider the limits of the numerator and the denominator separately in order to use Theorem 9.6. Starting with the numerator, $\displaystyle \lim\left(\frac{|a|}{n}\right) = |a|\cdot \lim\left(\frac{1}{n}\right)$ by Theorem 9.2 which in turn equals 0 by Theorem 9.7(a) with $p = 1$. Now for the denominator, $\lim\left(1 + \frac{1}{n}\right) = \lim(1) + \lim\left(\frac{1}{n}\right)$ by Theorem 9.3 which in turn equals 1 by Theorem 9.7(a). Notice that not only is the limit of the denominator equal to zero, but the the denominator is never equal to zero for any choice of $n$ so the conditions for Theorem 9.6 are met. Thus, 
\begin{align*}
\lim\left|\frac{s_{n+1}}{s_n}\right| &= \lim \frac{\frac{|a|}{n}}{1 + \frac{1}{n}}\\
&= \frac{\lim\left(\frac{|a|}{n}\right)}{\lim\left(1 + \frac{1}{n}\right)} &\text{by Theorem 9.6}\\
&= \frac{0}{1} &\text{by my above calculations}\\
&= 0
\end{align*}
Thus, I have shown that $\displaystyle \lim\left|\frac{s_{n+1}}{s_n}\right| = 0 < 1$ so I can use the result from 9.12(a) to conclude that $\lim(s_n) = 0$ where $s_n$ is our sequence of interest. 
\end{proof}

\end{document}