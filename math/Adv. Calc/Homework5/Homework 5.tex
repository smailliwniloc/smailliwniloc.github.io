\documentclass[10pt,a4paper]{article}
\usepackage[utf8]{inputenc}
\usepackage[english]{babel}
\usepackage{csquotes}
\usepackage{amsmath}
\usepackage{amsfonts}
\usepackage{amssymb}
\usepackage{graphicx}
\usepackage[margin=0.5in]{geometry}
\usepackage{amsthm}
\usepackage{enumitem}
\usepackage{tikz}
\usetikzlibrary{calc}
\newtheorem{question}{Question}
\newtheorem*{question*}{Question}
\newtheorem{theorem}{Theorem}
\newtheorem*{theorem*}{Theorem}
\newtheorem{lemma}{Lemma}

\theoremstyle{definition}
\newtheorem{answer}{Answer}
\newtheorem*{answer*}{Answer}


\title{Advanced Calc. Homework 5}
\author{Colin Williams}

\begin{document}
\maketitle

\section*{Introduction}
Note, I will use the following definition for the convergence (or limit) of a sequence $(s_n)$ to some real number $s$ if:
\begin{equation}
\forall \; \varepsilon > 0, \; \exists \text{ a number } N \text{ such that } n > N \text{ implies } |s_n - s| < \varepsilon,
\end{equation}
then the limit exists and is equal to $s$.

\section*{8.1}
Prove the following:
\begin{enumerate}[label = (\alph*)]
\item $\lim \frac{\displaystyle (-1)^n}{\displaystyle n} = 0$
\item $\lim \frac{\displaystyle 1}{\displaystyle n^{\frac{1}{3}}} = 0$
\item $\lim \frac{2n - 1}{3n + 2} = \frac{2}{3}$
\item $\lim \frac{n + 6}{n^2 - 6} = 0$
\end{enumerate}
$ $
\begin{proof}{\textbf{(a)}}
\\Using our definition in (1), let $\varepsilon > 0$ be some given number and $\displaystyle (s_n) = \frac{(-1)^n}{n}$, then we wish to show $s = 0$. I will first examine $|s_n - s|$:
\begin{align*}
|s_n - s| &= \left|\frac{(-1)^n}{n} - 0\right|\\
&= \left|\frac{(-1)^n}{n}\right|\\
&= \frac{|(-1)^n|}{|n|}\\
&= \frac{1}{n}
\end{align*}
Thus, if we want $|s_n - s| < \varepsilon$, we can make the following algebraic manipulations:
\begin{align*}
|s_n - s| &< \varepsilon\\
\iff \frac{1}{n} &< \varepsilon &\text{by the above examination}\\
\iff 1 &< n\cdot \varepsilon\\
\iff \frac{1}{\varepsilon} &< n
\end{align*}
Therefore, if we choose $\displaystyle N = \frac{1}{\varepsilon}$, then it is clear by the above calculations that having $n > N$ implies that $|s_n - s| < \varepsilon$, proving that the limit does, indeed, equal $0$.
\end{proof}

\begin{proof}{\textbf{(b)}}
\\Using our definition in (1), let $\varepsilon > 0$ be some given number and $\displaystyle (s_n) = \frac{1}{n^{\frac{1}{3}}}$, then we wish to show $s = 0$. I will first examine $|s_n - s|$:
\begin{align*}
|s_n - s| &= \left|\frac{1}{n^{\frac{1}{3}}} - 0\right|\\
&= \left|\frac{1}{n^{\frac{1}{3}}}\right|\\
&= \frac{|1|}{|n^{\frac{1}{3}}|}\\
&= \frac{1}{n^{\frac{1}{3}}}
\end{align*}
Thus, if we want $|s_n - s| < \varepsilon$, we can make the following algebraic manipulations:
\begin{align*}
|s_n - s| &< \varepsilon\\
\iff \frac{1}{n^{\frac{1}{3}}} &< \varepsilon &\text{by the above examination}\\
\iff \frac{1}{n} &< \varepsilon^3\\
\iff 1 &< n\cdot \varepsilon^3\\
\iff \frac{1}{\varepsilon^3} &< n
\end{align*}
Therefore, if we choose $\displaystyle N = \frac{1}{\varepsilon^3}$, then it is clear by the above calculations that having $n > N$ implies that $|s_n - s| < \varepsilon$, proving that the limit does, indeed, equal $0$.
\end{proof}

\begin{proof}{\textbf{(c)}}
\\Using our definition in (1), let $\varepsilon > 0$ be some given number and $\displaystyle (s_n) = \frac{2n - 1}{3n + 2}$, then we wish to show $\displaystyle s = \frac{2}{3}$. I will first examine $|s_n - s|$:
\begin{align*}
|s_n - s| &= \left|\frac{2n - 1}{3n + 2} - \frac{2}{3}\right|\\
&= \left| \frac{3(2n - 1)}{3(3n + 2)} - \frac{2(3n + 2)}{3(3n + 2)} \right|\\
&= \left|\frac{(6n - 3) - (6n + 4)}{9n + 6}\right|\\
&= \left| \frac{-7}{9n + 6} \right|\\
&= \frac{|-7|}{|9n + 6|}\\
&= \frac{7}{9n + 6}
\end{align*}
Thus, if we want $|s_n - s| < \varepsilon$, we can make the following algebraic manipulations:
\begin{align*}
|s_n - s| &< \varepsilon\\
\iff \frac{7}{9n + 6} &< \varepsilon &\text{by the above examination}\\
\iff 7 &< \varepsilon(9n + 6)\\
\iff 7 &< 9n\varepsilon + 6\varepsilon\\
\iff 7 - 6\varepsilon &< 9n\varepsilon\\
\iff \frac{7 - 6\varepsilon}{9\varepsilon} &< n
\end{align*}
Therefore, if we choose $\displaystyle N = \frac{7 - 6\varepsilon}{9\varepsilon}$, then it is clear by the above calculations that having $n > N$ implies that $|s_n - s| < \varepsilon$, proving that the limit does, indeed, equal $\displaystyle \frac{2}{3}$.
\end{proof}

\begin{proof}{\textbf{(d)}}
\\Using our definition in (1), let $\varepsilon > 0$ be some given number and $\displaystyle (s_n) = \frac{n + 6}{n^2 - 6}$, then we wish to show $s = 0$. I will first examine $|s_n - s|$:
\begin{align*}
|s_n - s| &= \left|\frac{n + 6}{n^2 - 6} - 0\right|\\
&= \left| \frac{n + 6}{n^2 - 6} \right|\\
&= \frac{|n + 6|}{|n^2 - 6|}\\
&= \frac{n + 6}{n ^2 - 6} &\text{since we can drop } |\cdot| \text{ as long as } n \geq 3\\
&\leq \frac{7n}{n^2 - 6} &\text{since } n + 6 \leq 7n \; \forall \; n \in \mathbb{N}\\
&\leq \frac{7n}{\frac{1}{3}n^2} &\text{since } n^2 - 6 \geq \frac{1}{3}n^2 \; \forall \; n \geq 3 \in \mathbb{N}\\
&= \frac{21}{n}
\end{align*}
Thus, if we want $|s_n - s| < \varepsilon$,  and since we have just shown that $|s_n - s| \leq \frac{21}{n}$, then, if we can show that under some conditions, we have $\frac{21}{n} < \varepsilon$, then we could conclude that $|s_n - s| < \varepsilon$ by the transitivity of \enquote{$<$} under those same conditions. We can do this by making the following algebraic manipulations:
\begin{align*}
|s_n - s| \leq \frac{21}{n} &< \varepsilon\\
\iff 21 &< n\cdot \varepsilon\\
\iff \frac{21}{\varepsilon} &< n
\end{align*}
Since the above calculations were only valid under the condition that $n \geq 3$, then we know that $N$ must be at least 3 as well. Therefore, if we choose $\displaystyle N = \max\left\{3, \frac{21}{\varepsilon}\right\}$, then it is clear by the above calculations that having $n > N$ implies that $\displaystyle \frac{21}{n} < \varepsilon$, which in turn shows that $|s_n - s| < \varepsilon$ since $\displaystyle |s_n - s| \leq \frac{21}{n}$ proving that the limit does, indeed, equal $\displaystyle 0$.
\end{proof}

\section*{8.2}
Determine the limits of the following sequences, and then prove your claims.
\begin{enumerate}[label = (\alph*)]
\item $\displaystyle a_n = \frac{n}{n^2 + 1}$
\item $\displaystyle b_n = \frac{7n - 19}{3n + 7}$
\item $\displaystyle c_n = \frac{4n + 3}{7n - 5}$
\item $\displaystyle d_n = \frac{2n + 4}{5n + 2}$
\item $\displaystyle s_n = \frac{1}{n}\sin(n)$
\end{enumerate}

\begin{answer*}{\textbf{(a)}}
\\I claim that \boxed{\lim a_n = 0}
\begin{proof}{$ $}
\\First, given some $\varepsilon > 0$, I will examine $|a_n - 0|$:
\begin{align*}
|a_n - 0| &= \left|\frac{n}{n^2 + 1} - 0\right|\\
&= \frac{|n|}{|n^2 + 1|}\\
&= \frac{n}{n^2 + 1}\\
&\leq \frac{n}{n^2} &\text{since } n^2 + 1 > n^2\\
&= \frac{1}{n}
\end{align*}
Thus, since $\displaystyle |a_n - 0| \leq \frac{1}{n}$, then finding a condition where $\displaystyle \frac{1}{n} < \varepsilon$ would be a sufficient condition to show that $|a_n - 0| < \varepsilon$, we will do this by the following algebraic manipulations:
\begin{align*}
\frac{1}{n} &< \varepsilon\\
\iff \frac{1}{\varepsilon} &< n
\end{align*}
Thus, choosing $\displaystyle N = \frac{1}{\varepsilon}$ guarantees that with $n > N$, then $\displaystyle \frac{1}{n} < \varepsilon$ which in turn guarantees that $|a_n - 0| < \varepsilon$, proving that the limit does, indeed, equal 0.
\end{proof}
\end{answer*}

\begin{answer*}{\textbf{(b)}}
\\I claim that \boxed{\lim b_n = \frac{7}{3}}
\begin{proof}{$ $}
\\First, given some $\varepsilon > 0$, I will examine $\displaystyle \left|b_n - \frac{7}{3}\right|$:
\begin{align*}
\left|b_n - \frac{7}{3}\right| &= \left|\frac{7n - 19}{3n + 7} - \frac{7}{3}\right|\\
&= \left|\frac{3(7n - 19)}{3(3n + 7)} - \frac{7(3n + 7)}{3(3n + 7)}\right|\\
&= \left|\frac{-106}{9n + 21}\right|\\
&= \frac{|-106|}{|9n + 21|}\\
&= \frac{106}{9n + 21}
\end{align*}
Thus, to find a condition in which $\displaystyle \left|b_n - \frac{7}{3}\right| < \varepsilon$, we can do the following algebraic manipulations:
\begin{align*}
\left|b_n - \frac{7}{3}\right| &< \varepsilon\\
\iff \frac{106}{9n + 21} &< \varepsilon &\text{by the above calculations}\\
\iff 106 &< 9n\varepsilon + 21\varepsilon\\
\iff 106 - 21\varepsilon &< 9n\varepsilon\\
\iff \frac{106 - 21\varepsilon}{9\varepsilon} &< n
\end{align*}
Thus, choosing $\displaystyle N = \frac{106 - 21\varepsilon}{9\varepsilon}$ guarantees that with $n > N$, then $\displaystyle \left|b_n - \frac{7}{3}\right| < \varepsilon$, proving that the limit does, indeed, equal $\displaystyle \frac{7}{3}$.
\end{proof}
\end{answer*}

\begin{answer*}{\textbf{(c)}}
\\I claim that \boxed{\lim c_n = \frac{4}{7}}
\begin{proof}{$ $}
\\First, given some $\varepsilon > 0$, I will examine $\displaystyle \left|c_n - \frac{4}{7}\right|$:
\begin{align*}
\left|c_n - \frac{4}{7}\right| &= \left|\frac{4n + 3}{7n - 5} - \frac{4}{7}\right|\\
&= \left|\frac{7(4n + 3)}{7(7n - 5)} - \frac{4(7n - 5)}{7(7n - 5)}\right|\\
&= \left|\frac{41}{49n - 35}\right|\\
&= \frac{|41|}{|49n - 35|}\\
&= \frac{41}{49n - 35} &\text{since } 49n - 35 > 0 \; \forall \; n \in \mathbb{N}
\end{align*}
Thus, to find a condition in which $\displaystyle \left|c_n - \frac{4}{7}\right| < \varepsilon$, we can do the following algebraic manipulations:
\begin{align*}
\left|c_n - \frac{4}{7}\right| &< \varepsilon\\
\iff \frac{41}{49n - 35} &< \varepsilon &\text{by the above calculations}\\
\iff 41 &< 49n\varepsilon - 35\varepsilon\\
\iff 41 + 35\varepsilon &< 49n\varepsilon\\
\iff \frac{41 + 35\varepsilon}{49\varepsilon} &< n
\end{align*}
Thus, choosing $\displaystyle N = \frac{41 + 35\varepsilon}{49\varepsilon}$ guarantees that with $n > N$, then $\displaystyle \left|c_n - \frac{4}{7}\right| < \varepsilon$, proving that the limit does, indeed, equal $\displaystyle \frac{4}{7}$.
\end{proof}
\end{answer*}

\begin{answer*}{\textbf{(d)}}
\\I claim that \boxed{\lim d_n = \frac{2}{5}}
\begin{proof}{$ $}
\\First, given some $\varepsilon > 0$, I will examine $\displaystyle \left|d_n - \frac{2}{5}\right|$:
\begin{align*}
\left|d_n - \frac{2}{5}\right| &= \left|\frac{2n + 4}{5n + 2} - \frac{2}{5}\right|\\
&= \left|\frac{5(2n + 4)}{5(5n + 2)} - \frac{2(5n + 2)}{5(5n + 2)}\right|\\
&= \left|\frac{16}{25n + 10}\right|\\
&= \frac{|16|}{|25n + 10|}\\
&= \frac{16}{25n + 10}
\end{align*}
Thus, to find a condition in which $\displaystyle \left|d_n - \frac{2}{5}\right| < \varepsilon$, we can do the following algebraic manipulations:
\begin{align*}
\left|d_n - \frac{2}{5}\right| &< \varepsilon\\
\iff \frac{16}{25n + 10} &< \varepsilon &\text{by the above calculations}\\
\iff 16 &< 25n\varepsilon + 10\varepsilon\\
\iff 16 - 10\varepsilon &< 25n\varepsilon\\
\iff \frac{16 - 10\varepsilon}{25\varepsilon} &< n
\end{align*}
Thus, choosing $\displaystyle N = \frac{16 - 10\varepsilon}{25\varepsilon}$ guarantees that with $n > N$, then $\displaystyle \left|d_n - \frac{2}{5}\right| < \varepsilon$, proving that the limit does, indeed, equal $\displaystyle \frac{2}{5}$.
\end{proof}
\end{answer*}

\begin{answer*}{\textbf{(e)}}
\\I claim that \boxed{\lim s_n = 0}
\begin{proof}{$ $}
\\First, given some $\varepsilon > 0$, I will examine $\displaystyle |s_n - 0|$:
\begin{align*}
|s_n - 0| &= \left|\frac{1}{n}\sin(n) - 0\right|\\
&= \left|\frac{1}{n}\right| \cdot |\sin(n)|\\
&\leq \frac{1}{n} &\text{since } -1 \leq \sin(n) \leq 1 \; \forall \; n \in \mathbb{N}\\
\end{align*}
Thus, to find a condition in which $|s_n - 0| < \varepsilon$, it suffices to find a condition in which $\displaystyle \frac{1}{n} < \varepsilon$ since $\displaystyle |s_n - 0| \leq \frac{1}{n}$. We have seen several times that this is equivalent to the condition that $\displaystyle \frac{1}{\varepsilon} < n$.\\
\\
Thus, choosing $\displaystyle N = \frac{1}{\varepsilon}$ guarantees that with $n > N$, then $\displaystyle |s_n - 0| \leq \frac{1}{n} < \varepsilon$, proving that the limit does, indeed, equal $0$.
\end{proof}
\end{answer*}

\section*{8.3}
Let $(s_n)$ be a sequence of nonnegative real numbers, and suppose $\lim s_n = 0$. Prove $\lim \sqrt{s_n} = 0$. This will complete the proof for Example 5.

\begin{proof}{$ $}
\\We are given that for every $\varepsilon_1 > 0$, there exists some $N_1 \in \mathbb{R}$ such that $|s_n - 0| < \varepsilon_1$ for every $n > N_1$. From this, we wish to show that for every $\varepsilon_2 > 0$, there exists some $N_2 \in \mathbb{R}$ such that  $|\sqrt{s_n} - 0| < \varepsilon_2$ for every $n > N_2$. Let us choose $\varepsilon_1 = \varepsilon_2^2$. Therefore we know the following holds for $n$ greater than or equal to some $N_1$:
\begin{align*}
|s_n - 0| &< \varepsilon_1 &\text{by assumption}\\
\implies |s_n| &< \varepsilon_1\\
\implies s_n &< \varepsilon_1 &\text{since the sequence is nonnegative}\\
\implies s_n &< \varepsilon_2^2 &\text{by our assignment of } \varepsilon_1\\
\implies \sqrt{s_n} &< \varepsilon_2 \\
\implies |\sqrt{s_n} - 0| &< \varepsilon_2 &\text{since square roots are positive and we can freely subtract 0}
\end{align*}
Thus, we know that $N_2$ exists where the above condition is satisfied. In particular, with $N_1$ given by the choice of $\varepsilon_1 = \varepsilon_2^2$, $N_2 = N_1$, so it obviously exists; thus, $\lim \sqrt{s_n} = 0$.
\end{proof}

\section*{8.4}
Let $(t_n)$ be a bounded sequence, i.e., there exists $M$ such that $|t_n| \leq M$ for all $n$, and let $(s_n)$ be a sequence such that $\lim s_n = 0$. Prove that $\lim(s_n t_n) = 0$.

\begin{proof}{$ $}
\\I will start by noting that for any $\varepsilon_1 > 0$, we know that there exists some $N_1$ such that $|s_n - 0| < \varepsilon$ for all $n > N_1$. In particular, if we are given some $\varepsilon_2 > 0$, then we know that $N_1$ exists for $\displaystyle \varepsilon_1 = \frac{\varepsilon_2}{M}$ From this I will examine $|s_nt_n - 0|$ in the following manner with some $\varepsilon_2$ given:
\begin{align*}
|s_nt_n - 0| &= |s_nt_n|\\
&= |s_n|\cdot |t_n|\\
&\leq |s_n| \cdot M &\text{since } |t_n| \leq M \; \forall \; n \in \mathbb{N}\\
&= |s_n - 0| \cdot M\\
&\leq \varepsilon_1 \cdot M &\text{for all } n > N_1 \text{ by assumption}\\
&= \frac{\varepsilon_2}{M} \cdot M &\text{by our choice of } \varepsilon_1\\
&= \varepsilon_2
\end{align*}
Thus, we have shown that $|s_nt_n - 0| < \varepsilon_2$ under the condition that $n > N_1$. Therefore, since $\varepsilon_2$ was arbitrarily given to us, we can conclude that $\lim (s_nt_n) = 0$, just as desired. 
\end{proof}
\section*{8.5}
\begin{enumerate}[label = (\alph*)]
\item Consider three sequences $(a_n), (b_n)$, and $(s_n)$ such that $a_n \leq s_n \leq b_n$ for all $n$ and $\lim a_n = \lim b_n = s$. Prove that $\lim s_n = s$. 
\item Suppose that $(s_n)$ and $(t_n)$ are sequences such that $|s_n| \leq t_n$ for all $n$ and $\lim t_n = 0$. Prove that $\lim s_n = 0$.
\end{enumerate}

\begin{proof}{\textbf{(a)}}
\\What we know:
\begin{enumerate}[label = (\arabic*)]
\item $a_n \leq s_n$ and $s_n \leq b_n$ for all $n \in \mathbb{N}$
\item For every $\varepsilon_1 > 0$ there exists some $N_1$ such that $|a_n - s| < \varepsilon_1$ for all $n > N_1$ (since $\lim a_n = s$)
\item For every $\varepsilon_2 > 0$ there exists some $N_2$ such that $|b_n - s| < \varepsilon_2$ for all $n > N_1$ (since $\lim a_n = s$)
\item $|x - y| < z$ is equivalent to $y - z < x < y + z$ according to Exercise (3.7)(b)
\end{enumerate}
I will use items (2) and (4) above to conclude that $s - \varepsilon_1 < a_n < s + \varepsilon_1$. Next, I will use items (3) and (4) above to conclude that $s - \varepsilon_2 < b_n < s + \varepsilon_2$. Using these results, I can conclude the following:
\begin{align*}
s - \varepsilon_1 < a_n \quad \text{and} \quad a_n \leq s_n && \implies && s - \varepsilon_1 < s_n\\
b_n < s + \varepsilon_2 \quad \text{and} \quad s_n \leq b_n && \implies && s_n < s + \varepsilon_2
\end{align*}
Since lines (2) and (3) above hold for all $\varepsilon_1$ or $\varepsilon_2$, respectively, I can choose $\varepsilon_1 = \varepsilon_2 = \varepsilon > 0$ and still get the same results above. Therefore, we have that $s - \varepsilon < s_n < s + \varepsilon$. Again, from line (4) above, this shows that $|s_n - s| < \varepsilon$. However, since line (2) only holds for $n > N_1$ and line (3) only holds for $n > N_2$, then we can only validly claim that the last inequality holds whenever $n > N := \max\{N_1, N_2\}$. Therefore, we have, at last, shown that for any given $\varepsilon > 0$, we have $|s_n - s| < \varepsilon$ for all $n > N$; thus, $\lim s_n = s$.
\end{proof}

\begin{proof}{\textbf{(b)}}
\\From Exercise (3.5)(a), we know that $|b| \leq a$ if and only if $-a \leq b \leq a$. From this we can use the assumption that $|s_n| \leq t_n$ for all $n$ to conclude that $-t_n \leq s_n \leq t_n$ for all $n$. Furthermore, we know from the assumption that $\lim t_n = 0$ that for some given $\varepsilon > 0$, that $|t_n - 0| = |t_n| < \varepsilon$ for all $n$ satisfying $n > N$ for some $N$. Using Exercise (3.7)(a), we can see that $-\varepsilon < t_n < \varepsilon$ for $n > N$. Thus,
\begin{align*}
t_n < \varepsilon && \implies && -\varepsilon < -t_n\\
-\varepsilon < -t_n \quad \text{and} \quad -t_n \leq s_n && \implies && -\varepsilon < s_n\\
s_n \leq t_n \quad \text{and} \quad t_n < \varepsilon && \implies && s_n < \varepsilon\\
-\varepsilon < s_n \quad \text{and} \quad s_n < \varepsilon && \implies && |s_n| < \varepsilon &\text{ by Exercise (3.7)(a) once again}\\
\end{align*}
These above inequalities only hold for $n > N$. Thus, we know that for some given $\varepsilon > 0$, we have that $|s_n| = |s_n - 0| < \varepsilon$ for all $n > N$ for the same $N$ as given above. Thus, we have shown that $\lim s_n = 0$.
\end{proof}

\section*{8.6}
Let $(s_n)$ be a sequence in $\mathbb{R}$
\begin{enumerate}[label = (\alph*)]
\item Prove $\lim s_n = 0$ if and only if $\lim |s_n| = 0$.
\item Observe that if $s_n = (-1)^n$, then $\lim |s_n|$ exists, but $\lim s_n$ does not exist. 
\end{enumerate}

\begin{proof}{\textbf{(a)}}
\\$\rightarrow$
\\Assume that $\lim s_n = 0$, i.e. given some arbitrary $\varepsilon > 0$, there exists some $N$ such that $|s_n - 0| = |s_n| < \varepsilon$ for all $n > N$. Now, I will examine $||s_n| - 0|$:
\begin{align*}
||s_n| - 0| &= ||s_n||\\
&= |s_n|\\
&< \varepsilon \quad \quad \forall \; n > N \text{ by assumption of the the limit of } s_n
\end{align*}
Thus, $||s_n| - 0| < \varepsilon$ for all $n > N$, so we have shown that $\lim |s_n| = 0$.
\\$\leftarrow$
\\Assume that $\lim |s_n| = 0$, i.e. given some arbitrary $\varepsilon > 0$, there exists some $N$ such that $||s_n| - 0| = ||s_n|| = |s_n| < \varepsilon$ for all $n > N$. I will examine $|s_n - 0|$:
\begin{align*}
|s_n - 0| &= |s_n|\\
&< \varepsilon \quad \quad \forall \; n > N \text{ by assumption of the the limit of } |s_n|
\end{align*}
Thus, we have shown that $|s_n - 0| < \varepsilon$ for all $n> N$, so we have shown that $\lim s_n = 0$.
\end{proof}

\begin{proof}{\textbf{(b)}}
\\First, note that $|s_n| = |(-1)^n| = 1$. Thus, $\lim |s_n| = \lim 1 = 1$ since $|1 - 1| = 0 < \varepsilon$ for all $\varepsilon > 0$ no matter what $N$ we choose. Therefore, $\lim |s_n| = 1$.
\\
\\However, when we look at $\lim s_n = \lim (-1)^n$, we get a different picture. Suppose that $\lim s_n$ does exist, i.e., $\lim s_n = s$ for some $s \in \mathbb{R}$. By the definition of the limit this means that for every $\varepsilon > 0$, there exists some $N$ such that $|s_n - s| < \varepsilon$ for all $n > N$. For convenience, let's limit ourselves to the specific case where $\varepsilon = 1$. This means that $|(-1)^n - s| < 1$ for all $n > N$. Let's break this down into two different cases, we have that $|-1 - s| < 1$ for all odd $n > N$ and $|1 - s| < 1$ for all even $n > N$. This means that the number $s$ must satisfy both of these inequalities. However, if we take these two inequalities to be true we get the following:
\begin{align*}
2 = |1 - (-1)| = |(1 - s) - (-1 - s)| = |(1 - s) + -1(-1 - s)| &\leq |1 - s| + |-1(-1 - s)| &\text{by Triangle Inequality}\\
&= |1 - s| + |-1 - s|\\
&< 1 + 1 &\text{by the above inequalities}\\
&= 2
\end{align*}
However, this asserts that $2 < 2$, which is ridiculous, so our initial assumption that the limit exists must have been false. Thus, $\lim s_n$ can not exist. 
\end{proof}

\section*{8.7}
Show that the following sequences do not converge.
\begin{enumerate}[label = (\alph*)]
\item $\cos\left(\frac{n\pi}{3}\right)$
\item $s_n = (-1)^n n$
\item $\sin\left(\frac{n\pi}{3}\right)$
\end{enumerate}

\begin{proof}{\textbf{(a)}}
\\Assume that the sequence converges (equivalently that the limit of the sequence exists), i.e. $\displaystyle \lim_{n \to \infty} \left(\cos\left(\frac{n\pi}{3}\right)\right) = a$ for some $a \in \mathbb{R}$. This means that for every $\varepsilon > 0$, there exists some $N$ such that $\displaystyle \left|\cos\left(\frac{n\pi}{3}\right) - a\right| < \varepsilon$ for all $n > N$. In particular, if $n>N$ is of the form $n = 6k$ or $n = 6k + 3$ for $k \in \mathbb{N}$, and we choose $\varepsilon = 1$, then we get the following 2 inequalities: $|\cos(\frac{6k\pi}{3}) - a| = |1 - a| < 1$ and $|\sin(\frac{(6k + 3)\pi}{3}) - a| = |-1 - a| < 1$. However, in Question 8.6(b) of this homework assignment, we have already shown that the existence of these two inequalities at one time leads to a contradiction asserting that $2 < 2$, which cannot happen. Therefore, our assumption on the existence of the limit must have been false, so this sequence does not converge. 
\end{proof}

\begin{proof}{\textbf{(b)}}
\\Assume that the sequence converges (equivalently that the limit of the sequence exists), i.e. $\lim (-1)^n n = b$ for some $b \in \mathbb{R}$. This means that for every $\varepsilon > 0$, there exists some $N$ such that $|(-1)^n n - b| < \varepsilon$ for all $n > N$. Let us consider $n > N$ to be even and then also consider $n + 2$ which must also be even (Note that $(-1)^n n = n$ for $n $ even). Furthermore, let $\varepsilon = 1$, giving the following inequalities: $|n - b| < 1$ and $|n + 2 - b| < 1$. From these inequalities, we can get:
\begin{align*}
2 = |(n + 2 - b) - (n - b)| = |(n + 2 - b) + -1(n - b)| &\leq |n + 2 - b| + |-1(n - b)| &\text{by Triangle Inequality}\\
&= |n + 2 -b| + |n - b|\\
&< 1 + 1 &\text{by the above inequalities}\\
&= 2
\end{align*}
Therefore, we conclude that $2<2$ (uh-oh). This is a contradiction, so our assumption that the limit exists must have been false; therefore, the sequence does not converge. 
\end{proof}

\begin{proof}{\textbf{(c)}}
\\Assume that the sequence converges (equivalently that the limit of the sequence exists), i.e. $\displaystyle \lim_{n \to \infty} \left(\sin\left(\frac{n\pi}{3}\right)\right) = c$ for some $c \in \mathbb{R}$. This means that for every $\varepsilon > 0$, there exists some $N$ such that $\displaystyle \left|\sin\left(\frac{n\pi}{3}\right) - c\right| < \varepsilon$ for all $n > N$. In particular, I will consider the cases where $n > N$ is of the form $n = 6k + 1$ and when $n > N$ is of the form $n = 6k + 4$ for $k \in \mathbb{N}$. Furthermore, taking $\varepsilon = \frac{\sqrt{3}}{2}$ gives the following two inequalities: $|\sin(\frac{(6k + 1)\pi}{3}) - c| = |\frac{\sqrt{3}}{2} - c| < \frac{\sqrt{3}}{2}$ and $|\sin(\frac{(6k + 4)\pi}{3}) - c| = |-\frac{\sqrt{3}}{2} - c| < \frac{\sqrt{3}}{2}$. From these inequalities, we get:
\begin{align*}
\sqrt{3} = \left|\frac{\sqrt{3}}{2} - \left(-\frac{\sqrt{3}}{2}\right)\right| &= \left|\left(\frac{\sqrt{3}}{2} - c\right) + -1\left(-\frac{\sqrt{3}}{2} - c\right)\right|\\
&\leq \left|\frac{\sqrt{3}}{2} - c\right| + \left|-1\left(-\frac{\sqrt{3}}{2} - c\right)\right| &\text{by Triangle Inequality}\\
&= \left|\frac{\sqrt{3}}{2} - c\right| + \left|-\frac{\sqrt{3}}{2} - c\right|\\
&< \frac{\sqrt{3}}{2} + \frac{\sqrt{3}}{2} &\text{by the above inequalities}\\
&= \sqrt{3}
\end{align*}
This shows that $\sqrt{3} < \sqrt{3}$ which is obviously a contradiction. Therefore, my assumption that the limit exists must have been false; thus, the sequence does not converge. 
\end{proof}

\section*{8.9}
Let $(s_n)$ be a sequence that converges.
\begin{enumerate}[label = (\alph*)]
\item Show that if $s_n \geq a$ for all but finitely many $n$, then $\lim s_n \geq a$.
\item Show that if $s_n \leq b$ for all but finitely many $n$, then $\lim s_n \leq b$.
\item Conclude that if all but finitely many $s_n$ belong to $[a,b]$, then $\lim s_n \in [a,b]$.
\end{enumerate}

\begin{proof}{\textbf{(a)}}
\\If $s_n \geq a$ for all but finitely many $n$, then that means there exists some $N_0$ such that $s_n \geq a$ for all $n > N_0$. Let $s = \lim s_n$ and assume that $s < a$. Since the sequence converges, $s \in \mathbb{R}$ and we know that for any $\varepsilon > 0$ there exists some $N$ such that $|s_n - s| < \varepsilon$ for all $n > N$. Let $\varepsilon = a - s$ (note $\varepsilon > 0$ since we assumed $s < a$) and use Exercise (3.7)(b) to conclude that $s - (a - s) < s_n < s + (a - s)$ for all $n > N$ which simplifies to $2s - a < s_n < a$ for all $n > N$. If we make sure our $N$ above satisfies $N > N_0$, then this shows that $s_n < a$ for all $n > N > N_0$. However, one of our conditions on $s_n$ is that it satisfies $s_n \geq a$ for all $n > N_0$, so we have a contradiction. Therefore, our assumption that $s < a$ must have been false, therefore $s \geq a$, just as desired. 
\end{proof}

\begin{proof}{\textbf{(b)}}
\\If $s_n \leq b$ for all but finitely many $n$, then that means there exists some $N_0$ such that $s_n \leq b$ for all $n > N_0$. Let $s = \lim s_n$ and assume that $s > b$. Since the sequence converges, $s \in \mathbb{R}$ and we know that for any $\varepsilon > 0$ there exists some $N$ such that $|s_n - s| < \varepsilon$ for all $n > N$. Let $\varepsilon = s - b$ (note $\varepsilon > 0$ since we assumed $s > b$) and use Exercise (3.7)(b) to conclude that $s - (s - b) < s_n < s + (s - b)$ for all $n > N$ which simplifies to $b < s_n < 2s - b$ for all $n > N$. If we make sure our $N$ above satisfies $N > N_0$, then this shows that $s_n > b$ for all $n > N > N_0$. However, one of our conditions on $s_n$ is that it satisfies $s_n \leq b$ for all $n > N_0$, so we have a contradiction. Therefore, our assumption that $s > b$ must have been false, therefore $s \leq b$, just as desired. 
\end{proof}

\begin{proof}{\textbf{(c)}}
\\If $s_n \in [a, b]$ for all but finitely many $n$, then that means there exists some $N_0$ such that $a \leq s_n \leq b$ for all $n > N_0$. Let $s = \lim s_n$, then by part (a) of this question, we know that $a \leq s$. Similarly, by part (b) of this question we can conclude that $s \leq b$. Together, this tells us that $a \leq s \leq b$ or, equivalently, $s \in [a, b]$, just as desired. 
\end{proof}

\section*{8.10}
Let $(s_n)$ be a convergent sequence, and suppose that $\lim s_n > a$. Prove there exists a number $N$ such that $n > N$ implies $s_n > a$.

\begin{proof}{$ $}
\\Since $s_n$ converges, we know that $\lim s_n = s$ for some $s \in \mathbb{R}$, i.e. for every $\varepsilon > 0$, there exists some $N$ such that $|s_n - s| < \varepsilon$ for all $n > N$. Since this question supposes that $s > a$, we may choose $\varepsilon = s - a$ (a positive number) and use Exercise (3.7)(b) to conclude that $s - (s - a) < s_n < s + (s - a)$ for all $n > N$ which simplifies to $a < s_n < 2s - a$ for all $n > N$. Thus, the left of these two inequalities shows that there does, indeed, exist some $N$ such that $s_n > a$ for all $n > N$ and that $N$ is precisely the same $N$ that guarantees $|s_n - s| < s - a$ for all $n > N$. 
\end{proof}

\end{document}