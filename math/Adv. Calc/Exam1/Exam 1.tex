\documentclass[10pt,a4paper]{article}
\usepackage[utf8]{inputenc}
\usepackage[english]{babel}
\usepackage{csquotes}
\usepackage{amsmath}
\usepackage{amsfonts}
\usepackage{amssymb}
\usepackage{graphicx}
\usepackage[margin=0.5in]{geometry}
\usepackage{amsthm}
\usepackage{enumitem}
\usepackage{tikz}
\usetikzlibrary{calc}
\newtheorem{question}{Question}
\newtheorem*{question*}{Question}
\newtheorem{theorem}{Theorem}
\newtheorem*{theorem*}{Theorem}
\newtheorem{lemma}{Lemma}

\theoremstyle{definition}
\newtheorem{answer}{Answer}
\newtheorem*{answer*}{Answer}


\title{Advanced Calc. Exam 1}
\author{Colin Williams}

\begin{document}
\maketitle

\section*{Question 1}
Prove that $P(n) = 8^n - 3^n$ is divisible by 5 for every $n \in \mathbb{N}$.

\begin{proof}{$ $}
\\\underline{Base case $n = 1$:}
\\For $n = 1$, we have $P(n) = P(1) = 8^1 - 3^1 = 8 - 3 = 5$ and 5 is clearly divisible by 5, so the above statement holds for $n=1$.
\\
\\\underline{Inductive Step:}
\\First, note that for a number to be divisible by 5, that is equivalent to saying that number is an integer multiple of 5. From this, assume that $P(n)$ is divisible by 5 for some $n = k \in \mathbb{N}$, i.e. $P(k) = 8^k - 3^k = 5m$ for some integer $m$, this is the Inductive Hypothesis. Next, I will examine whether $P(k+1)$ must also be divisible by 5 in the following manner:
\begin{align*}
P(k+1) &= 8^{k+1} - 3^{k+1} &\text{by the definition of } P(n)\\
&= 8\cdot 8^k - 3\cdot 3^k &\text{by exponent properties}\\
&= 8\cdot (8^k - 3^k + 3^k) - 3\cdot 3^k &\text{by adding and subtracting } 3^k\\
&= 8\cdot (8^k - 3^k) + 8\cdot 3^k - 3\cdot 3^k &\text{by distributing the 8}\\
&= 8\cdot (8^k - 3^k) + (8 - 3)\cdot 3^k &\text{by factoring}\\
&= 8\cdot (5m) + 5\cdot 3^k &\text{by using the Inductive Hypothesis}\\
&= 5(8m + 3^k) &\text{by factoring}.
\end{align*}
Thus, we have shown that $P(k+1) = 5\widetilde{m}$ for $\widetilde{m} = 8m + 3^k \in \mathbb{Z}$. Therefore, we have shown that if $P(k)$ is divisible by 5 for any $k \in \mathbb{N}$, then we must have that $P(k+1)$ is divisible by 5 as well. Additionally, since we have shown that $P(1)$ is divisible by 5, then by the Principle of Mathematical Induction, we can conclude that $P(n)$ is divisible by 5 for all $n \in \mathbb{N}$, exactly what we wanted to show.
\end{proof}

\newpage

\section*{Question 2}
Is $\sqrt[3]{5 - \sqrt{5}}$ a rational number?

\begin{answer*}{$ $}
\\First, note that if we let $x_0 = \sqrt[3]{5 - \sqrt{5}}$, we can find a polynomial with $x_0$ as one of its roots by doing the following algebraic manipulations:
\begin{align*}
x_0 &= \sqrt[3]{5 - \sqrt{5}}\\
\implies x_0^3 &= 5 - \sqrt{5} &\text{by cubing both sides}\\
\implies x_0^3 - 5 &= -\sqrt{5} &\text{by subracting 5 from both sides}\\
\implies (x_0^3 - 5)^2 &= 5 &\text{by squaring both sides}\\
\implies x_0^6 - 10x_0^3 + 25 &= 5 &\text{by expanding the binomial}\\
\implies x_0^6 - 10x_0^3 + 20 &= 0 &\text{by subracting 5 from both sides}
\end{align*}
Thus, we can see that $x_0 = \sqrt[3]{5 - \sqrt{5}}$ is a root of the polynomial $f(x) = x^6 - 10x^3 + 20$. Therefore, we can now apply the \enquote{Rational Zeros Theorem} to the polynomial $f$ with $n = 6$, $c_6 = 1$, $c_5 = c_4 = 0$, $c_3 = -10$, $c_2 = c_1 = 0$, and $c_0 = 20$. The Rational Zeros Theorem states that if $f$ has any rational zeros, $r$, then they must be of the form $\displaystyle r = \frac{c}{d}$ where $c$ divides $c_0 = 20$ and $d$ divides $c_6 = 1$. From this we know that $c \in \{\pm 1, \pm 2, \pm 4, \pm 5, \pm 10, \pm 20\}$ and $d \in \{\pm 1\}$ which implies that $r \in \{\pm 1, \pm 2, \pm 4, \pm 5, \pm 10, \pm 20\}$. Now, let's evaluate $f$ at all of the possible values for $r$:
\begin{align*}
f(1) &= (1)^6 - 10(1)^3 + 20 &= 11\\
f(-1) &= (-1)^6 - 10(-1)^3 + 20 &= 31\\
f(2) &= (2)^6 - 10(2)^3 + 20 &= 4\\
f(-2) &= (-2)^6 - 10(-2)^3 + 20 &= 164\\
f(4) &= (4)^6 - 10(4)^3 + 20 &= 3476\\
f(-4) &= (-4)^6 - 10(-4)^3 + 20 &= 4756\\
f(5) &= (5)^6 - 10(5)^3 + 20 &= 14395\\
f(-5) &= (-5)^6 - 10(-5)^3 + 20 &= 16895\\
f(10) &= (10)^6 - 10(10)^3 + 20 &= 990020\\
f(-10) &= (-10)^6 - 10(-10)^3 + 20 &= 1010020\\
f(20) &= (20)^6 - 10(20)^3 + 20 &= 63920020\\
f(-20) &= (-20)^6 - 10(-20)^3 + 20 &= 64080020\\
\end{align*}
Clearly, none of these values are $0$, so none of the possible values for $r$ are actually zeros of $f$. Thus, $f$ has no rational zeros, so since $x_0 = \sqrt[3]{5 - \sqrt{5}}$ \textit{is} a root of $f$, it cannot be rational. Therefore,\boxed{\sqrt[3]{5 - \sqrt{5}} \text{ is \underline{not} a rational number}}.
\end{answer*}

\newpage

\section*{Question 3}
Consider the set $X := \{\square, \triangle\}$ with operation \enquote{+} defined as follows:
\begin{align*}
\square + \square = \square && \square + \triangle = \triangle && \triangle + \square = \triangle && \triangle + \triangle = \square
\end{align*}
Show that $(X, +)$ satisfies A1 - A4 and find which element of $X$ plays the role of \enquote{0}.

\begin{answer*}{$ $}
\\\underline{A1}
\\I will check that $a + (b + c) = (a + b) + c$ for all $a,b,c \in X$:
\begin{align*}
\square + (\square + \square) = \square + \square = \square &\text{\quad and \quad } (\square + \square) + \square = \square + \square = \square &\checkmark\\
\square + (\square + \triangle) = \square + \triangle = \triangle &\text{\quad and \quad } (\square + \square) + \triangle = \square + \triangle = \triangle &\checkmark\\
\square + (\triangle + \square) = \square + \triangle = \triangle &\text{\quad and \quad } (\square + \triangle) + \square = \triangle + \square = \triangle &\checkmark\\
\square + (\triangle + \triangle) = \square + \square = \square &\text{\quad and \quad } (\square + \triangle) + \triangle = \triangle + \triangle = \square &\checkmark\\
\triangle + (\square + \square) = \triangle + \square = \triangle &\text{\quad and \quad } (\triangle + \square) + \square = \triangle + \square = \triangle &\checkmark\\
\triangle + (\square + \triangle) = \triangle + \triangle = \square &\text{\quad and \quad } (\triangle + \square) + \triangle = \triangle + \triangle = \square &\checkmark\\
\triangle + (\triangle + \square) = \triangle + \triangle = \square &\text{\quad and \quad } (\triangle + \triangle) + \square = \square + \square = \square &\checkmark\\
\triangle + (\triangle + \triangle) = \triangle + \square = \triangle &\text{\quad and \quad } (\triangle + \triangle) + \triangle = \square + \triangle = \triangle &\checkmark\\
\end{align*}
Thus, A1 is true for all possible choices of $a, b, c \in X$, so A1 is satisfied.
\\\underline{A2}
\\I will check that $a + b = b + a$ for all $a,b \in X$. First, note that for $a = b$, this is obviously true as the left and right sides of the equation would be the exact same expressions. Thus, I will consider all cases where $a \neq b$:
\begin{align*}
\square + \triangle = \triangle &\text{\quad and \quad} \triangle + \square = \triangle &\checkmark\\
\triangle + \square = \triangle &\text{\quad and \quad} \square + \triangle = \triangle &\checkmark
\end{align*}
Therefore, $a + b = b + a$ for all $a, b \in X$ both when $a = b$ and when $a \neq b$, so A2 is satisfied. 
\\\underline{A3}
\\I will check that $a + 0 = a$ for all $a \in X$. First, note that \enquote{0} in this context is given by $\square$, I will show this in the following:
\begin{align*}
\square + \square &= \square &\checkmark\\
\triangle + \square &= \triangle &\checkmark
\end{align*}
Thus, $a + \square = a$ for all $a \in X$, so \boxed{\text{\enquote{0}} = \square} and A3 is satisfied. 
\\\underline{A4}
\\I will check that for every $a \in X$, there exists an element $-a \in X$ such that $a + (-a) = 0$. In this context, since \enquote{0}$ = \square$, we are trying to show that this $-a$ satisfies $a + (-a) = \square$. I will try to find this $-a$ for every element in $X$:
\begin{align*}
\text{for } a = \square,\; -a = \square \text{ since } && \square + \square = \square &&\checkmark\\
\text{for } a = \triangle,\; -a = \triangle \text{ since } && \triangle + \triangle = \square &&\checkmark
\end{align*}
Thus, $-a$ exists for every $a \in X$, so A4 is satisfied. 
\end{answer*}

\newpage

\section*{Question 4}
Let $S, T \subset \mathbb{R}$ be nonempty bounded sets. Provide a proof or counterexample for the following:
\begin{align}
\sup(S) \leq \sup(T) &\implies S \subset T
\end{align}

\begin{answer*}{$ $}
\\I will provide a counterexample to show that the previous statement is false. Let $S = (2, 5)$ and let $T = (6, 10)$, then it is easy to see that $\sup(S) = 5$ and $\sup(T) = 10$. Thus, the condition for (1) is satisfied as $\sup(S) = 5 \leq 10 = \sup(T)$. However, the implication is not true because, for example, $\pi \in S$ but $\pi \not\in T$, so it is clear that $S \not\subset T$. Therefore, \boxed{\text{this counterexample shows that (1) is false}}.
\end{answer*}

\newpage

\section*{Question 5}
Fix an arbitrary $a \in \mathbb{R}$ and define the set $S := \{r \in \mathbb{Q}: r < a\}$. Show that $\sup(S) = a$.

\begin{answer*}{$ $}
\\Since $\sup(S)$ is defined as the least upper bound of $S$, we first need to check that $a$ is indeed an upper bound of $S$. However, due to how $S$ is defined, every $r \in S$ satisfies $r < a$, so in particular $r \leq a$ for all $r \in S$; thus, $a$ is an upper bound by the definition of Upper Bound.
\\
\\Next, I will show that $a$ is actually the \underline{Least} Upper Bound. To do this, assume that there exists a smaller upper bound for $S$, i.e. let $b \in \mathbb{R}$ be an upper bound for $S$ with $b < a$. However, by the Density of $\mathbb{Q}$, we know that for any two $a, b \in \mathbb{R}$ with $b < a$ there exists some $r \in \mathbb{Q}$ such that $b < r < a$. This implies that there is an $r_0 \in S$ that is greater than $b$ meaning $b$ cannot actually be an upper bound. Thus, our assumption of the existence of a smaller upper bound must have been false, so we know that $a$ is not only an upper bound of $S$, but, in fact, the \underline{Least} Upper Bound of $S$. Equivalently, we have shown that $\sup(S) = a$, just as desired. 
\end{answer*}


\end{document}