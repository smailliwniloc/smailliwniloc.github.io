\documentclass[10pt,a4paper]{article}
\usepackage[utf8]{inputenc}
\usepackage[english]{babel}
\usepackage{amsmath}
\usepackage{amsfonts}
\usepackage{amssymb}
\usepackage{graphicx}
\usepackage[margin=0.5in]{geometry}
\usepackage{amsthm}
\usepackage{enumitem}
\usepackage{tikz}
\usetikzlibrary{calc}
\newtheorem{question}{Question}
\newtheorem*{question*}{Question}
\newtheorem{theorem}{Theorem}
\newtheorem*{theorem*}{Theorem}
\newtheorem{lemma}{Lemma}

\theoremstyle{definition}
\newtheorem{answer}{Answer}
\newtheorem*{answer*}{Answer}


\title{Advanced Calc. Homework 2}
\author{Colin Williams}

\begin{document}
\maketitle
\section*{Useful Results/Properties}
Below are several properties and results that will be useful and that I will reference throughout this assignment.
\\
\\\underline{Additive Properties:}
\begin{enumerate}[label = A\arabic*.]
\item $a + (b + c) = (a + b) + c \quad \forall \; a,b,c$.
\item $a + b = b + a \quad \forall \; a,b$.
\item $a + 0 = a \quad \forall \; a$.
\item $\forall \; a, \exists$ an element $-a$ such that $a + (-a) = 0$.
\end{enumerate}
{$ $}\\
\underline{Multiplicative Properties}
\begin{enumerate}[label = M\arabic*.]
\item $a(bc) = (ab)c \quad \forall \; a,b,c$.
\item $ab = ba \quad \forall \; a,b$.
\item $a \cdot 1 = a \quad \forall \; a$.
\item $\forall \; a \neq 0, \exists$ an element $a^{-1}$ such that $aa^{-1} = 1$.
\end{enumerate}
{$ $}\\
\underline{Distributive Law}
\begin{enumerate}[label = DL.]
\item $a(b + c) = ab + ac \quad \forall \; a,b,c$.
\end{enumerate}
{$ $}\\
\underline{Ordering Properties}
\begin{enumerate}[label = O\arabic*.]
\item Given $a$ and $b$, either $a \leq b$ or $b \leq a$.
\item If $a \leq b$ and $b \leq a$, then $a = b$.
\item If $a \leq b$ and $b \leq c$, then $a \leq c$.
\item If $a \leq b$, then $a + c \leq b + c \quad \forall \; c$.
\item If $a \leq b$ and $0 \leq c$, then $ac \leq bc$.
\end{enumerate}

\begin{theorem*}{\textbf{3.1}}
\\The following are consequences of the field properties above for $a,b,c \in \mathbb{R}$:
\begin{enumerate}[label = (\roman*)]
\item $a + c = b = c$ implies $a = b$.
\item $a \cdot 0 = 0 \quad \forall \; a$.
\item $(-a)b = -ab \quad \forall \; a,b$.
\item $(-a)(-b) = ab \quad \forall \; a,b$.
\item $ac = bc$ and $c \neq 0$ implies $a = b$.
\item $ab = 0$ implies either $a = 0$ or $b = 0$.
\end{enumerate}

\end{theorem*}

\begin{theorem*}{\textbf{3.2}}
\\The following are consequences of the properties of an ordered field for $a,b,c \in \mathbb{R}$:
\begin{enumerate}[label = (\roman*)]
\item If $a \leq b$, then $-b \leq -a$.
\item If $a \leq b$ and $c \leq 0$, then $bc \leq ac$.
\item If $0 \leq a$ and $0 \leq b$, then $0 \leq ab$.
\item $0 \leq a^2 \quad \forall \; a$.
\item $0 < 1$.
\item If $0 < a, then 0 < a^{-1}$.
\item If $0 < a < b$, then $0 < b^{-1} < a^{-1}$.
\end{enumerate}
Note that $a < b$ means $a \leq b$ and $a\neq b$.
\end{theorem*}

\begin{theorem*}{\textbf{3.5}}
\\These represent the basic properties of the absolute value:
\begin{enumerate}[label = (\roman*)]
\item $0 \leq |a| \quad \forall \; a \in \mathbb{R}$
\item $|ab| = |a| \cdot |b| \quad \forall \; a, b \in \mathbb{R}$
\end{enumerate}
\end{theorem*}

\begin{theorem*}{\textbf{Triangle Inequality}}
\\$|a + b| \leq |a| + |b| \quad \forall \; a,b \in \mathbb{R}$.
\end{theorem*}
{$ $}
\\With all of that in place, let's begin with the questions:

\section*{3.1}
\begin{question*}{\textbf{a.)}}
\\Which of the properties {\normalfont A1-A4, M1-M4, DL, O1-O5} fail for $\mathbb{N}$?
\end{question*}

\begin{answer*}{\textbf{a.)}}
\\Since $\mathbb{N} \subset \mathbb{R}$, then the "$0$" element in $\mathbb{R}$ must be the same at the "$0$" element in $\mathbb{N}$. However, this $0$ is not an element in $\mathbb{N}$ since $\mathbb{N} = \{1,2,3,\ldots\}$; thus, it is impossible for A3 to hold since A3 implies the existence of 0 in the set.
\\Likewise, for some element $a \in \mathbb{N}$, A4 implies the existence of $-a$. However, $-a \notin \mathbb{N}$ for any $a \in \mathbb{N}$, so A4 also does not hold.
\\Lastly, M4 does not hold either since for any $a \in \mathbb{N}$ with $a \neq 1$, $a^{-1} \notin \mathbb{N}$. However, M4 requires that $a^{-1}$ exist for all $a \neq 0$ and this is not true. 
\\Therefore, \fbox{A3, A4, and M4 do not hold for $\mathbb{N}$}.

\end{answer*}

\begin{question*}{\textbf{b.)}}
\\Which of the properties {\normalfont A1-A4, M1-M4, DL, O1-O5} fail for $\mathbb{Z}$?
\end{question*}

\begin{answer*}{\textbf{b.)}}
\\M4 once again does not hold in $\mathbb{Z}$ since with $a \in \mathbb{Z}$ only for $a = 1$ or $a = -1$ does $a^{-1}$ exist. Since M4 requires the existence of $a^{-1}$ for all $a \in \mathbb{Z}$, \fbox{M4 does not hold for $\mathbb{Z}$}.
\end{answer*}
\section*{3.2}
\begin{question*}{\textbf{a.)}}
\\The commutative law {\normalfont A2} was used in the proof of (ii) in Theorem 3.1. Where?
\end{question*}

\begin{answer*}{\textbf{a.)}}
\\For convenience, here is the proof of (ii) in Theorem 3.1:
\begin{proof}{\textbf{Theorem 3.1 (ii)}}
\\Let $a,b \in \mathbb{R}$, then
\begin{align}
a \cdot 0 &= a \cdot (0 + 0) &\text{by A3}\\
&= a \cdot 0 + a \cdot 0 &\text{by DL}\\
\implies a \cdot 0 + 0 &= a \cdot 0 + a \cdot 0 &\text{since A3 says } a \cdot 0 = a \cdot 0 + 0\\
\implies 0 + a \cdot 0 &= a \cdot 0 + a \cdot 0 &\text{by A2}\\
\implies 0 &= a \cdot 0 &\text{by Theorem 3.1 (i)}\\
\implies a \cdot 0 &= 0 &\text{by symmetry of "=" sign}
\end{align}
\end{proof}
This is the full proof, and it is clear by my annotating that \fbox{A2 is used going from line (3) to line (4)} on the left hand side of the equation. 
\end{answer*}

\begin{question*}{\textbf{b.)}}
\\The commutative law {\normalfont A2} was also used in the proof of (iii) in Theorem 3.1. Where?
\end{question*}

\begin{answer*}{\textbf{b.)}}
\\Again, for convenience, here is the proof of (iii) in Theorem 3.1:

\begin{proof}{\textbf{Theorem 3.1 (iii)}}
\\Let $a,b \in \mathbb{R}$, then by A4, we know that
\begin{equation}
ab + (-ab) = 0.
\end{equation}
Also, we can find that:
\begin{align}
ab + (-a)b &= ba + b(-a) &\text{by M2 used twice}\\
&= b(a + (-a)) &\text{by DL}\\
&= b \cdot 0 &\text{by A4}\\
&= 0 &\text{by Theorem 3.1 (ii).}
\end{align}
Thus,
\begin{align}
ab + (-ab) &= ab + (-a)b &\text{since LHS and RHS both equal 0}\\
\implies (-ab) + ab &= (-a)b + ab &\text{by A2 used on both sides}\\
\implies -ab &= (-a)b &\text{by Theorem 3.1 (i)}\\
\implies (-a)b &= -ab &\text{by symmetry of "=".}
\end{align}
\end{proof}
This completes the proof, and you can see that by my annotations, \fbox{we use A2 going from line (12) to line (13)} on the LHS and the RHS . 
\end{answer*}

\section*{3.3}
\begin{question*}{$ $}
\\Prove (iv) and (v) of Theorem 3.1.
\end{question*}

\begin{proof}{\textbf{Theorem 3.1 (iv)}}
\\Let $a,b \in \mathbb{R}$, then by A4, we know that
\begin{equation}
ab + (-ab) = 0.
\end{equation}
Also, we can find that:
\begin{align*}
(-a)(-b) + (-ab) &= (-a)(-b) + (-a)b &\text{by Theorem 3.1 (iii)}\\
&= (-a)((-b) + b) &\text{by DL}\\
&= (-a)(b + (-b)) &\text{by A2}\\
&= (-a) \cdot 0 &\text{by A4}\\
&= 0 &\text{by Theorem 3.1 (ii)}
\end{align*}
Thus, 
\begin{align*}
(-a)(-b) + (-ab) &= ab + (-ab) &\text{since LHS and RHS are both equal to 0}\\
\implies (-a)(-b) &= ab &\text{by Theorem 3.1 (i)}
\end{align*}
\end{proof}

\begin{proof}{\textbf{Theorem 3.1 (v)}}
\\Let $a,b,c \in \mathbb{R}$. Since $c \neq 0$, M4 guarantees the existence of $c^{-1}$. Thus,
\begin{align*}
ac &= bc \\
\implies (ac)c^{-1} &= (bc)c^{-1} &\text{by multiplying both sides by } c^{-1}\\
\implies a(cc^{-1}) &= b(cc^{-1}) &\text{by M1 on both sides of the equation}\\
\implies a \cdot 1 &= b \cdot 1 &\text{by M4 on both sides of the equations}\\
\implies a &= b &\text{by M3}
\end{align*}

\end{proof}

\section*{3.4}
\begin{question*}{$ $}
\\Prove (v) and (vii) of Theorem 3.2.

\begin{proof}{\textbf{Theorem 3.2 (v)}}
\\We must prove, individually, that $0 \leq 1$ and that $0 \neq 1$. First, let $a \in \mathbb{R}$. By Theorem 3.2 (iv), we know that $0 \leq a^2$. If we choose $a = 1$, then we immediately get $0 \leq 1$ since $1^2 = 1 \cdot 1 = 1$ by M3. Thus, we only need to show that $0 \neq 1$. I will do so using a proof by contradiction:
\\
\\First, assume that $0 = 1$ and let $b$ be any element of $\mathbb{R}$. Then we get,
\begin{align*}
0 &= 1\\
\implies b \cdot 0 &= b \cdot 1 &\text{by multiplying both sides by b}\\
\implies 0 &= b &\text{by Theorem 3.1 (ii) on the LHS and M3 on the RHS}
\end{align*}
Since $b$ is any arbitrary element of $\mathbb{R}$ and we have shown that $b$ must be 0, this means all elements in $\mathbb{R}$ must be 0. Thus, $\mathbb{R}$ only has one element. However, this is nonsense, since $\mathbb{R}$ clearly has more than one element. Thus, our original claim that $0 = 1$ must have been false. Therefore, the negation of that statement must be true, i.e. $0 \neq 1$. 
\\
\\Since we've shown that $0 \leq 1$ and $0 \neq 1$, we have shown that $0 < 1$. 
\end{proof}

\begin{proof}{\textbf{Theorem 3.2 (vii)}}
\\Assume $0 < a < b$, i.e $0 < a$ and $a < b$. From this it follows that $0 \neq a$, $a \neq b$, and $0 \neq b$. Additionally, it tells us that $0 \leq a$ and $a \leq b$. From O3, it follows that $0 \leq b$. Since $0 \neq b$ and $0 \neq a$, we know that $0 < b$ and $0 < a$, so from Theorem 3.2 (vi), we can conclude that $0 < b^{-1}$ and $0 < a^{-1}$. Therefore, $0 \leq b^{-1}$, $0 \leq a^{-1}$, $0 \neq b^{-1}$, and $0 \neq a^{-1}$. Next, we will show that $b^{-1} < a^{-1}$. I will use the fact that we know $a \leq b$ to show this. 
\begin{align*}
a &\leq b\\
\implies aa^{-1} &\leq ba^{-1} &\text{by O5 and the fact that } 0 \leq a^{-1}\\
\implies 1 &\leq ba^{-1} &\text{by M4}\\
\implies 1 \cdot b^{-1} &\leq (ba^{-1})b^{-1} &\text{by O5 and the fact that } 0 \leq b^{-1}\\
\implies b^{-1} \cdot 1 &\leq b^{-1}(ba^{-1}) &\text{by M2}\\
\implies b^{-1} &\leq (b^{-1}b)a^{-1} &\text{by M3 on the left and M1 on the right}\\
&= (bb^{-1})a^{-1} &\text{by M2}\\
&= 1 \cdot a^{-1} &\text{by M4}\\
&= a^{-1} \cdot 1 &\text{by M2}\\
&= a^{-1} &\text{by M3}
\end{align*}
Thus, we've shown that $b^{-1} \leq a^{-1}$. To show $b^{-1} < a^{-1}$, we need to show $b^{-1} \neq a^{-1}$. I will prove this using a proof by contradiction:
\\
\\Assume that $b^{-1} = a^{-1}$. Then, we know that $aa^{-1} = 1$ by M4. Additionally, we know that:
\begin{align*}
ba^{-1} &= bb^{-1} &\text{by our assumption}\\
&= 1 &\text{by M4}
\end{align*}
Thus,
\begin{align*}
aa^{-1} &= ba^{-1} &\text{since LHS and RHS both equal 1}\\
\implies a &= b &\text{by Theorem 3.1 (v) and the fact that } a^{-1} \neq 0
\end{align*}
However, this last line is contradiction to the fact that $a \neq b$ as we stated previously. Therefore, our assumption that $b^{-1} = a^{-1}$ must be false, so its negation $b^{-1} \neq a^{-1}$ must be true.
\\
\\We have now shown all of the following:
\begin{itemize}
\item $0 < b^{-1}$
\item $b^{-1} \leq a^{-1}$
\item $b^{-1} \neq a^{-1}$
\end{itemize}
Thus, we can finally conclude that $0 < b^{-1} < a^{-1}$.
\end{proof}
\end{question*}

\section*{3.5}
\begin{question*}{\textbf{a.)}}
\\Show $|b| \leq a$ if and only if $-a \leq b \leq a$.
\end{question*}

\begin{proof}{\textbf{a.) $\rightarrow$}}
\\Assume that $|b| \leq a$. First, Theorem 3.5 (i) says that $0 \leq |b|$, so O3 says that $0 \leq a$. By the definition of $| \cdot |$, we have 2 cases: when $0 \leq b$ and when $b \leq 0$.
\\\underline{Case 1: $0 \leq b$}
\\If $0 \leq b$, then by the definition of $| \cdot |$ and the fact that $|b| \leq a$, we can see that $b \leq a$. Furthermore, since $0 \leq a$, we can see that $-a \leq -0$ by Theorem 3.2 (i). Lastly, I claim that $-0 = 0$:
\begin{align*}
0 + (-0) &= 0 &\text{by A4}\\
\implies 0 + (-0) &= 0 + 0 &\text{by A3}\\
\implies -0 + 0 &= 0 + 0 &\text{by A2 on the left}\\
\implies -0 &= 0 &\text{by Theorem 3.1 (i)}
\end{align*}
Thus, we've shown that $-a \leq 0$, so since $0 \leq b$, by O3, we have that $-a \leq b$. This, along with the fact that we already showed $b \leq a$ gives us $-a \leq b \leq a$, proving the statement for this case. 
{$ $}
\\\underline{Case 2: $b \leq 0$}
\\If $b \leq 0$, then by the defintion of $| \cdot |$ and the fact that $|b| \leq a$, we have that $-b \leq a$. From this inequality, we can use Theorem 3.2 (i), so say that $-a \leq -(-b)$. Next, I claim that $-(-b) = b$. First note that by A4, $b + (-b) = 0$. Also,
\begin{align*}
-(-b) + (-b) &= (-b) + [-(-b)] &\text{by A2}\\
&= 0 &\text{by A4 applied to } -b\\
\implies -(-b) + (-b) &= b + (-b) &\text{since we've shown that LHS and RHS both equal 0}\\
\implies -(-b) &= b &\text{by Theorem 3.1 (i)}
\end{align*}
Thus, we've shown that $-a \leq b$. To show $b \leq a$, we simply note that $b \leq 0$ and $0 \leq a$ which implies that $b \leq a$ by O3. Together this gives that that $-a \leq b \leq a$, proving the statement for this case.
\\Since we've proven the statement for all possible cases (since O1 states that either $0 \leq b$ or $b \leq 0$), we have sufficiently proven this statement in this direction. 
\end{proof}

\begin{proof}{\textbf{a.) $\leftarrow$}}
\\Assume that $-a \leq b \leq a$, i.e. $-a \leq b$ (I will call this the left inequality) and $b \leq a$ (I will call this the right inequality). Again, we will consider 2 cases $0 \leq b$ and $b \leq 0$, by O1 this constitutes all possible cases.
\\\underline{Case 1: $0 \leq b$}
\\If $0 \leq b$, then we know that $|b| = b$, so the given (right) inequality can be stated as $|b| \leq a$, proving the statement for this case.
\\\underline{Case 2: $b \leq 0$}
\\If $b \leq 0$, then we know that $|b| = -b$. First, I want to show that $-|b| = b$. First, $|b| + (-|b|) = 0$ by A4. Also,
\begin{align*}
b + |b| &= b + (-b) &\text{by the given relationship above}\\
&= 0 &\text{by A4}\\
\implies |b| + (-|b|) &= b + |b| &\text{since we've shown that the LHS and RHS are equal to 0}\\
\implies -|b| + |b| &= b + |b| &\text{by A2}\\
\implies -|b| &= b &\text{by Theorem 3.1 (i)}
\end{align*}
From this, the given (left) inequality can be stated as $-a \leq -|b|$. From Theorem 3.2 (i) it follows that $-(-|b|) \leq -(-a)$. We have already shown above that $-(-c) = c \quad \forall \; c$ in the previous proof. Thus, this last inequality is equivalent to $|b| \leq a$ which proves the statement for this case.
\\Since we've proven the statement for all possible cases, we have sufficiently proven the statement in this direction. 
\\Furthermore, since we've proven the statement in both directions, we have proven the statement in its entirety. 
\end{proof}

\begin{question*}{\textbf{b.)}}
\\Prove $||a| - |b|| \leq |a - b|$ for all $a, b \in \mathbb{R}$.
\end{question*}

\begin{proof}{\textbf{b.)}}
\\From part (a) of this question, it suffices to show that $-|a-b| \leq |a| - |b| \leq |a - b|$. In other words, $-|a - b| \leq |a| - |b|$ and $|a| - |b| \leq |a-b|$. I will start with the first of these inequalities:
\begin{align}
|b| &= |b + 0| &\text{by A3}\nonumber\\
&= |b + (a + (-a))| &\text{by A4}\nonumber\\
&= |b + (-a + a)| &\text{by A2}\nonumber\\
&= |(b + (-a)) + a| &\text{by A1}\nonumber\\
&\leq |b + (-a)| + |a| &\text{by Triangle Inequality}\nonumber\\
\implies |b| + (-|a|) &\leq (|b + (-a)| + |a|) + (-|a|) &\text{by O4}\nonumber\\
&= |b + (-a)| + (|a| + (-|a|)) &\text{by A1}\nonumber\\
&= |b + (-a)| + 0 &\text{by A4}\nonumber\\
\implies -|a| + |b| &\leq |b + (-a)| &\text{by A2 on the left and A3 on the right}\nonumber\\
\implies -|b + (-a)| &\leq -(-|a| + |b|) &\text{by Theorem 3.2 (i)}
\end{align}
Before I proceed, I will need to show that $-c = (-1)c$ for all $c$. We know that $c + (-c) = 0$ because of A4. Also, 
\begin{align*}
c + (-1)c &= c(1) + (-1)c &\text{by M3}\\
&= c(1) + c(-1) &\text{by M2}\\
&= c(1 + (-1)) &\text{by DL}\\
&= c \cdot 0 &\text{by A4}\\
&= 0 &\text{by Theorem 3.1 (ii)}\\
\implies c + (-c) &= c + (-1)c &\text{since LHS and RHS both equal 0}\\
\implies (-c) + c &= (-1)c + c &\text{by A2 on both sides}\\
\implies (-c) &= (-1)c &\text{by Theorem 3.1 (i)}
\end{align*}
Now, using this result on line (17) gives:
\begin{align*}
-|b + (-a)| &\leq (-1)(-|a| + |b|)\\
&= (-1)(-|a|) + (-1)(|b|) &\text{by DL}\\
&= 1\cdot |a| + (-(1\cdot |b|)) &\text{by Theorem 3.1 (iii) and (iv)}\\
&= |a| \cdot 1 + -(|b| \cdot 1) &\text{by M2}\\
&= |a| + -|b| &\text{by M3}
\end{align*}
The last thing I need to show is that $|b + (-a)| = |a + (-b)|$:
\begin{align*}
|b + (-a)| &= |(b + (-a)) \cdot 1| &\text{by M3}\\
&= |(b + (-a)) \cdot (1 \cdot 1)| &\text{by M3}\\
&= |(b + (-a)) \cdot [(-1)(-1)]| &\text{by Theorem 3.1 (iv)}\\
&= |[(b + (-a)) \cdot (-1)] \cdot (-1)| &\text{by M1}\\
&= |[(-1)(b + (-a))] \cdot (-1)| &\text{by M2}\\
&= |((-1)b + (-1)(-a)) \cdot (-1)| &\text{by DL}\\
&= |(-b + a)(-1)| &\text{by Theorem 3.1 (iii) and (iv)}\\
&= |a + (-b)|\cdot|-1| &\text{by A2 and Theorem 3.5 (ii)}\\
&= |a + (-b)| \cdot 1 &\text{since -1 $\leq$ 0 by Theorem 3.2 (v) and (i)}\\
&= |a + (-b)| &\text{by M3}
\end{align*}
Therefore, if we use the convention that $c + (-d) = c - d$, we get that $-|a-b| \leq |a| - |b|$ from applying the above results, which is the first inequality. Similarly, to get the second inequality I will do the following:
\begin{align*}
|a| &= |a + 0| &\text{by A3}\\
&= |a + (b + (-b))| &\text{by A4}\\
&= |a + (-b + b)| &\text{by A2}\\
&= |(a + (-b)) + b| &\text{by A1}\\
&\leq |a + (-b)| + |b| &\text{by Triangle Inequality}\\
\implies |a| + (-|b|) &\leq (|a + (-b)| + |b|) + (-|b|) &\text{by O4}\\
&= |a + (-b)| + (|b| + (-|b|)) &\text{by A1}\\
&= |a + (-b)| + 0 &\text{by A4}\\
&= |a + (-b)| &\text{by A3}
\end{align*}
Thus, if we use the convention that $c + (-d) = c - d$, then we get that $|a| - |b| \leq |a - b|$, which is the second inequality. With both inequalities in place, we have shown that $-|a-b| \leq |a| - |b| \leq |a - b|$, which is equivalent to $||a| - |b|| \leq |a - b|$ by part (a) of this question, giving the desired inequality
\end{proof}

\section*{3.6}
\begin{question*}{\textbf{a.)}}
\\Prove $|a + b + c| \leq |a| + |b| + |c|$ for all $a,b,c \in \mathbb{R}$.
\end{question*}

\begin{proof}{\textbf{a.)}}
\begin{align*}
|a + b + c| &= |(a + b) + c| &\text{by A1}\\
&\leq |a + b| + |c| &\text{by Triangle Inequality}\\
&\leq |a| + |b| + |c| &\text{by Triangle Inequality}
\end{align*}
Thus, we have the desired result.
\end{proof}

\begin{question*}{\textbf{b.)}}
\\Use induction to prove
\[P_n: |a_1 + a_2 + \cdots + a_n| \leq |a_1| + |a_2| + \cdots + |a_n|\]
is true for any $n$ numbers $a_1, a_2, \ldots, a_n$, for all natural numbers $n$.
\end{question*}

\begin{proof}{\textbf{b.)}}
\\\underline{Base case: $n = 1$}
\\For $n = 1$, $P_n$ says that $|a_1| \leq |a_1|$ which is obviously true by O1. 
\\\underline{Inductive step}
\\Assume that $P_k$ is true for some $k \in \mathbb{N}$, i.e. $|a_1 + a_2 + \cdots + a_k| \leq |a_1| + |a_2| + \cdots + |a_k|$. We wish to show that $P_{k+1}$ is also true, i.e., we want to show that $|a_1 + a_2 + \cdots + a_k + a_{k+1}| \leq |a_1| + |a_2| + \cdots + |a_k| + |a_{k+1}|$. We will start with the LHS of the previous inequality and arrive at the desired result:

\begin{align*}
|a_1 + a_2 + \cdots + a_k + a_{k+1}| &= |(a_1 + a_2 + \cdots + a_k) + a_{k+1}| &\text{by A1}\\
&\leq |a_1 + a_2 + \cdots + a_k| + |a_{k+1}| &\text{by Triangle Inequality}\\
&\leq |a_1| + |a_2| + \cdots + |a_k| + |a_{k+1}| &\text{by the Inductive Hypothesis}
\end{align*}
Thus, we have shown that if $P_k$ is true, then $P_{k+1}$ must also be true. Additionally, since $P_1$ is true, then by the Principle of Mathematical Induction, we have that $P_n$ must be true for all $n \in \mathbb{N}$.
\end{proof}


\section*{3.7}
\begin{question*}{\textbf{a.)}}
\\Show $|b| < a$ if and only if $-a<b<a$.
\end{question*}
{$ $}
\\This statement is equivalent to "$|b| < a$ if and only if $-a < b$ and $b < a$." Which is, in turn, equivalent to "($|b| \leq a$ and $|b| \neq a$) if and only if ($-a \leq b$ and $-a \neq b$, and $b \leq a$ and $b \neq a$)."
\begin{proof}{\textbf{a.) $\rightarrow$}}
\\Assume that $|b| \leq a$ and $|b| \neq a$. From Question 3.5 (a), we know that this implies $-a \leq b \leq a$, which means $-a \leq b$ and $b \leq a$. Thus, to prove the statement, we only need to show that $-a \neq b$ and $b \neq a$. I will consider 3 cases, $b = 0$, $0 < b$, and $b < 0$. I will also use the fact that we showed in Question 3.5 (a) that $0 \leq a$. 
\\\underline{Case 1: $b= 0$}
\\If $b = 0$, then $0 \neq a$ since $|0| = 0$. If we assume $-a = 0$, then we can conclude that $a - a = a$ by adding $a$ to both sides, but this gives us $0 = a$ by A4, which contradicts that we previously said $0 \neq a$; thus it is not true that $-a = 0$, so $-a \neq 0$. Thus, the statement holds for this case. 
{$ $}
\\\underline{Case 2: $0 < b$}
\\If $0 < b$, then we know that $|b| = b$, so we immediately get that $b \neq a$. We also cannot have $-a = b$ since $0 \leq a$ implies that $-a \leq 0$ by Theorem 3.2 (i) and $0 < b$ means the two can never be equal; thus $-a \neq b$. This proves the statement for this case. 
{$ $}
\\\underline{Case 3: $b < 0$}
\\If $b < 0$, then we know that $|b| = -b$, so we immediately get that $-b \neq a$. This implies $-a \neq b$ since $-a = b$ if and only if $a = -b$ by multiplying both sides by $(-1)$ and using the fact that we proved earlier that $(-1)c = -c$ and $-(-c) = c$ for all $c$. Since $a = -b$ if false, we know that $-a = b$ is false, so $-a \neq b$. Furthermore, since $0 \leq a$ and $b < 0$, the two can never be equal to one another, so $b \neq a$. This proves the statement for this case. 
\\Thus, the statement is proven in this direction. 
\end{proof}

\begin{proof}{\textbf{a.) $\leftarrow$}}
\\Now we assume that $-a<b<a$, i.e. $-a \leq b$, $b\leq a$, $-a \neq b$, and $b\neq a$. We already know that this implies $|b| \leq a$ based off of Question 3.5 (a). We only need to show that $|b| \neq a$. Assume that $|b| = a$. Thus, if $0 \leq b$, we get that $b = a$ which is a contradiction to the fact that $b \neq a$. Next, if $b \leq 0$, we know $-b = a$. However, we already showed above that this would imply that $-a = b$, but that is also a contradiction. Therefore, our assumption that $|b| = a$ must have been false. Thus, $|b| \neq a$ just as we wanted. 
\\Since we have shown the statement is true in both directions, we have proved it in its entirety. 
\end{proof}

\begin{question*}{\textbf{b.)}}
\\Show $|a-b| < c$ if and only if $b-c < a < b + c$.
\end{question*}

\begin{proof}{\textbf{b.) $\rightarrow$}}
\\Assume that $|a-b| < c$. We know from part (a) of this question that this implies $-c < a-b < c$, i.e. $-c < a-b$ and $a-b < c$. I will handle these 2 inequalities separately:
\begin{align*}
-c &< a-b\\
\iff -c + b &< (a-b) + b &\text{by O4}\\
\iff b - c &< a + (-b + b) &\text{by A2 on the left and A1 on the right}\\
\iff b - c &< a + (b - b) &\text{by A2}\\
\iff b - c &< a + 0 &\text{by A4}\\
\iff b - c &< a &\text{by A3}
\end{align*}
Thus, $b - c < a$ like desired. For the second inequality we have:
\begin{align*}
a - b &< c\\
\iff (a - b) + b &< b + c &\text{by O4}\\
\iff a + (-b + b) &< b + c &\text{by A1}\\
\iff a + (b - b) &< b + c &\text{by A2}\\
\iff a + 0 &< b + c &\text{by A4}\\
\iff a &< b + c &\text{by A3}
\end{align*}
Thus, we have $a < b + c$ as desired. Together, we have $b - c < a < b + c$ which proves the statement in this direction. 
\end{proof}

\begin{proof}{\textbf{b.) $\leftarrow$}}
\\Assume that $b - c < a < b + c$, i.e. $b-c < a$ and $a < b + c$. I will handle these inequalities separately. However, since all of the steps used in the proof in the previous direction are valid in both direction, I can use both of those to show that $b-c < a \implies -c < a- b$ and $a < b + c \implies a -b < c$. Altogether, this gives $-c < a- b < c$ which implies that $|a - b| < c$ by part (a) of this question.
\\Therefore, we have shown both direction of this statement are true, so we have proven it in its entirety. 
\end{proof}

\begin{question*}{\textbf{c.)}}
\\Show $|a-b| \leq c$ if and only if $b-c \leq a \leq b +c$.
\end{question*}

\begin{proof}{\textbf{c.) $\rightarrow$}}
\\Assume that $|a - b| \leq c$. From Question 3.5 (a), we can conclude that $-c \leq a - b \leq c$, i.e. $-c \leq a - b$ and $a- b \leq c$. Using the exact same steps used in part (b)$\rightarrow$ of this question, we can arrive at the result that $-c \leq a-b \implies b-c \leq a$ and $a -b \leq c \implies a \leq b + c$. Altogether, this gives $b - c \leq a \leq b +c$, precisely what we wanted to show.
\end{proof}

\begin{proof}{\textbf{c.) $\leftarrow$}}
\\Assume that $b - c \leq a \leq b +c$, i.e. $b - c \leq a$ and $a \leq b + c$. Since all of the steps used in part (b)$\rightarrow$ of this question (and by necessity all of the steps used in part (c) $\rightarrow$) are valid in both directions, we can use those same steps in reverse to show that $b - c \leq a \implies -c \leq a - b$ and $a \leq b + c \implies a - b \leq c$. Together, this gives the result that $ -c \leq a -b \leq c$. Thus, by Question 3.5 (a), we can say that $|a -b| \leq c$, proving the statement in this direction. With both directions proven, the statement in its entirety is proven. 
\end{proof}

\end{document}