\documentclass[10pt,a4paper]{article}
\usepackage[utf8]{inputenc}
\usepackage[english]{babel}
\usepackage{csquotes}
\usepackage{amsmath}
\usepackage{amsfonts}
\usepackage{amssymb}
\usepackage{graphicx}
\usepackage[margin=0.5in]{geometry}
\usepackage{amsthm}
\usepackage{enumitem}
\usepackage{tikz}
\usetikzlibrary{calc}
\newtheorem{question}{Question}
\newtheorem*{question*}{Question}
\newtheorem{theorem}{Theorem}
\newtheorem*{theorem*}{Theorem}
\newtheorem{lemma}{Lemma}

\theoremstyle{definition}
\newtheorem{answer}{Answer}
\newtheorem*{answer*}{Answer}


\title{Advanced Calc. Homework 8}
\author{Colin Williams}

\begin{document}
\maketitle

\section*{14.1}
Determine which of the following series converge. Justify your answers.
\begin{enumerate}[label = (\alph*)]
\item $\displaystyle \sum \frac{n^4}{2^n}$
	\begin{itemize}
	\item Using the ratio test, 
	\begin{align*}
	\left|\frac{a_{n+1}}{a_n}\right| = \left|\frac{(n+1)^4}{2^{n+1}} \cdot \frac{2^n}{n^4}\right| &= \frac{1}{2}\left|\frac{(n + 1)^4}{n^4}\right|\\
	\implies \lim_{n \to \infty} \left|\frac{a_{n+1}}{a_n}\right| &= \lim_{n \to \infty} \frac{1}{2}\left(\frac{n + 1}{n}\right)^4\\
	&= \frac{1}{2}\lim_{n \to \infty}\left(1 + \frac{1}{n}\right)^4\\
	&= \frac{1}{2}\cdot 1^4 = \frac{1}{2} < 1
	\end{align*}
	\item Thus, \boxed{\text{the series converges due to the ratio test}}
	\end{itemize}
\item $\displaystyle \sum \frac{2^n}{n!}$
	\begin{itemize}
	\item Using the ratio test,
	\begin{align*}
	\left|\frac{a_{n+1}}{a_n}\right| = \left|\frac{2^{n+1}}{(n+1)!} \cdot \frac{n!}{2^n}\right| &= \left|\frac{2}{n + 1}\right|\\
	\implies \lim_{n \to \infty}\left|\frac{a_{n+1}}{a_n}\right| &= \lim_{n \to \infty} 2 \cdot \frac{1}{n+1}\\
	&= 2 \cdot \lim_{n \to \infty} \frac{1}{n+1}\\
	&= 2 \cdot 0 = 0 < 1
	\end{align*}
	\item Thus, \boxed{\text{the series converges due to the ratio test}}
	\end{itemize}
\item $\displaystyle \sum \frac{n^2}{3^n}$
	\begin{itemize}
	\item Using the ratio test,
	\begin{align*}
	\left|\frac{a_{n+1}}{a_n}\right| = \left|\frac{(n+1)^2}{3^{n+1}} \cdot \frac{3^n}{n^2}\right| &= \frac{1}{3}\left|\frac{(n+1)^2}{n^2}\right|\\
	\implies \lim_{n \to \infty} \left|\frac{a_{n+1}}{a_n}\right| &= \lim_{n \to \infty} \frac{1}{3} \left(\frac{n + 1}{n}\right)^2\\
	&= \frac{1}{3} \cdot \lim_{n \to \infty} \left(1 + \frac{1}{n}\right)^2\\
	&= \frac{1}{3} \cdot 1^2 = \frac{1}{3} < 1
	\end{align*}
	\item Thus, \boxed{\text{the series converges due to the ratio test}}
	\end{itemize}
\item $\displaystyle \sum \frac{n!}{n^4 + 3}$
	\begin{itemize}
	\item Using the ratio test,
	\begin{align*}
	\left|\frac{a_{n+1}}{a_n}\right| = \left|\frac{(n+1)!}{(n+1)^4 + 3} \cdot \frac{n^4 + 3}{n!}\right| = \left|\frac{(n + 1)(n^4 + 3)}{(n + 1)^4 + 3}\right| &= \left|\frac{n^5 + n^4 + 3n + 3}{n^4 + 4n^3 + 6n^2 + 4n + 4}\right|\\
	\implies \lim_{n \to \infty} \left|\frac{a_{n+1}}{a_n}\right| &= \lim_{n \to \infty} \frac{n^5 + n^4 + 3n + 3}{n^4 + 4n^3 + 6n^2 + 4n + 4} \cdot \frac{\frac{1}{n^5}}{\frac{1}{n^5}}\\
	&= \lim_{n \to \infty} \frac{1 + \frac{1}{n} + \frac{3}{n^4} + \frac{3}{n^5}}{\frac{1}{n} + \frac{4}{n^2} + \frac{6}{n^3} + \frac{4}{n^4} + \frac{4}{n^5}}\\
	&= \infty \quad \text{since we are of the form $\frac{1}{0}$}\\
	&\not< 1
	\end{align*}
	\item Thus, \boxed{\text{the series diverges due to the ratio test}}
	\end{itemize}
\item $\displaystyle \sum \frac{\cos^2(n)}{n^2}$
	\begin{itemize}
	\item Consider $|a_n|$, then we have:
	\begin{align*}
	|a_n| = \left|\frac{\cos^2(n)}{n^2}\right| = \frac{|\cos(n)|^2}{|n|^2} = \frac{|\cos(n)|^2}{n^2} \leq \frac{1^2}{n^2} = \frac{1}{n^2} &&\text{for all $n$ since $\cos(n) \leq 1$ for all $n$.}
	\end{align*}
	\item Furthermore, $\displaystyle \sum \frac{1}{n^2}$ converges by the $p$-series test (example 2 pg 96-97), so \\\boxed{\text{the series converges by the comparison test with } \frac{1}{n^2}}
	\end{itemize}
\item $\displaystyle \sum_{n = 2}^{\infty} \frac{1}{\log(n)}$
	\begin{itemize}
	\item Consider $|a_n|$, then we have:
	\begin{align*}
	|a_n| = \left|\frac{1}{\log(n)}\right| &= \frac{1}{\log(n)} &\text{since $\log(n) > 0$ for all $n \geq 2$}\\
	&> \frac{1}{n} &\text{since $\log(n) < n$ for all $n \geq 2$}
	\end{align*}
	\item However, $\displaystyle \sum \frac{1}{n}$ diverges according to the $p$-series test (example 2 pg 96-97), so \\\boxed{\text{the series diverges by the comparison test with } \frac{1}{n}}
	\end{itemize}
\end{enumerate}

\section*{14.2}
Determine which of the following series converge. Justify your answers.
\begin{enumerate}[label = (\alph*)]
\item $\displaystyle \sum \frac{n - 1}{n^2}$
	\begin{itemize}
	\item Note that $\displaystyle \sum \frac{n - 1}{n^2} = \sum \left(\frac{n}{n^2}- \frac{1}{n^2}\right) = \sum \frac{1}{n} - \sum \frac{1}{n^2}$
	\item We know that the second sum converges, say to $L \in \mathbb{R}$. Then we are left with a divergent sum minus $L$, so we can conclude that the original sum \boxed{\text{diverges due to the divergence of } \sum \frac{1}{n}}
	\end{itemize}
\item $\displaystyle \sum (-1)^n$
	\begin{itemize}
	\item We require for $\sum a_n$ to converge, that $\lim (a_n) = 0$. However, in this case for $a_n = (-1)^n$, $\lim \inf (a_n) = -1 \neq 1 = \lim \sup(a_n)$. Therefore, $\lim (a_n)$ does not exist, so \boxed{\text{the series does not converge since }\lim(-1)^n \neq 0}
	\end{itemize}
\item $\displaystyle \sum \frac{3n}{n^3}$
	\begin{itemize}
	\item First, note that $\displaystyle \frac{3n}{n^3} = \frac{3}{n^2}$ for all $n$. 
	\item Thus, $\displaystyle \sum \frac{3n}{n^3} = \sum \frac{3}{n^2} = \left(\frac{3}{1} + \frac{3}{4} + \frac{3}{9} + \cdots\right) = 3\left(\frac{1}{1} + \frac{1}{4} + \frac{1}{9} + \cdots\right) = 3 \cdot \sum \frac{1}{n^2}$
	\item We know that $\displaystyle \sum \frac{1}{n^2}$ converges due to $p$-series test (example 2 pg 96-97), so in other words $\displaystyle \sum \frac{1}{n^2} = L \in \mathbb{R}$; thus, the series we're interested in converges to $3L \in \mathbb{R}$, so \boxed{\text{the series converges due to the convergence of } \sum \frac{1}{n^2}}
	\end{itemize}
\item $\displaystyle \sum \frac{n^3}{3^n}$
	\begin{itemize}
	\item Using the ratio test, 
	\begin{align*}
	\left|\frac{a_{n+1}}{a_n}\right| = \left|\frac{(n+1)^3}{3^{n+1}} \cdot \frac{3^n}{n^3}\right| &= \frac{1}{3}\left|\frac{(n + 1)^3}{n^3}\right|\\
	\implies \lim_{n \to \infty} \left|\frac{a_{n+1}}{a_n}\right| &= \lim_{n \to \infty} \frac{1}{3}\left(\frac{n + 1}{n}\right)^3\\
	&= \lim_{n \to \infty}\frac{1}{3}\left(1 + \frac{1}{n}\right)^3\\
	&= \frac{1}{3} \cdot 1^3 = \frac{1}{3} < 1
	\end{align*}
	\item Thus, \boxed{\text{the series converges by the ratio test}}
	\end{itemize}
\item $\displaystyle \sum \frac{n^2}{n!}$
	\begin{itemize}
	\item Using the ratio test, 
	\begin{align*}
	\left|\frac{a_{n+1}}{a_n}\right| = \left|\frac{(n+1)^2}{(n+1)!} \cdot \frac{n!}{n^2}\right| &= \left|\frac{1}{n + 1} \cdot \frac{(n + 1)^2}{n^2}\right|\\
	\implies \lim_{n \to \infty} \left|\frac{a_{n+1}}{a_n}\right| &= \lim_{n \to \infty} \left(\frac{n + 1}{n^2}\right)\\
	&= \lim_{n \to \infty} \left(\frac{\frac{1}{n} + \frac{1}{n^2}}{1}\right)\\
	&= 0 < 1
	\end{align*}
	\item Thus, \boxed{\text{the series converges by the ratio test}}
	\end{itemize}
\item $\displaystyle \sum\frac{1}{n^n}$
	\begin{itemize}
	\item Using the root test,
	\begin{align*}
	\sqrt[n]{|a_n|} = \sqrt[n]{\frac{1}{n^n}} &= \frac{1}{n}\\
	\implies \lim_{n \to \infty} \sqrt[n]{|a_n|} &= \lim_{n \to \infty} \frac{1}{n}\\
	&= 0 < 1
	\end{align*}
	\item Thus, \boxed{\text{the series converges by the root test}}
	\end{itemize}
\item $\displaystyle \sum \frac{n}{2^n}$
	\begin{itemize}
	\item Using the ratio test,
	\begin{align*}
	\left|\frac{a_{n+1}}{a_n}\right| = \left|\frac{n + 1}{2^{n+1}} \cdot \frac{2^n}{n}\right| &= \frac{1}{2}\left|\frac{n + 1}{n}\right|\\
	\implies \lim_{n \to \infty} \left|\frac{a_{n+1}}{a_n}\right| &= \lim_{n \to \infty} \frac{1}{2}\left(1 + \frac{1}{n}\right)\\
	&= \frac{1}{2} \cdot 1 = \frac{1}{2} < 1
	\end{align*}
	\item Thus, \boxed{\text{the series converges by the ratio test}}
	\end{itemize}
\end{enumerate}

\section*{14.3}
Determine which of the following series converge. Justify your answers.
\begin{enumerate}[label = (\alph*)]
\item $\displaystyle \sum \frac{1}{\sqrt{n!}}$
	\begin{itemize}
	\item Using the ratio test,
	\begin{align*}
	\left|\frac{a_{n+1}}{a_n}\right| = \left|\frac{\sqrt{n!}}{\sqrt{(n+1)!}}\right| &= \left|\frac{\sqrt{n!}}{\sqrt{n + 1}\sqrt{n!}}\right|\\
	\implies \lim_{n \to \infty} \left|\frac{a_{n+1}}{a_n}\right| &= \lim_{n \to \infty} \frac{1}{\sqrt{n + 1}}\\
	&= 0 < 1
	\end{align*}
	\item Thus, \boxed{\text{the series converges by the ratio test}}
	\end{itemize}
\item $\displaystyle \sum \frac{2 + \cos(n)}{3^n}$
	\begin{itemize}
	\item Let's consider $|a_n|$,
	\begin{align*}
	|a_n| = \left|\frac{2 + \cos(n)}{3^n}\right| = \frac{|2 + \cos(n)|}{|3^n|} &\leq \frac{|2| + |\cos(n)|}{3^n} &\text{by the Triangle Inequality}\\
	&\leq \frac{2 + 1}{3^n} &\text{since $\cos(n) \leq 1$ for all $n$}\\
	&= \frac{1}{3^{n-1}}
	\end{align*}
	\item Furthermore, we know that $\displaystyle \sum \frac{1}{3^{n-1}}$ converges as a geometric series; thus, \\\boxed{\text{the series converges by the comparison test with }\frac{1}{3^{n-1}}}
	\end{itemize}
\item $\displaystyle \sum \frac{1}{2^n + n}$
	\begin{itemize}
	\item Let's consider $|a_n|$,
	\begin{align*}
	|a_n| = \left|\frac{1}{2^n + n}\right| = \frac{1}{|2^n + n|} = \frac{1}{2^n + n} &\leq \frac{1}{2^n} &\text{since $2^n + n \geq 2^n$ for all $n \in \mathbb{N}$}\\
	\end{align*}
	\item Furthermore, we know that $\displaystyle \sum \frac{1}{2^n}$ converges as a geometric series; thus, \\\boxed{\text{the series converges by the comparison test with }\frac{1}{2^n}}
	\end{itemize}
\item $\displaystyle \sum \left(\frac{1}{2}\right)^n\left(50 + \frac{2}{n}\right)$
	\begin{itemize}
	\item Using the ratio test,
	\begin{align*}
	\left|\frac{a_{n+1}}{a_n}\right| = \left|\frac{\displaystyle \left(\frac{1}{2}\right)^{n+1}\left(50 + \frac{2}{n+1}\right)}{\displaystyle \left(\frac{1}{2}\right)^{n}\left(50 + \frac{2}{n}\right)}\right| &= \frac{1}{2}\left(\frac{50 + \frac{2}{n+1}}{50 + \frac{2}{n}}\right)\\
	\implies \lim_{n \to \infty} \left|\frac{a_{n+1}}{a_n}\right| &= \frac{1}{2}\lim_{n \to \infty}\left(\frac{50 + \frac{2}{n+1}}{50 + \frac{2}{n}}\right)\\
	&= \frac{1}{2}\frac{\lim_{n \to \infty}(50 + \frac{2}{n + 1})}{\lim_{n \to \infty}(50 + \frac{2}{n})}\\
	&= \frac{1}{2} \cdot \frac{50}{50} = \frac{1}{2} < 1
	\end{align*}
	\item Thus, \boxed{\text{the series converges by the ratio test}}
	\end{itemize}
\item $\displaystyle \sum \sin\left(\frac{n\pi}{9}\right)$
	\begin{itemize}
	\item We require for $\sum a_n$ to converge that $\lim(a_n) = 0$. However, in this case for $a_n = \sin\left(\frac{n\pi}{9}\right)$, we have $\lim \inf(a_n) = \sin\left(\frac{-4\pi}{9}\right) \approx -0.9848 \neq 0.9848 \approx \sin\left(\frac{4\pi}{9}\right) = \lim \sup (a_n)$. Therefore, $\lim(a_n)$ does not exist, so \boxed{\text{the series does not converge since }\lim\left(\sin\left(\frac{n\pi}{9}\right)\right) \neq 0}
	\end{itemize}
\item $\displaystyle \sum\frac{100^n}{n!}$
	\begin{itemize}
	\item Using the ratio test,
	\begin{align*}
	\left|\frac{a_{n+1}}{a_n}\right| = \left|\frac{100^{n + 1}}{(n+1)!} \cdot \frac{n!}{100^n}\right| &= 100 \cdot \left|\frac{1}{n + 1}\right|\\
	\implies \lim_{n \to \infty} \left|\frac{a_{n+1}}{a_n}\right| &= 100 \lim_{n \to \infty} \frac{1}{n + 1}\\
	&= 100 \cdot 0 = 0 < 1
	\end{align*}
	\item Thus, \boxed{\text{this series converges due to the ratio test}}
	\end{itemize}
\end{enumerate}

\section*{14.4}
Determine which of the following series converge. Justify your answers.
\begin{enumerate}[label = (\alph*)]
\item $\displaystyle \sum_{n = 2}^{\infty} \frac{1}{[n + (-1)^n]^2}$
	\begin{itemize}
	\item Let's consider $|a_n|$,
	\begin{align*}
	|a_n| = \left|\frac{1}{[n + (-1)^n]^2}\right| = \frac{1}{[n + (-1)^n]^2} &\leq \frac{1}{(n - 1)^2} &\text{since $n + (-1)^n \geq n - 1$ for all $n \geq 2$}
	\end{align*}
	\item We also know that $\displaystyle \sum_{n = 2}^{\infty} \frac{1}{(n - 1)^2} = \sum_{m = 1}^{\infty} \frac{1}{m^2}$ and we know that this second series is a convergent $p$-series; thus, the first series must also converge as well. Using this, we can conclude that \\\boxed{\text{the original series must converge by the comparison test with } \frac{1}{(n-1)^2}}
	\end{itemize}
\item $\displaystyle \sum \left[\sqrt{n + 1} - \sqrt{n}\right]$
	\begin{itemize}
	\item First, note that $\displaystyle \left[\sqrt{n + 1} - \sqrt{n}\right] = \left[\sqrt{n + 1} - \sqrt{n}\right] \cdot \frac{\sqrt{n + 1} + \sqrt{n}}{\sqrt{n + 1} + \sqrt{n}} = \frac{n + 1 + \sqrt{n}\sqrt{n + 1} - \sqrt{n}\sqrt{n+1} - n}{\sqrt{n + 1} + \sqrt{n}} = \frac{1}{\sqrt{n + 1} + \sqrt{n}}$.
	\item Thus, if we analyze $|a_n|$, 
	\begin{align*}
	|a_n| = \left|\sqrt{n + 1} - \sqrt{n}\right| = \left|\frac{1}{\sqrt{n + 1} + \sqrt{n}}\right| &= \frac{1}{\sqrt{n + 1} + \sqrt{n}}\\
	&> \frac{1}{2\sqrt{n + 1}} &\text{since $\sqrt{n} < \sqrt{n +1}$ for all $n \in \mathbb{N}$}\\
	&\geq \frac{1}{2\sqrt{2n}} &\text{since $n + 1 \leq 2n$ for all $n \in \mathbb{N}$}\\
	&= \frac{1}{2\sqrt{2}} \cdot \frac{1}{\sqrt{n}}
	\end{align*}
	\item However, we know that $\sum \frac{1}{\sqrt{n}}$ diverges due to the $p$-series test, so $\sum \frac{1}{2\sqrt{2}}\frac{1}{\sqrt{n}}$ diverges as well. Thus, \\\boxed{\text{this series diverges due to comparison test with } \frac{1}{2\sqrt{2n}}}
	\end{itemize}
\item $\displaystyle \sum \frac{n!}{n^n}$
	\begin{itemize}
	\item Using the ratio test,
	\begin{align*}
	\left|\frac{a_{n+1}}{a_n}\right| = \left|\frac{(n + 1)!}{(n+1)^{n+1}} \cdot \frac{n^n}{n!}\right| = \left|(n+1) \cdot \frac{n^n}{(n+1)^{n+1}}\right| = \left|\frac{n^n}{(n+1)^n}\right| &= \left(\frac{n}{n + 1}\right)^n\\
	\implies \lim_{n \to \infty} \left|\frac{a_{n+1}}{a_n}\right| &= \lim_{n \to \infty} \left(\frac{n}{n + 1}\right)^n\\
	&= \frac{1}{\displaystyle \lim_{n \to \infty} \frac{1}{(n/(n+1))^n}} &\text{by Lemma 9.5}\\
	&= \frac{1}{\displaystyle \lim_{n \to \infty} \left( \frac{n + 1}{n}\right)^n}\\
	&= \frac{1}{\displaystyle \lim_{n \to \infty} \left(1 + \frac{1}{n}\right)^n}\\
	&= \frac{1}{e}
	\end{align*}
	\item This last equality here follows from the discussion on Example 3(e) on page 37 of section 7. Therefore, since $\frac{1}{e} \approx 0.3679 < 1$, then \boxed{\text{this series converges due to the ratio test}}
	\end{itemize}
\end{enumerate}

\section*{14.5}
Suppose $\sum a_n = A$ and $\sum b_n = B$ where $A$ and $B$ are real numbers. Use limit theorems to quickly prove the following:
\begin{enumerate}[label = (\alph*)]
\item $\displaystyle \sum(a_n + b_n) = A + B$
\item $\displaystyle \sum k a_n = kA$ for $k \in \mathbb{R}$
\item Is $\displaystyle \sum a_nb_n = AB$ a reasonable conjecture? Discuss.
\end{enumerate}

\begin{proof}{\textbf{(a)}}
\\I will start by examining the series on the left and assuming their indices start at some arbitrary point $m$:
\begin{align*}
\sum_{n = m}^{\infty} (a_n + b_n) &= \lim_{N \to \infty} \sum_{n = m}^{N}(a_n + b_n)\\
&= \lim_{N \to \infty}\left[(a_m + b_m) + (a_{m+1} + b_{m+1}) + \cdots + (a_N + b_N)\right]\\
&= \lim_{N \to \infty} \left[(a_m + a_{m+1} + \cdots + a_N) + (b_m + b_{m+1} + \cdots + b_N)\right]\\
&= \lim_{N \to \infty}\left[(a_m + a_{m+1} + \cdots + a_N)\right] + \lim_{N \to \infty}\left[(b_m + b_{m+1} + \cdots + b_N)\right] &\text{by Theorem 9.3}\\
&= \lim_{N \to \infty} \sum_{n = m}^{N} a_n + \lim_{N \to \infty} \sum_{n = m}^{N} b_n\\
&= \sum_{n = m}^{\infty} a_n + \sum_{n = m}^{\infty} b_n\\
&= A + B &\text{by the given information}
\end{align*}
\end{proof}

\begin{proof}{\textbf{(b)}}
\\I will start by examining the series on the left and assuming the index starts at some arbitrary point $m$:
\begin{align*}
\sum_{n = m}^{\infty} ka_n &= \lim_{N \to \infty} \sum_{n = m}^{N}ka_n\\
&= \lim_{N \to \infty}\left[ka_m + ka_{m + 1} + \cdots + ka_N \right]\\
&= \lim_{N \to \infty} \left[k(a_m + a_{m+1} + \cdots + a_N)\right]\\
&= k \cdot \lim_{N \to \infty}\left[(a_m + a_{m+1} + \cdots + a_N)\right] &\text{by Theorem 9.2}\\
&= k \cdot \lim_{N \to \infty} \sum_{n = m}^{N} a_n\\
&= k \cdot \sum_{n = m}^{\infty} a_n\\
&= kA &\text{by the given information}
\end{align*}
\end{proof}

\begin{answer*}{\textbf{(c)}}
\\This is not a reasonable conjecture. For example, this does not even hold for finite sums (e.g. for 2 terms) since $a_1b_1 + a_2b_2 \neq (a_1 + a_2)(b_1 + b_2)$.
\end{answer*}

\section*{14.6}
\begin{enumerate}[label = (\alph*)]
\item Prove that if $\sum |a_n|$ converges and $(b_n)$ is a bounded sequence, then $\sum a_n b_n$ converges.
\item Observe that Corollary 14.7  is a special case of part (a).
\end{enumerate}

\begin{proof}{\textbf{(a)}}
\\First, since $(b_n)$ is bounded, there exists some $M \in \mathbb{R}$ such that $b_n \leq M$ for all $n$. Next, I will prove the statement by showing that $\displaystyle \sum_{k = 1}^{n} a_kb_k$ is a Cauchy sequence. Let $n \geq m$, then we have:
\begin{align*}
\left|\sum_{k = 0}^n a_kb_k - \sum_{k = 0}^m a_kb_k \right| = \left|\sum_{k = m + 1}^n a_kb_k\right| &= |a_{m + 1}b_{m + 1} + a_{m +2}b_{m +2} + \cdots + a_nb_n|\\
&\leq |a_{m + 1}b_{m + 1}| + |a_{m +2}b_{m + 2}| + \cdots + |a_nb_n| &\text{by the Triangle Inequality}\\
&= |a_{m + 1}||b_{m + 1}| + |a_{m +2}||b_{m + 2}| + \cdots + |a_n||b_n|\\
&\leq |a_{m + 1}|\cdot M + |a_{m + 2}| \cdot M + \cdots + |a_n|\cdot M &\text{by $(b_n)$'s boundedness}\\
&= M \cdot \sum_{k = m + 1}^{n} |a_k|
\end{align*}
Furthermore, since $\sum |a_n|$ converges (i.e. is Cauchy), then for every $\varepsilon > 0$, there exists some $N$ such that $n, m > N$ implies $\displaystyle \left| \sum_{k = 0}^n |a_k| - \sum_{k=0}^m |a_k|\right| = \left|\sum_{k = m + 1}^n |a_k|\right| < \frac{\varepsilon}{M}$. Using this, we can conclude that:
\begin{align*}
\left|\sum_{k = 0}^n a_kb_k - \sum_{k = 0}^m a_kb_k \right| &\leq M \cdot \sum_{k = m + 1}^{n} |a_k|\\
&< M \cdot \frac{\varepsilon}{M} = \varepsilon.
\end{align*}
Thus, $\displaystyle \sum_{k = 1}^{n} a_kb_k$ is a Cauchy sequence, so it converges. 
\end{proof}

\begin{answer*}{\textbf{(b)}}
\\Indeed, Corollary 14.7 asserts that absolutely convergent sequences are convergent. This can follow immediately from part (a) if we let $(b_n) := 1$.
\end{answer*}

\section*{14.7}
Prove that if $\sum a_n$ is a convergent series of nonnegative numbers and $p > 1$, then $\sum a_n^p$ converges. 

\begin{proof}{$ $}
\\Corollary 14.5 tells us that if $\sum a_n$ converges, then $\lim(a_n) = 0$. Furthermore, this tells us that for any $\varepsilon > 0$, there exists some $N$ such that $|a_n| < \varepsilon$ for all $n > N$. In particular, there exists some $N$ such that $|a_n| < 1$ for all $n > N$. We can therefore make the following conclusions for all $n > N$: $|a_n^p| = |a_n||a_n^{p-1}| < |a_n^{p-1}| < |a_n^{p-2}| < \cdots < |a_n| \leq a_n$. In other words, $|a_n^p| < a_n$ for all $n > N$. Thus, by the comparison test, $\sum a_n^p$ converges. 
\end{proof}

\section*{14.8}
Show that if $\sum a_n$ and $\sum b_n$ are convergent series of nonnegative numbers, then $\sum \sqrt{a_n b_n}$ converges.

\begin{proof}{$ $}
\\Claim: $\sqrt{ab} \leq a + b$ for $a, b \in [0,\infty)$
\\\underline{Proof of claim:}
\\We know that $0 \leq a^2 + ab + b^2$ since $0 \leq a^2$, $0 \leq ab$, and $0 \leq b^2$. From this, we can add $ab$ to both sides to get $ab \leq a^2 + 2ab + b^2 = (a + b)^2$. Thus, taking square roots give $\sqrt{ab} \leq a + b$ the desired inequality.
\\
\\Thus, we can now conclude that $\sqrt{a_n b_n} \leq a_n + b_n$ for all $n$ since $(a_n), (b_n) \in [0,\infty)$ for all $n$. Furthermore, since we showed that $\sum (a_n + b_n)$ converges in question 14.5(a), then we can use the comparison test with that series to conclude that $\sum \sqrt{a_n b_n}$ also converges. 
\end{proof}

\section*{14.9}
The convergence of a series does not depend on any finite number of the terms, though of course the value of the limit does. More precisely, consider series $\sum a_n$ and $\sum b_n$ and suppose the set $\{n \in \mathbb{N} : a_n \neq b_n\}$ is finite. Then the series both converge or else they both diverge. Prove this.

\begin{proof}{$ $}
\\If we let $N_0 = \max\{n \in \mathbb{N} : a_n \neq b_n\}$, then the fact that that set is finite implies that $N_0 < \infty$. Thus, for $n \geq m > N_0$ we know that $a_k = b_k$ for all $k \geq m$. Thus, $\displaystyle \sum_{k = m}^n a_k = \sum_{k = m}^n b_k$. Furthermore, $\displaystyle \lim_{n \to \infty} \sum_{k = m}^n a_k = \lim_{n \to \infty} \sum_{k = m}^n b_k$. Since the convergence of a series does not depend on any finite number of terms, then $\displaystyle \sum a_n$ converges if and only if $\displaystyle \lim_{n \to \infty} \sum_{k = m}^n a_k$ converges and similarly $\displaystyle \sum b_n$ converges if and only if $\displaystyle \lim_{n \to \infty} \sum_{k = m}^n b_k$ converges. Thus, if $\displaystyle \lim_{n \to \infty} \sum_{k = m}^n a_k = \lim_{n \to \infty} \sum_{k = m}^n b_k$, then $\sum a_n$ and $\sum b_n$ both converge if those limits are finite and $\sum a_n$ and $\sum b_n$ both do not converge if those limits are not a real (finite) number -- proving the statement.
\end{proof}

\end{document}