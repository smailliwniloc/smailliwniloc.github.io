\documentclass[10pt,a4paper]{article}
\usepackage[utf8]{inputenc}
\usepackage[english]{babel}
\usepackage{csquotes}
\usepackage{amsmath}
\usepackage{amsfonts}
\usepackage{amssymb}
\usepackage{graphicx}
\usepackage[margin=0.5in]{geometry}
\usepackage{amsthm}
\usepackage{enumitem}
\usepackage{tikz}
\usetikzlibrary{calc}
\newtheorem{question}{Question}
\newtheorem*{question*}{Question}
\newtheorem{theorem}{Theorem}
\newtheorem*{theorem*}{Theorem}
\newtheorem{lemma}{Lemma}

\theoremstyle{definition}
\newtheorem{answer}{Answer}
\newtheorem*{answer*}{Answer}


\title{Complex Analysis Homework 5}
\author{Colin Williams}

\begin{document}
\maketitle

\section*{Question 2}
\begin{enumerate}[label = (\alph*)]
\item Consider $z_1 = -1-i$ and $z_2 = -1$. Show that if all powers involved are principal values, then 
\begin{align*}
(z_1z_2)^i \neq z_1^iz_2^i
\end{align*}
\item Let $c_1, c_2$ and $z \neq 0$ be complex numbers. Prove that if all powers involved are principal values, then
\begin{align*}
z^{c_1}z^{c_2} = z^{c_1 + c_2}
\end{align*}
\end{enumerate}

First, note the definition of the power function for $c, w$ complex numbers with $w \neq 0$:
\begin{align*}
w^c = e^{c \log(w)}
\end{align*}
If we are using principal values, then we replace $\log(w)$ with Log$(w) = \ln(r) + i\theta$ for $w = re^{i\theta}$ and $r > 0$ and $-\pi < \theta \leq \pi$.
\begin{answer*}{\textbf{(a)}}
\\On the one hand, we can calculate $(z_1z_2)^i$ in the following manner:
\begin{align*}
(z_1z_2)^i &= \left[(-1-i)(-i)\right]^i\\
&= \left[i + i^2\right]^i\\
&= [-1 + i]^i\\
&= e^{\displaystyle i\text{Log}(-1 + i)}\\
&= e^{\displaystyle i\left(\ln(\sqrt{2}) + i\left(\frac{3\pi}{4}\right)\right)} &\text{since } -1 + i = \sqrt{2}e^{\displaystyle i \frac{3 \pi}{4}}\\
&= e^{\displaystyle i \ln(\sqrt{2})}e^{\displaystyle -\frac{3\pi}{4}}\\
&= e^{\displaystyle -\frac{3\pi}{4}}\left(\cos(\ln(\sqrt{2})) + i\sin(\ln(\sqrt{2}))\right)
\end{align*}
On the other hand, when we calculate $z_1^iz_2^i$, we get:
\begin{align*}
z_1^iz_2^i &= (-1-i)^i(-i)^i\\
&= e^{\displaystyle i \text{Log}(-1-i)}e^{\displaystyle i \text{Log}(-i)}\\
&= e^{\displaystyle i\left(\ln(\sqrt{2}) + i\left(-\frac{3\pi}{4}\right)\right)}e^{\displaystyle i\left(\ln(1) + i\left(-\frac{\pi}{2}\right)\right)} &\text{since} -1-i = \sqrt{2}e^{\displaystyle -i\frac{3\pi}{4}} \text{ and } -i = e^{\displaystyle -i \frac{\pi}{2}}\\
&= e^{\displaystyle i \ln(\sqrt{2})}e^{\displaystyle \left( \frac{3\pi}{4} + \frac{\pi}{2}\right)}\\
&= e^{\displaystyle \frac{5\pi}{4}}\left(\cos(\ln(\sqrt{2})) + i\sin(\ln(\sqrt{2}))\right)
\end{align*}

However, since $e^{5\pi / 4} \neq e^{-3\pi / 4}$ these two numbers do not have the same magnitude, so they obviously cannot be equal to one another. 
\end{answer*}

\begin{proof}{\textbf{(b)}}
\\Let $c_1, c_2, z \in \mathbb{C}$ and let $z \neq 0$. Therefore, we have
\begin{align*}
z^{c_1}z^{c_2} &= e^{c_1 \text{Log}(z)}e^{c_2 \text{Log}(z)}\\
&= e^{c_1 \text{Log}(z) + c_2 \text{Log}(z)}\\
&= e^{(c_1 + c_2)\text{Log}(z)}\\
&= z^{c_1 + c_2}
\end{align*}
Thus, the statement is proven
\end{proof}

\end{document}