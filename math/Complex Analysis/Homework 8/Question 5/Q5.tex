\documentclass[10pt,a4paper]{article}
\usepackage[utf8]{inputenc}
\usepackage[english]{babel}
\usepackage{csquotes}
\usepackage{amsmath}
\usepackage{amsfonts}
\usepackage{amssymb}
\usepackage{graphicx}
\usepackage[margin=0.5in]{geometry}
\usepackage{amsthm}
\usepackage{enumitem}
\usepackage{tikz}
\usetikzlibrary{calc}
\newtheorem{question}{Question}
\newtheorem*{question*}{Question}
\newtheorem{theorem}{Theorem}
\newtheorem*{theorem*}{Theorem}
\newtheorem{lemma}{Lemma}

\theoremstyle{definition}
\newtheorem{answer}{Answer}
\newtheorem*{answer*}{Answer}


\title{Complex Analysis Homework 8}
\author{Colin Williams}

\begin{document}
\maketitle

\section*{Question 5}
Suppose that $z_n$ is a sequence of distinct points in $D(0,1)$ such that $z_n \to 0$ as $n \to \infty$. Using the Uniqueness Theorem show that:
\begin{enumerate}[label = (\alph*)]
\item If $f$ is analytic in $D(0,1)$ and $f(z_n) = \cos(z_n)$ for all $n \in \mathbb{N}$, then $f(z) = \cos(z)$ for all $z \in D(0,1)$.
	\begin{itemize}
	\item Since $z_n$ converges to 0 as $n \to \infty$, that means that the point 0 is a limit point of the set $S = \{z_n : n \in \mathbb{N}\}$. However, this set also happens to be the same set for which $z \in S$ implies that $f(z) = \cos(z)$. Therefore, by the Uniqueness Theorem, since the limit point of $S$, 0, is inside $D(0,1)$ and both $f$ and $\cos(\cdot)$ are analytic functions, then we can conclude that $f(z) = \cos(z)$ for all $z \in D(0,1)$.
	\end{itemize}
\item There is no analytic function $f$ defined on $D(0,1)$ such that $f(z_n) = 0$ when $n$ is even and such that $f(z_n) = z_n$ when $n$ is odd.
	\begin{itemize}
	\item Assume that such a function exists. First, consider the set $S_1 = \{z_n : n \in \mathbb{N}$ and $n$ is odd$\}$. This set still has a limit point of 0 since any subsequence of a convergent sequence has the same limit. Thus, 0 is a limit point of $S_1$. Furthermore, for any $z \in S_1$ we have that $f(z) = z$ by assumption. Thus, since $f$ and the identify function are both analytic and since $0 \in D(0,1)$, we can use the Uniqueness Theorem to conclude that $f(z) = z$ for all $z \in D(0,1)$. In particular, this implies that $f(z_2) = z_2$ and $f(z_4) = z_4$. Furthermore, by assumption, we know that $f(z_n) = 0$ for all even $n$ which means that $z_2 = z_4 = 0$. However, this contradicts the definition of the sequence which states that $z_n$ is a sequence of \underline{distinct} points in $D(0,1)$ which in particular means that $z_2 \neq z_4$. This means our assumption of the existence of such an $f$ was false, so no such $f$ exists. $\qed$
	\end{itemize}
\end{enumerate}

\end{document}