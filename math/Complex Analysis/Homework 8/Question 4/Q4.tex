\documentclass[10pt,a4paper]{article}
\usepackage[utf8]{inputenc}
\usepackage[english]{babel}
\usepackage{csquotes}
\usepackage{amsmath}
\usepackage{amsfonts}
\usepackage{amssymb}
\usepackage{graphicx}
\usepackage[margin=0.5in]{geometry}
\usepackage{amsthm}
\usepackage{enumitem}
\usepackage{tikz}
\usetikzlibrary{calc}
\newtheorem{question}{Question}
\newtheorem*{question*}{Question}
\newtheorem{theorem}{Theorem}
\newtheorem*{theorem*}{Theorem}
\newtheorem{lemma}{Lemma}

\theoremstyle{definition}
\newtheorem{answer}{Answer}
\newtheorem*{answer*}{Answer}


\title{Complex Analysis Homework 8}
\author{Colin Williams}

\begin{document}
\maketitle

\section*{Question 4}
Let $f$ be holomorphic in $\mathbb{C}$.
\begin{enumerate}[label = (\alph*)]
\item Prove that if $|f(z)| > M$ in $\mathbb{C}$, then $f$ is constant.
	\begin{itemize}
	\item Consider the function $g = 1/f$. Then, since $|f(z)| > M$ for all $z \in \mathbb{C}$, we know that $f(z) \neq 0$ for all $z \in \mathbb{C}$ which means that $g$ is also holomorphic in $\mathbb{C}$. Furthermore, we can see that $g$ is bounded since:
	\begin{align*}
	|g(z)| = \left|\frac{1}{f(z)}\right| = \frac{1}{|f(z)|} &< \frac{1}{M} &\text{for all $z \in \mathbb{C}$}
	\end{align*}
	\item Thus, Liouville's Theorem tells us that $g$ must be a constant function since it is bounded and entire. Therefore, if $g$ is constant, say $g(z) = C$ for all $z \in \mathbb{C}$, then we can say that $f(z) = 1/g(z) = 1/C$ for all $z \in \mathbb{C}$, so $f$ is constant as well. $\qed$
	\end{itemize}
\item Prove that if $e^f$ is bounded in $\mathbb{C}$, then $f$ is constant. 
	\begin{itemize}
	\item Recall that $g(z) = e^z$ is an entire function, thus, since $f$ is also entire, then the function $g \circ f = e^f$ is also entire. Thus, since we're assuming $e^f$ is bounded, then we can apply Liouville's Theorem to say that $e^f$ is constant. Thus, since the derivative of a constant is 0, we know that
	\begin{align*}
	\frac{d}{dz}\left(e^{f(z)}\right) = e^{f(z)}f'(z) &= 0 &\text{for all $z \in \mathbb{C}$}
	\end{align*}
	This last equality implies that $f'(z) = 0$ for all $z \in \mathbb{C}$ since the exponential function never maps to 0. Thus, $f$ must also be a constant function since its derivative is zero in all of $\mathbb{C}$.
	\end{itemize}
\end{enumerate}

\end{document}