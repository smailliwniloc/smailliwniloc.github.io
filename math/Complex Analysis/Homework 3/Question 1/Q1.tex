\documentclass[10pt,a4paper]{article}
\usepackage[utf8]{inputenc}
\usepackage[english]{babel}
\usepackage{amsmath}
\usepackage{amsfonts}
\usepackage{amssymb}
\usepackage{graphicx}
\usepackage[margin=0.5in]{geometry}
\usepackage{amsthm}
\usepackage{enumitem}
\newtheorem{question}{Question}
\newtheorem*{question*}{Question}
\newtheorem{answer}{Answer}
\newtheorem*{answer*}{Answer}
\newtheorem{theorem}{Theorem}
\newtheorem{lemma}{Lemma}

\title{Complex Analysis Homework 3}
\author{Colin Williams}

\begin{document}
\maketitle
\section*{Question 1}
\begin{question*}{$ $}
\\Define a function $f$ by 
\[f(z) = \frac{z}{1 + |z|}.\]
\begin{enumerate}[label = (\alph*)]
\item Prove that $f$ is continuous on $\mathbb{C}$.
\item Prove that $f(z_1) = f(z_2)$ implies that $z_1 = z_2$.
\item Prove that $f$ maps $\mathbb{C}$ onto $D(0,1)$.
\end{enumerate}

\begin{proof}{\textbf{(a)}}
\\To prove this function is continuous, I will use that we know the following
\begin{enumerate}
\item The sum of continuous function is continuous
\item The quotient of continuous function is continuous as long as the denominator is not equal to zero. 
\item A function $f$ is continuous at point $z_0 \in \mathbb{C}$ if for every $\varepsilon > 0$, there exists some $\delta > 0$ such that whenever $|z - z_0| < \delta$, we have $|f(z) - f(z_0)| < \varepsilon$.
\end{enumerate}
Thus, to prove that $f(z)$ is continuous, we merely need to show that $z$, $1$, and $|z|$ are continuous and that $1 + |z| \neq 0$. First, since $|z| \geq 0$ for all $z \in \mathbb{C}$, we know that $1 + |z| \geq 1$ for all $z \in \mathbb{C}$; thus, $1 + |z|$ is never equal to zero, so we need not worry about this.\\
\\Next, I will show that $g(z) = z$ is a continuous function by examining $|g(z) - g(z_0)|$ for some arbitrary $z_0 \in \mathbb{C}$. In fact, it is as simple as:
\begin{align*}
|g(z) - g(z_0)| &= |z - z_0|
\end{align*}
Thus, $|g(z) - g(z_0)| < \varepsilon$ for any $\varepsilon > 0$ precisely when $|z - z_0| < \delta$ for $\delta$ any number $ \leq \varepsilon$. Since this holds true for any $z_0 \in \mathbb{C}$, it is clear that $g(z) = z$ satisfies the above definition of continuity for all points in $\mathbb{C}$.\\
\\Similarly, I will show that $g(z) = 1$ is a continuous function by looking at $|g(z) - g(z_0)|$ for some arbitrary $z_0 \in \mathbb{C}$. However, it is clear that $|g(z) - g(z_0)| = |1 - 1| = |0| = 0$. Thus, $|g(z) - g(z_0)| < \varepsilon$ for any $\varepsilon > 0$ no matter how we choose $\delta$. Since $z_0$ can be any point in $\mathbb{C}$, $g(z) = 1$ clearly satisfies the definition of continuity for all points in $\mathbb{C}$.\\
\\Lastly, I will show that $g(z) = |z|$ is a continuous function. Again, take $z_0$ to be some arbitrary point in $\mathbb{C}$, then examine $|g(z) - g(z_0)|$ as follows:
\begin{align*}
|g(z) - g(z_0)| &= ||z| - |z_0||\\
&= ||z| - |-z_0|| &\text{since } |w| = |-w| \text{ for all } w \in \mathbb{C}\\
&\leq |z + (-z_0)| &\text{by the "reverse" Triangle Inequality}\\
&= |z - z_0|
\end{align*}
Thus, $|g(z) - g(z_0)| < \varepsilon$ for any $\varepsilon > 0$ precisely when $|z - z_0| < \delta$ for $\delta$ any number $\leq \varepsilon$. Since this is true for any $z_0 \in \mathbb{C}$, it is clear that $g(z) = |z|$ satisfies the above definition of a continuous function for all points in $\mathbb{C}$.\\
\\Thus, we have shown $z$, $1$, and $|z|$ are all continuous function; therefore, $1 + |z|$ is also a continuous function and since $1 + |z| \neq 0$ for all $z \in \mathbb{C}$, we must have that \[f(z) = \frac{z}{1 + |z|}\] is a continuous function as well. 
\end{proof}

\begin{proof}{\textbf{(b)}}
\\Assume that two points $z_1, z_2 \in \mathbb{C}$ satisfy $f(z_1) = f(z_2)$. It will be convenient to express $z_1$ and $z_2$ in polar coordinates (exponential form), say $z_1 = r_1e^{i\theta_1}$ and $z_2 = r_2e^{i\theta_2}$ for $r_1, r_2 > 0$ and $\theta_1, \theta_2$ the argument of $z_1$ and $z_2$ respectively; thus, $f(r_1e^{i\theta_1}) = f(r_2e^{i\theta_2})$. This gives:
\begin{align*}
\frac{r_1e^{i\theta_1}}{1 + |r_1e^{i\theta_1}|} &= \frac{r_2e^{i\theta_2}}{1 + |r_2e^{i\theta_2}|}\\
\implies \frac{r_1e^{i\theta_1}}{1 + r_1} &= \frac{r_2e^{i\theta_2}}{1 + r_2} &\text{since } |re^{i\theta}| = r\\
\implies \frac{r_1}{1 + r_1}e^{i\theta_1} &= \frac{r_2}{1 + r_2}e^{i\theta_2}\\
\implies \frac{r_1}{1 + r_1} = \frac{r_2}{1 + r_2} &\text{ and } \theta_1 = \theta_2 + 2k\pi &\text{by definition of equality of complex numbers in polar form}\\
\end{align*}
Working with the left equality, we can see that $r_1 + r_1r_2 = r_2 + r_2r_1$ by cross multiplying the fractions. Thus, by subtracting $r_1r_2$, we get $r_1 = r_2$. This, together with $\theta_1 = \theta_2 + 2k\pi$ implies that $z_1 = z_2$ by definition of equality of complex numbers in polar form. Thus, $f(z_1) = f(z_2) \implies z_1 = z_2$ just as desired.
\end{proof}

\begin{proof}{\textbf{(c)}}
\\We know that for any $z_0 \in D(0,1)$, we can express $z_0$ in polar form as $z_0 = r_0e^{i\theta_0}$ where $0< r_0 < 1$ and $\theta_0$ is the argument of $z_0$. From part (b), we showed that for any $z = re^{i\theta}$, 
\begin{equation}
f(z) = \frac{r}{1 + r}e^{i\theta}
\end{equation}
Thus, $f(z) = z_0$ whenever
\[\frac{r}{1 + r} = r_0 \quad \text{and} \quad \theta = \theta_0 + 2k\pi\]
For simplicity, we can choose $\theta = \theta_0$, and then we solve the left equality:
\begin{align*}
\frac{r}{1 + r} &= r_0\\
\implies r &= r_0(1 + r)\\
\implies r &= r_0 + r \cdot r_0\\
\implies r - r \cdot r_0 &= r_0\\
\implies r(1 - r_0) &= r_0\\
\implies r &= \frac{r_0}{1 - r_0}
\end{align*}
Thus, we have shown that every $z_0 = r_0e^{i\theta_0} \in D(0,1)$ has a pre-image given by $\displaystyle z_0^* = \frac{r_0}{1 - r_0}e^{i\theta_0}$ such that $f(z_0^*) = z_0$. To complete the picture note that for $f(z)$ defined in (1), the modulus, given by $\displaystyle \frac{r}{1 + r}$, is always between 0 and 1 since $0 < r < 1 + r$. Thus, for every $z \in \mathbb{C}$, $f(z) \in D(0,1)$ and for every $z_0 \in D(0,1)$, we have shown that there exists a $z_0^*$ such that $f(z_0^*) = z_0$. Therefore, we have shown that $f$ does indeed map every element of $\mathbb{C}$ onto $D(0,1)$.
\end{proof}
\end{question*}


\end{document}