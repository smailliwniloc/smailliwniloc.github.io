\documentclass[10pt,a4paper]{article}
\usepackage[utf8]{inputenc}
\usepackage[english]{babel}
\usepackage{amsmath}
\usepackage{amsfonts}
\usepackage{amssymb}
\usepackage{graphicx}
\usepackage[margin=0.5in]{geometry}
\usepackage{amsthm}
\usepackage{enumitem}
\newtheorem{question}{Question}
\newtheorem*{question*}{Question}
\newtheorem{answer}{Answer}
\newtheorem*{answer*}{Answer}
\newtheorem{theorem}{Theorem}
\newtheorem{lemma}{Lemma}

\title{Complex Analysis Homework 3}
\author{Colin Williams}

\begin{document}
\maketitle
\section*{Question 5}

\begin{question*}{$ $}
\\Prove that $f$ defined by
\[f(z) = \sqrt{|\text{Re}(z)\text{Im}(z)|}\]
satisfies the Cauchy-Riemann Equations at $z = 0$, but is not differentiable there. 
\end{question*}
$ $\\First, recall that the Cauchy Riemann Equations for $f(z) = u(x,y)+ iv(x,y)$ are the following:
\begin{align*}
u_x(x, y) = v_y(x, y) && v_x(x,y) = -u_y(x,y)
\end{align*}\\
Next, recall that for $f(z)$ to be differentiable at $z = 0$, the following limit must exist:
\[\lim_{z \to 0}\frac{f(z) - f(0)}{z - 0}\]
\begin{proof}{$ $}
\\If $z = x + iy$, then $f(z) = \sqrt{|\text{Re}(z)\text{Im}(z)|}$ reduces to $f(z) = \sqrt{|xy|}$. Thus, $u(x,y) = \sqrt{|xy|}$ and $v(x,y) = 0$. First, I will calculate $u_x(0,0)$:
\begin{align*}
u_x(0,0) &= \lim_{\Delta x \to 0}\frac{u(\Delta x,0) - u(0,0)}{\Delta x} &\text{by the definition of a partial derivative}\\
&= \lim_{\Delta x \to 0}\frac{\sqrt{|\Delta x \cdot 0|} - \sqrt{|0\cdot 0|}}{\Delta x}\\
&= \lim_{\Delta x \to 0}\frac{0}{\Delta x} = \lim_{\Delta x \to 0} 0 = 0.
\end{align*}
Similarly, I will calculate $u_y(0,0)$:
\begin{align*}
u_y(0,0) &= \lim_{\Delta y \to 0}\frac{u(0, \Delta y) - u(0,0)}{\Delta y} &\text{by the definition of a partial derivative}\\
&= \lim_{\Delta y \to 0}\frac{\sqrt{|0 \cdot \Delta y|} - \sqrt{|0\cdot 0|}}{\Delta y}\\
&= \lim_{\Delta y \to 0}\frac{0}{\Delta y} = \lim_{\Delta y \to 0} 0 = 0.
\end{align*}
Thus, it is clear that $u_x(0,0) = 0 = v_y(0,0)$ and $v_x(0,0) = 0 = -u_y(0,0)$, so the Cauchy-Riemann Equations are satisfied at the point $z = 0$. Next, I will examine the differentiability of $f$ at this point by seeing if the required limit exists:
\begin{align*}
\lim_{z \to 0}\frac{f(z) - f(0)}{z - 0} &= \lim_{z \to 0}\frac{\sqrt{|xy|} - \sqrt{|0 \cdot 0|}}{x + iy}\\
&= \lim_{z \to 0}\frac{\sqrt{|xy|}}{x + iy}\\
\end{align*}
Let's first look at this limit as $z$ approaches 0 along the line Im($z) = 0$, i.e. $z$ is of the form $z = x + i(0) = x$, a purely Real Number.
\begin{align*}
\lim_{z \to 0}\frac{\sqrt{|xy|}}{x + iy} &= \lim_{x \to 0}\frac{\sqrt{|x \cdot 0|}}{x + i(0)}\\
&= \lim_{x \to 0}\frac{0}{x} = \lim_{x \to 0} 0 = 0.
\end{align*}
On the other hand, let's look at this limit along the line Re($z) = \text{Im}(z)$ with Re($z) \geq 0$. In other words, $z = x + ix$ for $x \geq 0$. 
\begin{align*}
\lim_{z \to 0}\frac{\sqrt{|xy|}}{x + iy} &= \lim_{x \to 0}\frac{\sqrt{|x \cdot x|}}{x + ix}\\
&= \lim_{x \to 0}\frac{\sqrt{|x^2|}}{x + ix}\\
&= \lim_{x \to 0}\frac{\sqrt{x^2}}{x + ix} &\text{since for all } x \in \mathbb{R}, x^2 \geq 0.\\
&= \lim_{x \to 0}\frac{|x|}{x + ix}\\
&= \lim_{x \to 0}\frac{x}{x(1 + i)} &\text{since } x > 0 \implies |x| = x\\
&= \lim_{x \to 0}\frac{1}{1 + i} = \frac{1}{1+i}
\end{align*}
Thus, we see that in one case, this limit is equal to $0$, and in another case, this limit is equal to $\displaystyle \frac{1}{1+i} \text{ and } 0 \neq \frac{1}{1 + i}$, so the limit does not exist. This means that $f(z)$ must not be differentiable at $z = 0$. 
\end{proof}

\end{document}