\documentclass[10pt,a4paper]{article}
\usepackage[utf8]{inputenc}
\usepackage[english]{babel}
\usepackage{csquotes}
\usepackage{amsmath}
\usepackage{amsfonts}
\usepackage{amssymb}
\usepackage{graphicx}
\usepackage[margin=0.5in]{geometry}
\usepackage{amsthm}
\usepackage{enumitem}
\usepackage{tikz}
\usetikzlibrary{calc}
\newtheorem{question}{Question}
\newtheorem*{question*}{Question}
\newtheorem{theorem}{Theorem}
\newtheorem*{theorem*}{Theorem}
\newtheorem{lemma}{Lemma}

\theoremstyle{definition}
\newtheorem{answer}{Answer}
\newtheorem*{answer*}{Answer}


\title{Complex Analysis Homework 4}
\author{Colin Williams}

\begin{document}
\maketitle

\section*{Question 4}

Use the estimation theorem to obtain the following upper bounds:
\begin{enumerate}[label = (\alph*)]
\item $\displaystyle \left|\int_{\gamma}\frac{dz}{z^2 - 1}\right| \leq \frac{\pi}{3}$
\\where $\gamma$ is the arc of the circle $|z| = 2$ from $z = 2$ to $z = 2i$ in the first quadrant
\item $\displaystyle \left|\int_{\gamma(0;R)} \frac{z-1}{z+1}dz\right| \leq \frac{2\pi R(R + 1)}{R-1}$
\\where $\gamma(0;R)$ denotes the circular contour with center 0 and radius $R > 1$.
\end{enumerate}
$ $
\\In order to provide these estimates, I will use the result that given some path $\gamma : [a,b] \to \mathbb{C}$ and some continuous function $f: \gamma^* \to \mathbb{C}$. Suppose there exists some $M \geq 0$ such that $|f(z)| \leq M$ for all $z \in \gamma^*$, then
\begin{equation}
\left|\int_{\gamma} f(z)dz\right| \leq M \cdot L(\gamma)
\end{equation}
for $L(\gamma)$ the length of the path $\gamma$.

\begin{answer*}{\textbf{(a)}}
\\First, we need to find our upper bound $M$ for $\displaystyle |f(z)| = \left|\frac{1}{z^2 - 1}\right|$ with $z \in \gamma^*$ the image of $\gamma = 2e^{it}, t \in [0, \frac{\pi}{2}]$:
\begin{align*}
|f(z)| &= \left|\frac{1}{z^2 - 1}\right|\\
&= \frac{1}{|z^2 - 1|}\\
&\leq \frac{1}{\displaystyle \left||z^2| - |-1|\right|}&\text{by reverse triangle inequality}\\
&= \frac{1}{\displaystyle \left||z|^2 - 1\right|}\\
&= \frac{1}{|2^2 - 1|} &\text{since } z \in \gamma^* \text{ has modulus 2}\\
&= \frac{1}{3}
\end{align*}
Therefore we will define our chosen upper bound $M$ to be $\displaystyle \frac{1}{3}$. Next, we will calculate $L(\gamma)$. First, note the Circumference of a circle is given by the formula $C = 2r\pi$ for $r$ the radius of the circle. Thus, since our $\gamma$ is one-fourth of the circumference of the circle with radius 2, we get that $\displaystyle L(\gamma) = \frac{2(2)\pi}{4} = \pi$. Therefore,
\[\boxed{\left|\int_{\gamma} \frac{dz}{z^2 - 1}\right| \leq \frac{\pi}{3}}\]
\end{answer*}

\begin{answer*}{\textbf{(b)}}
\\First, we need to find our upper bound $M$ for $\displaystyle |f(z)| = \left|\frac{z-1}{z + 1}\right|$ with $z \in \gamma^*$ the image of $\gamma(0;R) = Re^{it}, t \in [0, 2\pi]$:
\begin{align*}
|f(z)| = \left|\frac{z-1}{z + 1}\right|
&= \frac{|z - 1|}{|z + 1|}\\
&\leq \frac{|z| + |-1|}{|z + 1|} &\text{by the triangle inequality}\\
&\leq \frac{|z| + 1}{\displaystyle \left| |z| - |1| \right|} &\text{by the reverse triangle inequality}\\
&= \frac{R + 1}{|R - 1|} &\text{since } z \in \gamma^* \text{ has modulus }R\\
&= \frac{R + 1}{R - 1} &\text{since } R>1\\
\end{align*}

Thus, we can define our upper bound $M$ to be this above fraction. Furthermore, since $\gamma$ is simply a circle with radius $R$, we know that $L(\gamma) = 2\pi R$. Therefore,
\[\boxed{\left|\int_{\gamma} \frac{z - 1}{z + 1}dz\right| \leq \frac{2\pi R(R+1)}{R-1}}\]
\end{answer*}

\end{document}