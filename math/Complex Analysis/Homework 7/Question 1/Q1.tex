\documentclass[10pt,a4paper]{article}
\usepackage[utf8]{inputenc}
\usepackage[english]{babel}
\usepackage{csquotes}
\usepackage{amsmath}
\usepackage{amsfonts}
\usepackage{amssymb}
\usepackage{graphicx}
\usepackage[margin=0.5in]{geometry}
\usepackage{amsthm}
\usepackage{enumitem}
\usepackage{tikz}
\usetikzlibrary{calc}
\newtheorem{question}{Question}
\newtheorem*{question*}{Question}
\newtheorem{theorem}{Theorem}
\newtheorem*{theorem*}{Theorem}
\newtheorem{lemma}{Lemma}

\theoremstyle{definition}
\newtheorem{answer}{Answer}
\newtheorem*{answer*}{Answer}


\title{Complex Analysis Homework 7}
\author{Colin Williams}

\begin{document}
\maketitle

\section*{Question 1}
Write down an expansion of the form $\displaystyle \sum_{n = 0}^{\infty} a_nz^n$ for the following:
\begin{enumerate}[label = (\alph*)]
\item $\displaystyle \frac{1}{(1 + z)^3}$,
\item $\displaystyle ze^{z^2}$.
\end{enumerate}
In each case, specify where the expansion is valid. 

\begin{answer*}{\textbf{(a)}}
\\Examine the following function:
\[f(z) := \frac{1}{1 + z}\]
By taking derivatives, we can see that,
\begin{align*}
f'(z) &= \frac{-1}{(1 + z)^2}\\
f''(z) &= \frac{2}{(1 + z)^3}
\end{align*}
Also, note that by using the fact that $\displaystyle \frac{1}{1 - w} = \sum_{n = 0}^{\infty} w^n$ for all $|w| < 1$. Thus,
\[f(z) = \frac{1}{1 + z} = \frac{1}{1 - (-z)} = \sum_{n =0}^{\infty} (-z)^n = \sum_{n = 0}^{\infty}(-1)^n z^n \quad \text{for all $|z| < 1$}\]

The function we are interested in is $\displaystyle \frac{1}{(1 + z)^3} = \frac{1}{2}f''(z)$ for $f$ defined above. Therefore,
\begin{align*}
\frac{1}{(1 + z)^3} = \frac{1}{2} \frac{d^2}{dz^2} \left(\frac{1}{1 + z}\right) &= \frac{1}{2} \frac{d^2}{dz^2} \left(\sum_{n = 0}^{\infty} (-1)^n z^n\right) &\text{by the examination above}\\
&= \frac{1}{2} \sum_{n = 0}^{\infty} \frac{d^2}{dz^2}\left((-1)^n z^n\right)\\
&= \frac{1}{2}\sum_{n = 1}^{\infty} \frac{d}{dz}\left(n(-1)^nz^{n - 1}\right)\\
&= \frac{1}{2} \sum_{n = 2}^{\infty} (-1)^n n(n - 1)z^{n-2}\\
&= \sum_{n = 0}^{\infty} \frac{(-1)^{n + 2}}{2}(n + 2)(n + 1)z^n\\
&= \sum_{n = 0}^{\infty} \frac{(-1)^{n}}{2}(n + 1)(n + 2)z^n\\
\end{align*}
The first series was valid whenever $|z| < 1$ and since the derivative of a power series has the same radius of convergence, our final series \boxed{\text{converges  for } |z| < 1} and for \boxed{a_n = \frac{(-1)^{n}}{2}(n + 1)(n + 2)}.
\end{answer*}

\begin{answer*}{\textbf{(b)}}
\\Recall the series expansion for $e^w$:
\[e^w = \sum_{n = 0}^{\infty} \frac{1}{n!}w^n \quad \text{for all $z \in \mathbb{C}$}\]
Thus, we can also find an expansion for $f(z) := e^{z^2}$:
\[f(z) = e^{z^2} = \sum_{n = 0}^{\infty} \frac{1}{n!}(z^2)^n = \sum_{n = 0}^{\infty} \frac{1}{n!}z^{2n}\]
Additionally, we can note that $f'(z) = 2ze^{z^2}$. Thus, the function we are interested in is $ze^{z^2} = \frac{1}{2}f'(z)$, so we can find it's expansion by doing the following:
\begin{align*}
ze^{z^2} = \frac{1}{2} \frac{d}{dz}\left(e^{z^2}\right) &= \frac{1}{2} \frac{d}{dz}\left(\sum_{n = 0}^{\infty} \frac{1}{n!}z^{2n}\right) &\text{by the examination above}\\
&= \frac{1}{2} \sum_{n = 0}^{\infty} \frac{d}{dz}\left(\frac{1}{n!}z^{2n}\right)\\
&= \frac{1}{2}\sum_{n = 1}^{\infty} \frac{2n}{n!}z^{2n - 1}\\
&= \sum_{n = 1}^{\infty} \frac{1}{(n - 1)!}z^{2n - 1}\\
&= \sum_{n =0}^{\infty} \frac{1}{n!}z^{2n + 1}
\end{align*}
Since the first series was valid for all $z \in \mathbb{C}$ and the fact that the derivative of a power series has the same radius of convergence, we know that this final series \boxed{\text{converges for all } z \in \mathbb{C}}. Note this isn't quite the form we want for this question, so I will make the following definition:
\[
a_n = \begin{cases}
\frac{1}{k!} &\text{if $n$ is an odd number of the form $n = 2k + 1$ for $k \in \mathbb{N}$}\\
0 &\text{if $n$ is even}
\end{cases}
\]
Thus, the above series is in the desired form of $\displaystyle \sum_{n = 0}^{\infty} a_nz^n$ with the same convergence as discussed above. 
\end{answer*}

\end{document}