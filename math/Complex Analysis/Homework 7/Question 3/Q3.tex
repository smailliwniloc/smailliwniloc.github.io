\documentclass[10pt,a4paper]{article}
\usepackage[utf8]{inputenc}
\usepackage[english]{babel}
\usepackage{csquotes}
\usepackage{amsmath}
\usepackage{amsfonts}
\usepackage{amssymb}
\usepackage{graphicx}
\usepackage[margin=0.5in]{geometry}
\usepackage{amsthm}
\usepackage{enumitem}
\usepackage{tikz}
\usetikzlibrary{calc}
\newtheorem{question}{Question}
\newtheorem*{question*}{Question}
\newtheorem{theorem}{Theorem}
\newtheorem*{theorem*}{Theorem}
\newtheorem{lemma}{Lemma}

\theoremstyle{definition}
\newtheorem{answer}{Answer}
\newtheorem*{answer*}{Answer}


\title{Complex Analysis Homework 7}
\author{Colin Williams}

\begin{document}
\maketitle

\section*{Question 3}
Use the definition to show that the sequence of functions $\displaystyle f_n(z) = \frac{1}{nz}$ is pointwise convergent, but not uniformly convergent, to $f(z) = 0$ on the domain $\Omega = D(0,1)\backslash \{0\}$.

\begin{proof}
I will first show that $f_n(z)$ is pointwise convergent. Let $z \in D(0,1) \backslash \{0\}$ and $\varepsilon > 0$ be fixed. Let us examine $|f_n(z) - f(z)| = |f_n(z)|$:
\begin{align*}
|f_n(z)| &= \left|\frac{1}{nz}\right|\\
&= \frac{1}{n|z|}
\end{align*}
Thus, if we define $N_{\varepsilon}(z) := \frac{1}{\varepsilon |z|}$, then for $n > N_{\varepsilon}(z)$ we have the following:
\begin{align*}
|f_n(z) - f(z)| = |f_n(z)| &= \frac{1}{n|z|}\\
&< \frac{1}{N_{\varepsilon}(z)|z|}\\
&= \frac{1}{|z|/(\varepsilon |z|)}\\
&= \varepsilon
\end{align*}
This shows that $f_n$ is pointwise convergent to $f(z) = 0$. However, if we look at our choice for $N_{\varepsilon}(z)$, we see that it has no upper bound because as $|z| \to 0$, then $N_{\varepsilon}(z) \to \infty$. Thus, it is impossible to find an $N_{\varepsilon}$ that does not depend on the specific point $z$, so $f_n$ does not converge uniformly to $f(z) = 0$.
\end{proof}


\end{document}