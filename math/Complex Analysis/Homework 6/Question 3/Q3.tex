\documentclass[10pt,a4paper]{article}
\usepackage[utf8]{inputenc}
\usepackage[english]{babel}
\usepackage{csquotes}
\usepackage{amsmath}
\usepackage{amsfonts}
\usepackage{amssymb}
\usepackage{graphicx}
\usepackage[margin=0.5in]{geometry}
\usepackage{amsthm}
\usepackage{enumitem}
\usepackage{tikz}
\usetikzlibrary{calc}
\newtheorem{question}{Question}
\newtheorem*{question*}{Question}
\newtheorem{theorem}{Theorem}
\newtheorem*{theorem*}{Theorem}
\newtheorem{lemma}{Lemma}

\theoremstyle{definition}
\newtheorem{answer}{Answer}
\newtheorem*{answer*}{Answer}


\title{Complex Analysis Homework 6}
\author{Colin Williams}

\begin{document}
\maketitle

\section*{Question 3}
Let $f$ be analytic in a domain $G$. Let $g$ be defined by $g(z) = e^{f(z)}$. Prove that if $g$ is constant, then $f$ is constant. 

\begin{proof}{$ $ }
\\First, if we let $h(z) = e^z$ (which we know to be analytic everywhere, in particular analytic in $G$), then it is clear that $g(z) = (h \circ f)(z) = h(f(z))$. Thus, since $g$ is the composition of two functions who are analytic in $G$, then $g$ must also be analytic in $G$.
\\
\\Thus, for every point $z \in G$, $g'(z) = h'(f(z))\cdot f'(z) = e^{f(z)}f'(z)$ by the chain rule. Furthermore, from the assumption that $g$ is constant, we know that $g'(z) = 0$ for all $z$. Thus, we have that $e^{f(z)}f'(z) = 0$ for all $z \in G$. Additionally, since we know that $e^w \neq 0$ for all $w \in \mathbb{C}$, we can conclude that $f'(z) = 0$ for all $z \in G$. Finally, from the Theorem at the beginning of Section 25 (on page 73), we can conclude that since $f'(z) = 0$ for all $z \in G$, then $f(z)$ must be constant throughout $G$.

\end{proof}

\end{document}