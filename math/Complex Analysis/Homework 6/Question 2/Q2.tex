\documentclass[10pt,a4paper]{article}
\usepackage[utf8]{inputenc}
\usepackage[english]{babel}
\usepackage{csquotes}
\usepackage{amsmath}
\usepackage{amsfonts}
\usepackage{amssymb}
\usepackage{graphicx}
\usepackage[margin=0.5in]{geometry}
\usepackage{amsthm}
\usepackage{enumitem}
\usepackage{tikz}
\usetikzlibrary{calc}
\newtheorem{question}{Question}
\newtheorem*{question*}{Question}
\newtheorem{theorem}{Theorem}
\newtheorem*{theorem*}{Theorem}
\newtheorem{lemma}{Lemma}

\theoremstyle{definition}
\newtheorem{answer}{Answer}
\newtheorem*{answer*}{Answer}


\title{Complex Analysis Homework 6}
\author{Colin Williams}

\begin{document}
\maketitle

\section*{Question 2}
Solve the equation $\cos(z) = 2$.

\begin{answer*}{$ $}
\\I will start by using the identity that for $z = x + iy$, $\cos(z) = \cos(x)\cosh(y) - i\sin(x)\sinh(y)$. From this we get,
\begin{align*}
\cos(z) &= 2\\
\iff \cos(x)\cosh(y) - i\sin(x)\sinh(y) &= 2\\
\iff
\begin{cases}
    \cos(x)\cosh(y) = 2\\
    \sin(x)\sinh(y) = 0
\end{cases} \quad \text{AND}
\end{align*}
I will proceed by examining the second of these equalities:
\begin{align*}
\sin(x)\sinh(y) = 0\\
\iff
\begin{cases}
	\sin(x) = 0\\
	\sinh(y) = 0
\end{cases} \quad \text{OR}\\
\iff
\begin{cases}
	x = k\pi, k \in \mathbb{Z}\\
	y = 0 
\end{cases} \quad \text{OR}\\
\end{align*}
Let's first consider what would happen if we allow $y = 0$. This would cause our other equation, $\cos(x)\cosh(y) = 2$ to be equivalent to $\cos(x)\cosh(0) = \cos(x) = 2$. However, $\cos(x) = 2$ has no solutions in $\mathbb{R}$ so this is not possible. Therefore, we can only have the possibility that $x = k\pi, k \in \mathbb{Z}$. From this, our other equation becomes $\cos(k\pi)\cosh(y) = (-1)^k\cosh(y) = 2$. This is equivalent to saying $\cosh(y) = 2\cdot (-1)^k$. However, we know that $\cosh(y) > 0$ for all $y \in \mathbb{R}$, so we must have that $(-1)^k$ is positive, i.e. that $k$ is even, say $k = 2n$ for $n \in \mathbb{Z}$. Thus, our solution is $x = 2n\pi$ and $y = \cosh^{-1}(2)$, i.e. \boxed{z = 2n\pi + i\cosh^{-1}(2) \text{ for } n \in \mathbb{Z}}
\end{answer*}
\end{document}